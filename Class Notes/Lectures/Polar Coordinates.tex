\newpage
\section{Polar Coordinates}

\subsection{Laplace's Equation in Polar Coordinates}

Let's say we want to solve $\Delta u = 0$ with Dirichlet Boundary Conditions on a disk or annulus.

Problem: $\Delta u = u_{xx} + u_{yy} \leftarrow $ in terms of $x$ and $y$.

We must find it in terms of $r$ and $\theta$.
%
\begin{align}
  u(x, y) \rightarrow u(r, \theta)\\
  x & = r \cos \theta\\
  y & = r \sin \theta\\
  r & = \sqrt{x^2 + y^2}\\
  \theta & = \arctan \frac{y}{x}\\
  \tan \theta & = \frac{y}{x}
\end{align}

We are going to find : $u_x, u_{xx}, u_{yy}$
%
\begin{enumerate}
  \item $u_x$ :
  %
  \begin{align}
    u(x, y) & = u( x(r, \theta), y(r, \theta)) = u(r, \theta) = u(r(x, y), \theta(x, y))
  \end{align}
  Here, we break our chain rule as the following: % TODO
  \begin{center}
    $u$\\
    $|$ \quad \quad $|$\\
    $r$ \quad \quad $\theta$\\
    $|$ \quad $|$ \quad $|$ \quad $|$\\
    $x$ \quad $y$ \quad $x$ \quad $y$
  \end{center}

  According to our tree, we have two routes.
  %
  \begin{align}
    \frac{\p u}{\p x} & = \frac{\p u}{\p r} \frac{\p r}{\p x} + \frac{\p u}{\p \theta} \frac{\p \theta}{\p x}\\
    & = u_r \frac{x}{\sqrt{x^2 + y^2}} + u_\theta \frac{- \frac{y}{x^2}}{1 + \left( \frac{y}{x}\right)^2}
  \end{align}

  Note that we know $r$ from line 370. We can rewrite $r$ as $(x^2 + y^2)^{\frac{1}{2}}$.

  Now, let us multiply our equation by $\frac{x^2}{x^2}$.
  %
  \begin{align}
    & = u_r \frac{x}{\sqrt{x^2 + y^2}} + u_\theta \frac{- \frac{y}{x^2}}{1 + \left( \frac{y}{x}\right)^2} \cdot \frac{x^2}{x^2}\\
    & = u_r \frac{r \cos\theta}{r} - u_\theta \frac{r \sin \theta}{r^2}\\
    & = u_r \cos\theta - u_\theta \frac{\sin \theta}{r}
  \end{align}
\end{enumerate}

%

\topic{February 14, 2021}

Idea: What if we are on a disk?
%
\begin{align}
  \Delta u  & = 0 \Rightarrow u_{xx} + u_{yy} = 0\\
  u_x       & = u_r \cos \theta u_\theta \frac{\sin \theta}{r}\\
  u_{xx}
  & = \frac{\p}{\p x} [u_x]\\
  & = \frac{\p u_x}{\p r} \frac{\p r}{\p x} + \frac{\p u_x}{\p \theta} \frac{\p \theta}{\p x}\\
  & =
  \frac{\p}{\p r}
  \left[u_r \cos \theta -u_\theta \frac{\sin \theta}{r}\right] +
  \frac{\p}{\p \theta}
  \left[ u_r \cos \theta - u_\theta \frac{\sin \theta}{r} \right]
  \left[ - \frac{\sin \theta}{r}\right]\\
  & =
  \left[ u_{rr} \cos \theta - u_{\theta r} \frac{\sin \theta}{r} + u_\theta \frac{\sin \theta}{r^2} \right] \cos \theta +
  \left[ u_{r \theta} \cos \theta - u_r \sin \theta - u_{\theta \theta} \frac{\sin \theta}{r} - u_\theta \frac{\cos \theta}{r} \right]
  \frac{\sin \theta}{r}\\
  & =
  u_{rr} \cos^2 \theta
  - 2 u_{\theta r} \frac{\sin \theta \cos \theta}{r}
  + 2 u_\theta \frac{\sin \theta \cos \theta}{r^2}
  + u_r \frac{\sin^2 \theta}{r} + u_{\theta \theta} \frac{\sin^2 \theta}{r^2}\\
  u_{yy} &
  = u_{rr} \sin^2 \theta
  + 2u_{\theta r} \frac{\sin \theta \cos \theta}{r}
  - 2u_\theta \frac{\sin \theta \cos \theta}{r^2}
  + u_r \frac{\cos^2 \theta}{r}
  + u_{\theta \theta} \frac{\cos^2 \theta}{r^2}\\
  \Delta u & = u_{xx} + u_{yy} \\ &
  = u_{rr}
  + \frac{u_r}{r}
  + \frac{u_{\theta \theta}}{r^2} = 0
\end{align}

\subsection{Solving Laplace's Equation in Polar Coordinates}

If we have an open disk, akin to a washer, we have two boundaries: The innter boundary and outer boundary.
%
\begin{enumerate}
  \item Assume $u(r, \theta) = R(r) \Theta(\theta)$
  %
  \begin{align}
    u_{rr} + \frac{u_r}{r} + \frac{u_{\theta\theta}}{r^2} & = 0\\
    R^{\prime\prime}\Theta + \frac{R^\prime \Theta}{r} + \frac{R\Theta^{\prime\prime}}{r^2} & = 0\\
    R^{\prime\prime} \Theta + \frac{R^\prime \Theta}{r} & = - \frac{R \Theta^{\prime\prime}}{r^2}\\
    \frac{r^2}{R^{\prime\prime}}{R} + \frac{r R^\prime}{R} & = - \frac{\Theta^{\prime\prime}}{\Theta} = \lambda
  \end{align}

  \item Solve for $\Theta$ : $-\frac{\Theta^{\prime\prime}}{\Theta} = \lambda \Rightarrow \Theta^{\prime\prime} = -\lambda \Theta$
  %
  If $\lambda > 0$, then we have:
  %
  \begin{align}
    \Theta(\theta) & = A \sin(\sqrt \lambda \theta) + B \cos(\sqrt \lambda \theta)
  \end{align}

  Since the solution is periodic in terms of $\theta$ :
  $\displaystyle
  \begin{cases}
    \Theta(0) & = \Theta(2 \pi)\\
    \Theta^\prime(0) & = \Theta^\prime(2 \pi)
  \end{cases}$
  %
  \begin{align}
    \Theta(0) = \Theta(2 \pi) \Rightarrow B & = A \sin(\sqrt \lambda 2 \pi) + B \cos ( \sqrt \lambda 2 \pi)\\
    \Theta^\prime(\theta) & = A \sqrt \lambda \cos(\sqrt \lambda \theta) - B \sqrt \lambda \sin( \sqrt \lambda \theta)\\
    \Theta^\prime(0) = \Theta^\prime(2 \pi) \Rightarrow A \sqrt \lambda & = A \sqrt \lambda \cos(\sqrt \lambda 2 \pi) - B \sqrt \lambda \sin(\sqrt \lambda 2 \pi)
  \end{align}

  These equations are satisfied when $\sqrt \lambda 2 \pi = 2 \pi n$, $n \in \Z^+$. Then, $\lambda_n = n^2$. If we consider $n$, then we can also write $\Theta_n(\theta) = A_n \sin(n \theta) + B_n \cos(n \theta)$.

  If $\lambda = 0 : \Theta^{\prime\prime}_0 = A \theta + B$.
  %
  \begin{align}
    \Theta_0(0) = \Theta_0(2 \pi) & \Rightarrow B = A 2\pi + \theta\\
    & \Rightarrow A = 0\\
    \Theta^\prime_0 (\theta) = 0 & \Rightarrow \Theta^\prime(0) = \Theta^\prime(2 \pi)\\
    & \Rightarrow \Theta_0(\theta) = B_0
  \end{align}
  Note: $\lambda < 0$ yields the trivial solution.

  \item Solve for $R$
  %
  \begin{align}
    r^2 \frac{R^{\prime\prime}_n}{R_n} + r \frac{R^\prime_n}{R_n} = \lambda_n & \Rightarrow r^2 R + rR^\prime_n = n^2 R_n\\
    & \Rightarrow r^2 R^{\prime\prime}_n + rR^\prime_n - n^2 R_n = 0
  \end{align}

  Here, let our guess be $R_n(r) = r^m$. Let us plug our guess in:
  %
  \begin{align}
    r^2 m(m - 1)r^{m - 2} + r mr^{m - 1} - n^2 r^m & = 0\\
    r^2 \left( m(m - 1) + m - n^2 \right) & = 0\\
    r^2 \left( m^2 - n^2 \right) & = 0 \Rightarrow m = \pm n
  \end{align}

  If $n > 0$,
  %
  \begin{align}
    R_n(r) & = C_nr^n + D_nr^{-n}
  \end{align}

  If $n = 0$,
  %
  \begin{align}
    r^2R^{\prime\prime}_0 + rR^\prime_0 & = 0\\
    R_0(r) & = C_0 r^0 + D_0 \ln r\\
    & = C_0 + D_0 \ln r
  \end{align}
  %
  \item Combine to find $u_n$ and $u$:
  %
  \begin{align}
    u_n(r, \theta) & =
    \begin{cases}
      B_0( C_0 + D_0 \ln r) & n = 0\\
      C_nr^n + D_n r^{-n}\left(A_n \cos(n \theta) + B_n \cos(n \theta) \right) & n \in Z^+
    \end{cases}
  \end{align}

  By linearity,
  %
  \begin{align}
    u(n \theta) & = B_0(C_0 + D_0 \ln r) + \sum^\infty_{n = 1} \left(C_n r^n + D_n r^{-n}\right) \left( A_n \sin(n \theta) + B_n \cos(n \theta) \right)\\
    & = c_0 + d_0 \ln r + \sum^\infty_{n = 1} (a_nr^n + b_nr^{-n}) \sin(n \theta) + (c_nr^n + d_nr^{-n}) \cos(n \theta)
  \end{align}
\end{enumerate}

%______________________________________________________________________________%

\topic{February 16, 2022}
\begin{enumerate}
  \setcounter{enumi}{4}
  \item Let us find the oefficients using the boundary conditions.
  \begin{align}
    u(R_1, \theta) & = g_1(\theta)\\
    %%%%
    \Rightarrow g_1(\theta) & = \underline{C_0 + d_0 \ln R_1} +
    \sum^\infty_{n = 1}
    \left[
    (\underline{a_nR^n_1 + b_nR^{-n}_1}) \sin(n \theta) +
    (\underline{C_n R^n_1 + d_n R^{-n}_1}) \cos(n \theta)
    \right]\\
    %%%%
    u(R_2, \theta) & = G_2(\theta)\\
    \Rightarrow g_2(\theta) & = \underline{C_0 + d_0 \ln R_2} +
    \sum^\infty_{n = 1}
    \left[
    (\underline{a_n R^n_2 + b_n R^{-n}_2}) \sin(n \theta) +
    (\underline{C_n R^n_2 + d_n R^{-n}_2}) \cos(n \theta)
    \right]
  \end{align}
  Underlines book-scan: $B_0, A_n, B_n, \twiddle B_0, \twiddle A_n, \twiddle B_n$:
  %
  \begin{align}
    &
    \begin{cases}
      B_0 & = c_0 + d_0 \ln R_1\\
      \twiddle B_0 & = c_0 + d_0 \ln R_12
    \end{cases}\\
    %
    &
    \begin{cases}
                A_n & = a_n R^n_1 + b_nR^{-n}_1\\
      \twiddle  A_n & = a_n R^n_2 + b_nR^{-n}_2
    \end{cases}\\
    %
    &
    \begin{cases}
               B_n & = c_n R^n_1 + d_n R^{-n}_1\\
      \twiddle B_n & = c_n R^n_2 + d_n R^{-n}_2
    \end{cases}
  \end{align}
\end{enumerate}
\ex Solve $\Delta u = 0$, where
\begin{itemize}
  \item $u(1, \theta) = 3 \sin( 2 \theta )$
  \item $u(2, \theta) = 7 \cos( 5 \theta )$
\end{itemize}
%
\begin{enumerate}
  \item Assume $u(r, \theta) = R(r) \Theta(\theta)$
  %
  \begin{align}
    \Delta u = u_{rr} + \frac{u_r}{r} + \frac{u_{\theta\theta}}{r^2} & = R^{\prime\prime} \Theta + \frac{R^\prime \Theta}{r} + \frac{R \Theta^{\prime\prime}}{r^2} = 0\\
    %%%%
    & \Rightarrow R^{\prime\prime} \Theta + \frac{R^\prime \Theta}{r} = - \frac{R\Theta^{\prime\prime}}{r^2}\\
    & \Rightarrow r^2 \frac{R^{\prime\prime}}{R} + r \frac{R^\prime}{R} = - \frac{\Theta^{\prime\prime}}{\Theta} = \lambda
  \end{align}
  %
  \item Solve for $\Theta$ : $- \frac{\Theta^{\prime\prime}}{\Theta} = \lambda \Rightarrow \Theta^{\prime\prime} = -\lambda \Theta$.

  If $\lambda > 0$, then
  %
  \begin{align}
    \Theta(\theta) & =
    A \sin(\sqrt \lambda \theta) + B \cos(\sqrt \lambda \theta)\\
    %%
    \Theta^\prime(\theta) & =
    A \sqrt \lambda \cos ( \sqrt \lambda \theta) - B \sqrt \lambda \sin (\sqrt \lambda \theta)\\
    %%
    \sqrt \lambda 2 \pi = 2 n \pi \Rightarrow \lambda_n = n^2, n \in \Z^+ &
    \begin{cases}
      \Theta(0) = \Theta(2\pi) & \Rightarrow
      B = A \sin(\sqrt \lambda 2 \pi) + B \cos(\sqrt \lambda 2 \pi)\\
      %%
      \Theta^\prime = \Theta^\prime(2 \pi) & \Rightarrow
      A \sqrt \lambda = A \sqrt \lambda \cos (\sqrt \lambda 2 \pi) - B \sqrt \lambda \sin (\sqrt \lambda 2 \pi)
    \end{cases}\\
    %%
    = n^2 & \Rightarrow
    \Theta(n)(\theta) = A_n \sin(n \theta) + B_n \cos(n \theta)
  \end{align}
  If $\lambda = 0$, then the second derivative is $0$.
  %
  \begin{align}
    \Theta^{\prime\prime}_0 & \Rightarrow
    \Theta_0(\theta) = A_0\Theta + B_0\\
    & \Rightarrow \Theta^\prime_0 (\theta) = A_0\\
    & \Rightarrow \Theta_0(0) = \Theta_0(2 \pi) \Rightarrow B_0 = 2 \pi A_0 + B_0 \Rightarrow A_0 = 0\\
    %%
    & \Rightarrow \Theta^\prime_0(0) = \Theta^\prime_0 (2 \pi) = 0
  \end{align}
  \item Solve for $R$ : $r^2 \frac{R^{\prime\prime}_n}{R_n} + \frac{rR^\prime_n}{R_n} = \lambda_n$
  %
  \begin{align}
    r^2 R^{\prime\prime}_n + rR^\prime_n - n^2 R_n = 0\\
  \end{align}

  Try $R_n(r) = R^m$, then
  %
  \begin{align}
    r^2 m( m - 1) r^{m - 2} + r mr^{m - 1} - n^2 r^m & = 0\\
    r^m \left[ m( m - 1) + m - n^2 \right] & = 0\\
    m^n - n^2 & = 0\\
    m & = \pm n
  \end{align}
  Next, let us write:
  %
  \begin{align}
    & \Rightarrow
    \begin{cases}
      R_n(r) & = C_n r^n + D_n r^{-n}, n \in \Z^+\\
      R_n(r) & = C_0 + D_0 \ln r
    \end{cases}
  \end{align}
  \item Combine to obtain $u_n$ and $u$,
  %
  \begin{align}
    u_n(r, \theta) & =
    \begin{cases}
      B_0(C_0 + D_0 \ln r) & n = 0\\
      (C_n r^n + D_n r^{-n})(A_n \sin(n \theta) + B_n \cos(n \theta)) & n \in Z^+
    \end{cases}
  \end{align}
  By linearity,
  %
  \begin{align}
    u(r, \theta) & =
    c_0 + d_0 \ln r + \sum^\infty_{n = 1}
    \left(
    (C_n r^n + D_n r^{-n})(A_n \sin(n \theta) + B_n \cos(n \theta))
    \right)\\
    %%
    & =
    c_0 + d_0 \ln r + \sum^\infty_{n = 1}
    \left(
    (a_n r^n + b_n r^{-n})\sin(n \theta) + (c_n r^n + d_n r^{-n}) \cos(n \theta)
    \right)
  \end{align}
  %
  \item Find coefficients using $BCs$ :
  %
  \begin{align}
    u(1, \theta) & = 3 \sin(2 \theta)\\
    u(2, \theta) & = 7 \cos(5 \theta)\\
    u(1, \theta) & = c_0 + d_0 \ln(1) + \sum^\infty_{n = 1}
    \left[
    (a_n + b_n) \sin(n \theta) +
    (c_n + d_n) \cos(n \theta)
    \right]\\
    &
    \begin{cases}
      c_0       & = 0\ \           \\
      c_n + d_n & = 0\ \ \forall n \\
      a_2 + b_2 & = 3\ \           \\
      a_n + b_n & = 0\ \ \forall n, n \neq 2
    \end{cases}
  \end{align}
  Now, let us write:
  %
  \begin{align}
    u(2, \theta) & = 7 \cos(5 \theta)\\
    u(2, \theta) & = c_0 + d_0 \ln 2 + \sum^\infty_{n = 1}
    \left[
    (a_n 2^n + b_n 2^{-n}) \sin( n \theta ) +
    (c_n 2^n + d_n 2^{-n}) \cos( n \theta )
    \right] = 7 \cos(5 \theta)\\
    %
    &
    \begin{cases}
      c_0 + d_0 \ln 2       & = 0 \Rightarrow d_0 = 0\\
      a_n 2^n + b_n 2^{-n}  & = 0\ \forall n\\
      c_5 2^5 + d_5 2^{-5}  & = 7\\
      c_n 2^n + d_n 2^{-n}  & = 0\ \forall n, n \neq 5
    \end{cases}
  \end{align}
  If $n \neq 5$:
  %
  \begin{align}
    \begin{cases}
      c_n + d_n & = 0\\
      c_n 2^n + d_n 2^{-n} & = 0
    \end{cases}
    \Rightarrow c_n = d_n = 0
  \end{align}
  %
  If $n \neq 2$:
  %
  \begin{align}
    \begin{cases}
      a_n     + b_n & = 0\\
      a_n 2^n + b_n 2^{-n} & = 0
    \end{cases}
    \Rightarrow a_n = b_n = 0
  \end{align}
  %
  If $n = 5$,
  %
  \begin{align}
    &
    \begin{cases}
      c_5 2^5 + d_5 2^{-5} & = 7\\
      c_5     + d_5        & = 0 \Rightarrow d_5 = -c_5
    \end{cases}\\
    c_5 2^5 - c_5 2^{-5} = &\ 7\\
    c_5(32 - \frac{1}{32}) = &\ 7\\
    c_5 = &\ \frac{7}{32 - \frac{1}{32}}
  \end{align}
  %
  If $n = 2$,
  %
  \begin{align}
    4(a_2 + b_2 & = 3) \\
    - a_2 2^2 + b_2 2^{-3} & = 0\\
    \hline
    \frac{15}{4} b_2 = 12 & \Rightarrow b_2 = \frac{48}{15} = \frac{16}{5}\\
    & \Rightarrow a_2 = 3 - b_2 = 3 - \frac{16}{5} = -\frac{1}{5}
  \end{align}
  %
  \item $u_y(x, 0) = 0 \Rightarrow X(x)Y^\prime(0) = 0 \Rightarrow Y^\prime(0) = 0$
\end{enumerate}

\subsection{Heat and Wave Equations in Polar Coordinates}

\topic{February 23, 2022}

Heat Equation:
%
\begin{align}
  u_t & = \alpha^2 \Delta u
\end{align}

Let $\alpha = 1$, then let us write:
%
\begin{align}
  u_t & = u_{rr} + \frac{u_r}{r} + \frac{u_{\theta\theta}}{r^2}
\end{align}

Wave Equation:
%
\begin{align}
  u_{tt} & = c^2 \Delta u
\end{align}

Here, let $c = 1$:
%
\begin{align}
  u_{tt} & = u_rr + \frac{u_r}{r} + \frac{u_{\theta\theta}}{r^2}
\end{align}

In the last two equations, we worked with three variables: $t, r, \theta$.
%
\begin{align}
  u_t & = u_{rr} + \frac{u_r}{r} + \frac{u_{\theta\theta}}{r^2}
\end{align}

Here, assume $u(r, \theta, t) = R(r)\Theta(\theta)T(t)$
%
\begin{align}
  R \Theta T^\prime & = R^{\prime\prime} \Theta T + \frac{R^\prime \Theta T}{r} + \frac{R \Theta^{\prime\prime}T}{r^2}\\
  %
  \frac{T^\prime}{T} & = \frac{R^{\prime\prime}}{R} + \frac{R^\prime}{rR} + \frac{\Theta^{\prime\prime}}{r^2 \Theta} = -\lambda\\
  %
  \frac{T^\prime}{T} & = -\lambda
\end{align}

Here, let us put a pin on $T$ and solve for the second part of the equation:
%
\begin{align}
  \frac{R^{\prime\prime}}{R} + \frac{R^\prime}{rR} + \frac{\Theta^{\prime\prime}}{r^2\Theta} & = - \lambda\\
  %
  \frac{r^2R^{\prime\prime}}{R} + \frac{rR^\prime}{R} + \frac{\Theta^{\prime\prime}}{\Theta} & = - \lambda r^2\\
  %
  \frac{r^2 R^{\prime\prime}}{R} +
  \frac{r   R^\prime}{R} +
  \lambda r^2 & = -\frac{\Theta^{\prime\prime}}{\Theta} = \mu
\end{align}

We now have separate ODEs for each of the functions $T, R, \Theta$. The solution for $\Theta$ looks like Laplace in polar. Recall, we set $\lambda$ as $n^2$:
%
\begin{align}
  \frac{r^2 R^{\prime\prime}}{R} + \frac{r R^\prime}{R} + \lambda r^2 & n^2\\
  %
  r^2 R^{\prime\prime} + rR^\prime + (\lambda r^2 - n^2)R & = 0\\
  %
  R^{\prime\prime} + \frac{R^\prime}{r}
  + \left(\lambda - \frac{n^2}{r^2}\right)R & = 0
\end{align}

Here, this is Bessel's Equation.

We use the power series to solve Bessel's Equation and obtain:
%
\begin{align}
  R_n(r) & = \sum^\infty_{i = 1} \frac{(-\lambda)^i r^{2i + n}}{2^{n + 2}(i + n)! i!}
\end{align}

If $\lambda = 1$, we get the Bessel function:
%
\begin{align}
  J_n(x) & = \sum^\infty_{i = 1} \frac{(-1)^i x^{2i + n}}{2^{n + 2}(i + n)! i!}
\end{align}

\subsection{Laplace in Spherical Coordinates}
%
\begin{align}
  \Delta u & = u_{xx} + u_{yy} + u_{zz}
\end{align}

Using the chain rule, we obtain:
%
\begin{align}
  \Delta u & =
  u_{\rho \rho} +
  \frac{2}{\rho}u_\rho +
  \frac{1}{p^2} u_{\phi\phi} +
  \frac{\cot \theta}{\rho^2} u_\phi +
  \frac{1}{\rho^2 \sin^2 \phi} u_{\theta\theta} = 0
\end{align}

Here, assume $u(\rho, \theta, \Phi) = P(\rho)\Theta(\theta)\Phi(\phi)$:

Equation for $\Theta$
%
\begin{align}
  (1 - x^2) \Theta^{\prime\prime}(x) -
  2 \Theta^\prime(x) +
  n(n + 1) \Theta(x)
  & = 0
\end{align}

Where $x = \cos(\theta)$. This equation is Legendre's Equation.

The solutions are Legendre Polynomials $P_n(x)$:
%
\begin{center}
  \begin{tabular}{c|c}
    $n$ & $P_n(x)$\\
    \hline
    $0$ & $1$\\
    $1$ & $x$\\
    $2$ & $\frac{3x^2 - 1}{2}$\\
    $3$ & $\frac{5x^3 - 3x}{2}$\\
    $4$ & $\frac{35 x^4 - 30x^2 + 3}{8}$
  \end{tabular}
\end{center}

Legendre Polynomials ? make an orthogonal set on $[-1, 1]$
%
\begin{align}
  \int^1_{-1} P_m(x) P_n(x) \text{ dx} = 0, m \neq n
\end{align}

Laplace's Equation: $\Delta u = 0$, $u \in \Omega \subset \R$, $x \in \twiddle{\Omega} \subseteq \R^n$

\dfn A function that satisfies Laplace's Equation is called a harmonic function.

\dfn Let the ball of radius $r$ centered on point $x_0$ be:
%
\begin{align}
  B_r(x_0) & = \{x : ||x - x_0||_2 \leq r \}
\end{align}

Here, let us revisit what the norm notation indicates:
%
\begin{align}
  x & =
  \begin{bmatrix}
    x_1\\
    x_2\\
    \ldots\\
    x_n
  \end{bmatrix}\\
  %
  ||x||_{\ell_2} & = \sqrt[2]{x^2_1 + x^2_2 + \ldots + x^2_n} \text{ Standard/Euclidean Norm}\\
  ||x||_{\ell_1} & = |x_1| + |x_2| + \ldots + |x_n| \text{ Take $|x_n|$ for odd powers}\\
  ||x||_{\ell_4} & = \sqrt[4]{x^4_1 + x^4_2 + \ldots + x^4_n}\\
  ||x||_{\ell_\infty} & = \max\{ |x_1|, |x_2|, \ldots, |x_n| \}
\end{align}

Here, let us write:
%
\begin{align}
  ||f||_{L_1} & = \int_\Omega |f(x)| \dx\\
  %
  ||f||_{L_2} & = \sqrt{\int_\Omega f(x)^2 \dx}\\
  %
  ||f||_{L_n} & = \sqrt[n]{\int_\Omega f(x)^n \dx}\\
  %
  ||f||_{L_\infty} & = \text{essup } |f(x)|
\end{align}
Essential Supremum, $x \in \Omega$.

$\ell_2$ and $L_2$ are the only two that correspond to a Hilbert space.

\thm If $u$ is harmonic and $B_r(x_0) \subset \Omega$, then then average value of $u$ in the ball equals $u(x_0)$.
%
\begin{align}
  u(x_0) & = \frac
  {
    \int_{B_r(x_0)} u(x) \dx
  }
  {
    \int_{B_r(x_0)} 1 \text{ dr}
  }
\end{align}

\newpage

\topic{February 25, 2022}

\note These theorems are t? for any ball of any radians in dimensions $n$.
The theorems don't care about the shape of $\Omega$ or Boundary Conditions.

\ex $n = 1$, $\Delta u = u_{xx} = 0$

Here, the function is a linear function
%
\begin{align}
  u & = Ax + B
\end{align}

When $n = 1$, we are working with an interval $[x_0 - r, x_0 + r]$.

Theorem 1:
%
\begin{align}
  u(x_0) & = \frac
  {
    \int^{x_0 + r}_{x_0 - r} u(x) \text{ dx}
  }
  {
    \int^{x_0 + r}_{x_0 - r} 1 \text{ dx}
  }\\
  & = \frac
  {
    \int^{x_0 + r}_{x_0 - r} u(x) \text{ dx}
  }
  {
    2r
  }
\end{align}

Here, we have the definition for average.

Theorem:
%
\begin{align}
  u(x_0) & =
  \frac
  {
    u(x_0 - r) + u(x_0 + r)
  }
  {
    2
  }
\end{align}
