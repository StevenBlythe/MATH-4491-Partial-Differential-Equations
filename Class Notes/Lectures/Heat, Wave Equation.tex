\newpage
\section{Heat, Wave Equation}

\subsection{1-D Heat Equation}

Assume cross sections are uniform
Imagine a cross section:
%
\begin{center}
  O o==========o L
\end{center}

Assume cross sections are uniform and the lateral sides are well insulated $\Rightarrow$ heat only flows in the x-direction.

We need the following:
%
\begin{itemize}
  \item $u(x, t)$ : Temperature of rod at position $x$ and time $t$.
  \item $u(x, 0)$ : Initial temperature
  \item $u(0, t)$ and $u(L, t)$ : Boundary Conditions
\end{itemize}

\dfn
%
\begin{itemize}
  \item $g(x, t)$ : heat flux (energy / area time)
  \item $Q(x, t)$ : heat energy density (energy / volume)
  \item $A$ : Cross sectional area
  \item $C_P$ : Heat capacity or specific heat
  \item $\rho$ : Density
  \item $K$ : Thermal conductivity
\end{itemize}

We want to find an equation for the temperature evolution. We will use conservation of energy : Look at a little $\Delta x$ section of the rod starting at $x_0$.
%
\begin{center}
  $\Delta x$\\
  o=====$|$o$|$=====o\\
  $x_0\  x_0\Delta x$
\end{center}

Conservation of energy : heat in - heat out = heat accumulated

Heat in $ = 'qA\Delta t' = A \int_{t_0}^{t_0 + \Delta t} q(x_0, t) \text{ dt}$

Heat out $ = A\int_{t_0}^{t_0 + \Delta t} q(x_0 + \Delta x, t) \text{ dt}$

Heat Accumulated $ = QA\Delta x |_{t_0 + \Delta t} - QA\Delta x|_{t_0}$
%
\begin{align}
  & = A\int^{x_0 + \Delta x}_{x_0} Q(x, t_0 + \Delta t) \text{ dx} - A\int^{x_0 + \Delta x}_{x_0} Q(x, t_0) \text{ dx}
\end{align}

\bigbreak

\subsection{Heat Equation, Conservation of Energy}

\topic{January 24, 2022}

Heat in - heat out = heat accumulated
%
\begin{align}
  A \int^{t_0 \rightarrow \Delta t}_{t_0} g(x_0, t) \dt -
  A \int^{t_0 \rightarrow \Delta t}_{t_0} q(x_0 + \Delta x, t) \dt =
  A \int^{t_0 \rightarrow \Delta t}_{t_0} Q(x, t_0 + \Delta t) \dx -
  A \int^{t_0 \rightarrow \Delta t}_{t_0} Q(x, t_0) \dx
\end{align}

Let us simplify and divide by $A$. Then, let us combine the integrals:
%
\begin{align}
  \int^{t_0 \rightarrow \Delta t}_{t_0} [q(x_0, t) - q(x_0 + \Delta x, t)] \dt =
  \int^{t_0 \rightarrow \Delta t}_{t_0} [Q(x, t_0 + \Delta t) - Q(x, t_0)] \dx
\end{align}

Divide by $\Delta x \Delta t$ and take limit as $\Delta x, \Delta t \rightarrow 0$
%
\begin{align}
  \lim_{\Delta t, \Delta x \rightarrow 0} \frac{1}{\Delta x \Delta t}
  \int^{t_0 \rightarrow \Delta t}_{t_0} [q(x_0, t) - q(x_0 + \Delta x, t)] \dt & =
  \lim_{\Delta t, \Delta x \rightarrow 0} \frac{1}{\Delta x \Delta t}
  \int^{t_0 \rightarrow \Delta t}_{t_0} [Q(x, t_0 + \Delta t) - Q(x, t_0)] \dx\\
  \lim_{\Delta t} \frac{1}{\Delta t} \int^{t_0 \rightarrow \Delta t}_{t_0} [\lim_{\Delta x \rightarrow 0} \frac{q(x_0, t) - q(x_0 + \Delta x, t)}{\Delta x}] \dt & =
  \lim_{\Delta x \rightarrow 0} \frac{1}{\Delta x}
  \int^{t_0 \rightarrow \Delta t}_{t_0} \lim_{\Delta t \rightarrow 0} \frac
  {Q(x, t_0 + \Delta t) - Q(x, t_0)}{\Delta t} \dx
\end{align}

On the left side, we see the order is a bit difference. We want the delta to come first, such as in the difference quotient. The eft is now $-q_x(x_0, t)$ and the right is $Q_t(x, t_0)$.
%
\begin{align}
  \lim_{\Delta t \rightarrow} \frac{1}{\Delta t} \int^{t_0 + \Delta t}_{t_0} - q_x(x_0 t) \dt & =
  \lim_{\Delta x \rightarrow 0} \frac{1}{\Delta x} \int^{x_0 + \Delta x}_{x_0} Q_t(x, t_0) \dx\\
  \lim_{\Delta t \rightarrow 0} -q_x(x_0, t_0 + \Delta t) & = \lim_{\Delta x \rightarrow 0} Q_t(x_0 + \Delta x, t_0)
\end{align}

At step 28, we used the fundamental theorem of calculus and derived both sides.
%
\begin{align}
  -q_x(x_0, t_0) & = Q_t(x_0, t_0)
\end{align}

Since $x_0$ and $t_0$ are arbitrary, $-q_x(x, t) = Q_t(x, t)$\\
$q$ and $Q$ are related to $u$:
%
\begin{align}
  Q = \rho c_p u \qquad & \qquad q = -Ku_x\\
  -q_x = Q_t & \Rightarrow Ku_{xx} = \rho c_p u_t\\
  & \Rightarrow u_t = \frac{k}{\rho c_p} u_{xx}\\
  & \Rightarrow u_t = \alpha^2 u_{xx}\\
  \alpha & = \sqrt{\frac{K}{\rho c_p}}
\end{align}

$\alpha$ is thermal diffusivity

$u_t = \alpha^2 u_{xx} \leftarrow$ 1-D heat equation (diffusivity equation)

We have a steady-state:
$(t \rightarrow \infty)$, $u_t = 0 \Rightarrow u_{xx} = 0 \Rightarrow$ straight line

1-D: $-q_x = Q_t \Rightarrow -\grad \cdot \vec q = Q_t, \qquad \vec q$ is a vector.
%
\begin{align}
  q = -K \grad u & \Rightarrow - \grad \cdot (-K \grad u) = \rho c_p u_t\\
  & \Rightarrow K \Delta u = \rho c_p u_t\\
  & \Rightarrow u_t = \alpha^2 \Delta u
\end{align}

What about a steady-state? $u_t = 0$
%
\begin{align}
  \Delta u & = 0
\end{align}

Here, we have Laplace's equation.

\note It is not dependent on time.

\subsection{The Wave Equation}

$u(x, t)$ is the height of the rope.
We use Newton's $2^{nd}$ law on small segments of rope.
%
\begin{itemize}
  \item $\rho = $ density of rope.
  \item $\text{dm} = \rho \dx$
\end{itemize}
% Image here
% March 30, 2022 comment -- No idea what that image should be
\begin{align}
  F & = ma\\
  T \sin(\theta(x + \Delta x)) - T \sin(\theta(x)) & = \int^{x + \Delta x}_x u_{tt} \text{ dm}\\
  T[ \sin(\theta (x + \Delta x)) - \sin(\theta (x))] & = \rho \int^{x + \Delta x}_{x} u_{tt} \dx
\end{align}

Let us assume $\theta$ is small, $\sin \theta \approx \tan \theta$
%
\begin{align}
  T[\tan(\theta(x + \Delta x)) - \tan(\theta(x))] & = \rho \int^{x + \Delta x}_x u_{tt} \dx
\end{align}

Also, $\tan(\theta(x)) = u_x(x, t)$.
%
\begin{align}
  T[u_x(x + \Delta x, t) - u_x(x, t)] & =
  \rho \int^{x + \Delta x}_x u_{tt} \dx
\end{align}

Now, let us divide both sides by
$\Delta x$ and take the limit as
$\Delta x \rightarrow 0$
%
\begin{align}
  \lim_{\Delta x \rightarrow 0}
  T\Big[\frac{u_x(x + \Delta x, t) - u_x(x, t)}{\Delta x}\Big]
  & =
  \rho \lim_{\Delta x \rightarrow 0}
  \frac{\int^{x + \Delta x}_x u_{tt} \dx}{\Delta x}
\end{align}

On the left side, we have $u_xx$ and the right side we have
$u_{tt}(x + \Delta x, t)$.
%
\begin{align}
  Tu_{xx}(x, t) & = \rho u_{tt}(x, t)\\
  u_{tt} = \frac{T}{\rho} u_{xx} & = c^2 u_{xx}\\
  c & = \sqrt{\frac{T}{\rho}} = \text{ wave speed }
\end{align}

On the left, we have the $1-D$ wave equation which is used for light, sound, rope, etc.

In 2-D, it corresponds to a vibrating membrane (drum)
%
\begin{align}
  u_{tt} & = c^2\Delta u
\end{align}

\underline{Remark}:
%
\begin{align}
  u_t = u_{xx} \quad & \text{Heat Equation}\\
  u_{xx} + u_{yy} = 0 \quad & \text{ Laplace Equation}\\
  u_{tt} = u_{xx} \quad & \text{ wave}
\end{align}

Here, we can replace:

$u_t$ with $t$\\
$u_x$ with $x$\\
$u_{xx}$ with $x^2$
%
\begin{enumerate}
  \item $t = x^2$ parabola
  \item $x^2 + y^2 = 0$ ellipse
  \item $t^2 = x^2$ hyperbolas
\end{enumerate}

So, the equations behave like the following:
%
\begin{enumerate}
  \item The Heat Equation is called parabolic
  \item The Laplace Equation is called elliptic
  \item The Wave Equation is called hyperbolic
\end{enumerate}
