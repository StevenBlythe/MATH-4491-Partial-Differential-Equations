\newpage
\section{Complex Analysis, Fourier Transform}

\subsection{Gauss' MVT in Complex Analysis}

If $f(z)$ is analytic, then
%
\begin{align}
  f(z_0) & =
  \frac
  {
    \oint_c f(z) \text{ dz}
  }
  {
    2 \pi r
  }
\end{align}

Fourier Series are for $f(x)$ where $x$ is defined over a finite interval.
Fourier Transforms are for $f(x)$ where $x$ is defined on $(-\infty, \infty)$.

There is a different form of the Fourier Series.

Here, let us consider Euler's Formula
%
\begin{align}
  e^{ix} & = \cos x + i \sin x\\
  e^{ix} & = \cos x - i \sin x
\end{align}

When we combine both formulas, we get
%
\begin{align}
  \begin{cases}
    \cos x& = \frac{e^{ix} + e^{-ix}}{2}\\
    \sin x & = \frac{e^{ix} - e^{-ix}}{2i} = \frac{\sin ix}{2i}
  \end{cases}
\end{align}

Now, let us rewrite Fourier Series:
%
\begin{align}
  f(X) & = \sum^\infty_{n = 0} A_n \sin\left(\frac{n \pi x}{L}\right) + B_n \cos\left(\frac{n \pi x}{L}\right)
\end{align}

Here, let us replace our Fourier Series with terms we found,
%
\begin{align}
  f(x) & =
  \sum^\infty_{n = 0}
  A_n \frac
  {
    e^{i \frac{n \pi x}{L}} - e^{-i \frac{n \pi x}{L}}
  }
  {
    2 i
  }
  + B_n \frac
  {
    e^{i\frac{n \pi x}{L}} + e^{-i \frac{n \pi x}{L}}
  }
  {
    2
  }\\
  & =
  \sum^\infty_{n = 0}
  \left[
    \frac{A_n}{2i} + \frac{B_n}{2}
  \right]
  e^{\frac{i n \pi x}{L}}
  +
  \left[
    - \frac{A_n}{2i} + \frac{B_n}{2}
  \right]
  e^{- \frac{i n \pi x}{L}}\\
  & = \sum^\infty_{n = -\infty} \alpha_n e^{\frac{i n \pi x}{L}}
\end{align}

Here, we found an alternative Fourier Series where
%
\begin{align}
  \alpha_n & = \frac{A_n}{2i} + \frac{B_n}{2}  n = 0, 1, 2, \ldots\\
  \alpha_n & = -\frac{A_n}{2i} + \frac{B_n}{2} n = 0, -1, -2, \ldots
\end{align}

In the alternative Fourier Series, there is a basis function aside $\alpha_n$

The basis functions are almost orthogonal
%
\begin{align}
  \int^L_{-L}
  e^{\frac{i m \pi x}{L}} e^{\frac{i n \pi x}{L}} \text{ dx} & =
  \int^L_{-L}
  e^{\frac{i \pi x (m + n)}{L}} \text{ dx}\\
  & = \frac{L}{i \pi(m + n)} e^{\frac{i \pi x(m + n)}{L}} \Big|^L_{-L}\\
  & = \frac{L}{i \pi(n + m)} \left[ e^{i \pi(m + n)} - e^{-i \pi(m + n)} \right]\\
  & = \frac{L}{i \pi(m + n)} 2 i \sin(\pi(m + n)) = 0
\end{align}

If $m = -n$, then we get:
%
\begin{align}
  \int^L_{-L} e^{\frac{i m \pi x}{L}} e^{\frac{-i m \pi x}{L}} \text{ dx}
  & = \int^L_{-L} 1 \text{ dx} = 2L
\end{align}

To find $\alpha_n$:
%
\begin{align}
  f(x) & = \sum^\infty_{n = - \infty} \alpha_n e^{\frac{i n \pi x}{L}}
\end{align}

Here, let us multiply by $e^{-\frac{i n \pi x}{L}}$ and integrate.
%
\begin{align}
  \int^L_{-L} f(x)e^{-\frac{i k \pi x}{L}} \text{ dx} & =
  %
  \sum^\infty_{n = -\infty} \alpha_n \int^L_{-L} e^{\frac{i n \pi x}{L}} e^{\frac{-i k \pi x}{L}} \text{ dx}
\end{align}

Here, the integral is $0$ except when $k = n$.
%
\begin{align}
  \int^L_{-L} f(x) e^{- \frac{n \pi x}{L}} \text{ dx} & = \alpha_n 2L\\
  \alpha_n & = \frac{1}{2L} \int^L_{-L} f(x)e^{-\frac{i n \pi x}{L}} \text{ dx}\\
  f(x) & = \sum^\infty_{n = -\infty} \alpha_n e^{\frac{i n \pi x}{L}}\\
  & = \sum^\infty_{n = -\infty} \frac{1}{2L} \int^L_{-L} f(x) e^{- \frac{i n \pi x}{L}} \text{ dx} e^{\frac{i n \pi x}{L}}
\end{align}

\subsection{Fourier Transform}

Define $\xi_n = \frac{n \pi}{L}$, $\Delta \xi = \frac{\pi}{L}$.
%
\begin{align}
  f(x) & = \sum^\infty_{n = -\infty} \frac{\Delta \xi}{2 \pi} \int^L_{-L} f(x) \text{ dx} e^{i \xi n x}
\end{align}

This is a Riemann Sum.

Now let $L \to \infty, \Delta \xi \to d \xi$, replace $\xi_n$ with $\xi$ and $\sum \rightarrow \int$.
%
\begin{align}
  f(x) & = \int^\infty_{-\infty} \frac{1}{2 \pi} \int^\infty_{-\infty} f(x) e^{-i \xi x} \text{ dx } e^{i \xi x} d\xi
\end{align}

Define the Fourier Transform
%
\begin{align}
  F[f] & = \frac{1}{\sqrt {2 \pi}} \int^\infty_{- \infty} f(x) e^{-i \xi x} \text{ dx} = \hat f(\xi)\\
  f^{-1}[\hat f] & = \frac{1}{\sqrt{2 \pi}} \int^\infty_{- \infty} \hat f(\xi) e^{i \xi x} \text{ d}\xi = f(x)
\end{align}
The first line is the Fourier Transform, the second line is the Inverse Fourier Transform.

\note Laplace Transform
%
\begin{align}
  F(s) & = \frac{1}{\sqrt{2 \pi}} \int^\infty_0 f(t) e^{-st} \text{ dt}\\
  f(t) & = \frac{1}{\sqrt{2 \pi} i} \int^{c + i \infty}_{c - i\infty} F(s)e^{st} \text{ ds}
\end{align}

The first line is the Laplace Transform, whereas the second line is the Inverse Laplace Transform.

We use Laplace Transforms on $[0, \infty)$.

\note We use Fourier Transforms for functions $f(x)$ where
%
\begin{align}
  \int^\infty_{-\infty} |f(x)| \text{ dx } < \infty
\end{align}

\note $c < \infty$ indicates finite.

\newpage

\topic{February 28, 2022}

\underline{Fourier Transform}
%
\begin{align}
  F[f(x)] & = \hat f(\xi)
\end{align}

Here, $F[f]$ represents the frequencies in $f$.

\bigbreak

\subsubsection{Panseval's Equality}
\begin{enumerate}
  \item if $x \in [-L, L]$
  %
  \begin{align}
    \frac{1}{2L} \int^L_{-L} [f(x)]^2 \text{ dx} & = \sum^\infty_{n = -\infty} |\alpha_n|^2
  \end{align}

  On the left integral, we have the inner product of $f$ with itself.
  On the right side, we have the coefficients of Fourier Series
  %
  \item If $x \in (-\infty, \infty)$
  %
  \begin{align}
    \int^\infty_{-\infty} [f(x)]^2 \text{ dx} & = \int^\infty_{-\infty} [\hat f(\xi)]^2 \text{ d}\xi
  \end{align}
\end{enumerate}

\subsection{Key Property of the Fourier Transform}
%
\begin{align}
  \hat f(\xi) & = \frac{1}{\sqrt{2 \pi}} \int^\infty_{-\infty} f(x) e^{-i x \xi} \text{ dx}\\
  F\left[\frac{df}{dx}\right] & = \frac{1}{\sqrt{2 \pi}} \int^\infty_{-\infty} \frac{df}{dx} e^{- i x \xi} \text{ dx}
\end{align}

Here, let us integrate our derivative to get $f(x)$.
\begin{center}
  \begin{tabular}{c|c}
    $u = e^{-i x \xi}$ & $f(x)$\\
    \hline
    $du = -i \xi e^{-i x \xi}$ & $\frac{df}{dx}$ dx
  \end{tabular}
\end{center}

\begin{align}
  F\left[\frac{df}{dx}\right] & = \frac{1}{\sqrt{2 \pi}}
  \left[
  f(x) e^{-i x \xi}\Big|^\infty_{-\infty} + i \xi \int^\infty_{-\infty} f(x) e^{- i x \xi} \text{ dx}
  \right]
\end{align}

Recall, the $L_1$ norm of $f$ is finite, allowing us to remove the term on the left in our brackets. Here, we are left with the integral:
%
\begin{align}
  F\left[\frac{df}{dx}\right] & = \frac{i \xi}{\sqrt{2 \pi}} \int^\infty_{-\infty} f(x) e^{-i x \xi} \text{ dx}\\
  & = i \xi \hat f(\xi)
\end{align}

So a derivative in real space corresponds to multiplication in Fourier Space.
%
\begin{align}
  F\left[ \frac{d^nf}{dx^n} \right] & = (i \xi)^n \hat f(\xi)
\end{align}

We can use the Fourier Transform to help solve any linear PDE where the domain of a spatial variable is $(-\infty, \infty)$.
