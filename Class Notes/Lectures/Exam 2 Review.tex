\newpage
\section{Exam 2 Review}

\topic{April 4, 2022}

\subsection*{Heat Kernel, Review Number 5}

\begin{itemize}
  \item Use the Heat Kernel to solve $u_t = u_{xx}$ on $x \in (-\infty, \infty)$ and $t \in [0, \infty]$, where $u(x, 0) = x^2$
\end{itemize}

Here, let us consider $\alpha = 1$, then we have:
%
\begin{align}
  u(x, t) & = \frac{1}{\sqrt{4 \pi t}} \int^\infty_{-\infty} f(y) e^{-\frac{(x - y)^2}{4t}}\ \text dy
\end{align}

Here, we are given a function in our initial condition. Let us write it in:

\begin{align}
  u(x, t) & = \frac{1}{\sqrt{4 \pi t}} \int^\infty_{-\infty} y^2 e^{-\frac{(x - y)^2}{4t}}\ \text dy
\end{align}

Here, let us make the power and use w-substitution.

$w = \frac{x - y}{2\sqrt t}$ and $dw = -\frac{1}{2 \sqrt t}\ \text dy$

Now, let us make our substitution:
%
\begin{align}
  u(x, t) & = - \frac{1}{\sqrt \pi} \int^{-\infty}_{\infty} y^2 e^{-w^2}\ \text dw
\end{align}

Here, let us solve for $y$ using our sub. Using our substitution, we find $y = x - 2 \sqrt t w$.
%
\begin{align}
  u(x, t) & = - \frac{1}{\sqrt \pi} \int^{-\infty}_{\infty} (x - 2 \sqrt t w)^2 e^{-w^2}\ \text dw\\
  & = \frac{1}{\sqrt \pi}
  \left[
  x^2 \int^\infty_{-\infty} e^{-w^2}\ \text dw
  - 4 \sqrt t x \int^\infty_{-\infty} w e^{-w^2}\ \text dw
  + 4t \int^\infty_{-\infty} w^2 e^{-w^2}\ \text dw
  \right]
\end{align}

For the first term, we use our formulas to find $\sqrt \pi$.

For the second term, we use $v$ substitution to find $v = -w^2$, $dv = -2w dw$

For the third term, with the following table:
%
\begin{center}
  \begin{tabular}{c|c}
    $w$ & $we^{-w^2}$\\
    \hline
    $1$ & $-\frac{1}{2} e^{-w^2}$
  \end{tabular}
\end{center}

We perform integration by parts.
%
\begin{align}
  u(x, t) & = - \frac{1}{\sqrt \pi} \int^{-\infty}_{\infty} (x - 2 \sqrt t w)^2 e^{-w^2}\ \text dw\\
  & = \frac{1}{\sqrt \pi}
  \left[
  \sqrt \pi
  + 0
  + 4t
  \left(
  -\frac{1}{2} we^{-w^2} \bigg|^\infty_{-\infty} + \frac{1}{2} \int^\infty_{-\infty} e^{-w^2}\ \text dw
  \right)
  \right]\\
  & = x^2 + \frac{4t}{\sqrt \pi} \cdot \frac{1}{2} \sqrt \pi\\
  & = x^2 + 2t
\end{align}

\subsection*{Range of Influence - Review Number 4}

Find the range of influence of the point $(-1, 0)$ given the equation $u_{tt} = 16u_{xx}$. For the same equation, find the domain of dependence for the point $(5, 7)$

\subsection*{Fourier Transform - Review Number 1}

Find the Fourier transform for the following instruction:
%
\begin{itemize}
  \item $f(x) = e^{-|x|}$
\end{itemize}

Recall,
%
\begin{align}
  \hat f(\xi) & = \frac{1}{\sqrt{2\pi}} \int^\infty_{-\infty} f(x) e^{-i x \xi \text dx}\\
  & = \frac{1}{\sqrt 2 \pi} \int^\infty_{-\infty} e^{-|x|} e^{-i x \xi} \text dx
\end{align}

Here, let us rewrite our integral to fit in our absolute value:
%
\begin{align}
  & = \frac{1}{\sqrt{2 \pi}}
  \left[
  \int^0_{-\infty} e^x e^{-i x \xi}\ \text dx +
  \int^\infty_{-\infty} e^{-x} e^{-i x \xi}\ \text dx
  \right]\\
  & = \frac{1}{\sqrt{2 \pi}}
  \left[
  \int^0_{-\infty} e^{x(1 - i \xi)}\ \text dx +
  \int^\infty_{-\infty} e^{x(-1 - i \xi)}\ \text dx
  \right]\\
  & = \frac{1}{\sqrt{2 \pi}}
  \left[
  \int^0_{-\infty} e^{x(1 - i \xi)}\ \text dx +
  \int^\infty_{-\infty} e^{x(-1 - i \xi)}\ \text dx
  \right]\\
  & = \frac{1}{\sqrt{2 \pi}}
  \left[
  \frac{1}{1 - i \xi} e^{x(1 - i \xi)} \bigg|^0_{-\infty} -
  \frac{1}{1 + i \xi} e^{x(-1 - i\xi)} \bigg|^\infty_0
  \right]\\
  & = \frac{1}{\sqrt{2 \pi}}
  \left[
  \frac{1}{1 - i \xi} e^{x} e^{-i x \xi} \bigg|^0_{-\infty} -
  \frac{1}{1 + i \xi} e^{-x}e^{-i x \xi} \bigg|^\infty_0
  \right]
  \\
  & = \frac{1}{\sqrt{2 \pi}}
  \left[
  \frac{1}{1 - i \xi} e^{x} (\cos(x \xi) - i\sin(x \xi) \bigg|^0_{-\infty} -
  \frac{1}{1 + i \xi} e^{-x}(\cos(x \xi) - i\sin(x \xi) \bigg|^\infty_0
  \right]\\
  & = \frac{1}{\sqrt{2 \pi}}
  \left[
  \frac{1}{1 - i\xi} - 0 - 0 + \frac{1}{1 + i \xi}
  \right]
\end{align}

\subsection*{Heat Equation, Delta Functions - Review Number 8}

Find the solution of the heat equation $u_t = u_{xx}$, on $x \in (-\infty, \infty)$ and $t \in [0, \infty)$ if $f(x) = 4 \delta(x - 1) + 2 \delta(x + 5)$

Here, we have the heat kernel once again.
%
\begin{align}
  u(x, t)
  & = \frac{1}{\sqrt{4 \pi t}} \int^\infty_{-\infty} f(y) e^{- \frac{(x - y)^2}{4t}}\ \text dy\\
  & = \frac{1}{\sqrt{4 \pi t}} \int^\infty_{-\infty}
  \left[
  4 \delta(y - 1) + 2 \delta(y + 5)
  \right]
  e^{-\frac{(x - y)^2}{4t}}\ \text dy
\end{align}

Here, our delta function is not zero when $y = 1$ and $y = 5$.
%
\begin{align}
  & = \frac{1}{\sqrt{4 \pi t}}
  \left[
  4 \int^\infty_{-\infty} \delta(y - 1) e^{-\frac{(x - y)^2}{4t}}\ \text dy +
  2 \int^\infty_{-\infty} \delta(y + 5) e^{-\frac{(x - y)^2}{4t}}\ \text dy
  \right]\\
  & = \frac{1}{\sqrt{4 \pi t}}
  \left[
  4 e^{- \frac{(x - 1)^2}{4t}} + 2e^{-\frac{(x + 5)^2}{4t}}
  \right]
\end{align}

Steps to the problem: Identify when the integral is non-zero (When the delta function is zero) and plug in what is left.

\subsection*{Harmonic Function - Review Number 7}

Show that $u(x, y) = xy - x^2 + y^2$ is a harmonic function and find its min and max on $0 \leq x \leq 1$ and $0 \leq y \leq 1$

Here, let us consider the following properties for our $u(x, y)$:
%
\begin{itemize}
  \item $u_x = y - 2x$
  \item $u_{xx} = -2x$
  \item $u_y = x + 2y$
  \item $u_{yy} = 2$
\end{itemize}

Here, let us find the min/max on one boundary.
\begin{itemize}
  \item $y = 0 : u(x, 0) = -x^2$

  min: $-1$, max: $0$
  \item $y = 1 : u(x, 1) = x - x^2 + 1$

  The points are not obvious, $u^\prime(x, 1) = 1 - 2x$, we have our critical number at $x = \frac{1}{2}$.

  $u(\frac{1}{2}, 1)$, $u(0, 1)$, $u(1, 1)$ are evaluated and checked for min/max. Always check the endpoints.
\end{itemize}

\subsection*{Fourier Transform - Review Number 3}

Use Fourier Transform to solve $u_{xt} = 4u_x$, where $u(x, 0) = xe^{-x^2}$, with $x \in (-\infty, \infty)$ and $t \in [0, \infty)$.

\begin{enumerate}
  \item Fourier Transform on both sides.
  %
  \begin{align}
    F[u_{xt}] & = i \xi \hat u_t\\
    F[4u_x] & = 4 i \xi \hat u
  \end{align}

  With every $x$ partial, we obtain an $i \xi$. For each $t$ partial, we keep $t$.
  %
  \begin{align}
    i \xi \hat u_t & = 4 i \xi \hat u\\
    \hat u_t & = 4 \hat u
  \end{align}

  Recall our initial condition, $f(X) = u(x, 0)$,
  %
  \begin{align}
    F[f(x)] & = \hat f(\xi)
  \end{align}

  \item Evaluate
  \begin{align}
    \hat u(\xi, t) & = A(\xi) e^{4t}
  \end{align}

  When we plug in $t = 0$ for our initial condition, we simply get:
  %
  \begin{align}
    \hat u(\xi, 0) & = A(\xi) = \hat f(\xi)
  \end{align}

  Now, we can write:
  %
  \begin{align}
    \hat u(\xi, t) & = \hat f(\xi) e^{4t}
  \end{align}

  \item Retransform
  \begin{align}
    u(x, t) & = \frac{1}{\sqrt{2 \pi}}
    \int^\infty_{-\infty} \hat f(\xi) e^{4t} e^{i x \xi}\ \text d\xi\\
    & = \frac{e^{4t}}{\sqrt{2 \pi}}
    \int^\infty_{-\infty} \hat f(\xi) e^{i x \xi}\ \text d \xi\\
    & = e^{4t} f(x)
  \end{align}
\end{enumerate}
