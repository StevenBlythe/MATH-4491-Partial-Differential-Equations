\section{Wave Equation on Semi-Infinite Domain}
\begin{itemize}
  \item $x \in [0, \infty), t \in [0, \infty)$
  \item $u_{tt} = c^2 u_{xx}$
  \item $u(x, 0) = f(x)$
  \item $u_t(x, 0) = g(x)$
  \item $u(0, t) = 0$
\end{itemize}

Recall: If $x \in (-\infty, \infty)$, we use d'Alembert's Formula:
%
\begin{align}
  u(x, t) & = \frac{1}{2} \left[ f(x + ct) + f(x - ct) \right] +
  \frac{1}{2c} \int^{x + ct}_{x - ct} g(y)\ \text dy
\end{align}

We would like to use the solution to the wave equation for
$x \in (-\infty, \infty)$
to help solve the wave equation when
$x \in [0, \infty)$.

To do this, we use the odd extension of the initial conditions:
%
\begin{align}
  \twiddle f(x) & =
  \begin{cases}
    f(x) & x > 0\\
    0 & x = 0\\
    -f(-x) & x < 0
  \end{cases}\\
  \twiddle g(x) & =
  \begin{cases}
    f(x) & x > 0\\
    0 & x = 0\\
    -f(-x) & x < 0
  \end{cases}
\end{align}

This system can be solved using d'Alembert's Formula:
%
\begin{align}
  u(x, t) & = \frac{1}{2} \left[ \twiddle f(x + ct) + \twiddle f(x - ct) \right]
  + \frac{1}{2c} \int^{x + ct}_{x - ct} \twiddle g(y)\ \text dy
\end{align}

\note This solves some PDE on $[0, \infty)$, since it solves it on
$(-\infty, \infty)$.

\note
$u(0, t) = \frac{1}{2} \left[ \twiddle f(ct) + \twiddle f(-ct)\right]
+ \frac{1}{2} \int^{ct}_{-ct} \twiddle g(y)\ \text dy$, but our integral will
zero out since it is odd. In addition, since our functions are odd, the
$\twiddle f$ will cancel out as well.
%
\topic{Case 1}: $x - ct > 0$
%
\begin{align}
  u(x, t)
  & = \frac{1}{2} \left[ \twiddle f(x + ct) + \twiddle f(x - ct) \right] +
  \frac{1}{2c} \int^{x + ct}_{x - ct} \twiddle g(y)\ \text dy\\
  & = \frac{1}{2} \left[ f(x + ct) + f(x - ct) \right] +
  \frac{1}{2c} \int^{x + ct}_{x - ct} g(y)\ \text dy
\end{align}

Staying on the right, we do not hit a wall and nothing changes.

\topic{Case 2}: $x - ct < 0$
%
\begin{align}
  u(x, t)
  & = \frac{1}{2} \left[ \twiddle f(x + ct) + \twiddle f(x - ct) \right] +
  \frac{1}{2c} \int^{x + ct}_{x - ct} \twiddle g(y)\ \text dy\\
  & = \frac{1}{2} \left[ f(x + ct) - f(ct - x) \right] +
  \frac{1}{2c}
  \left[
  \int^{0}_{x - ct} \twiddle g(y)\ \text dy +
  \int^{x + ct}_{0} \twiddle g(y)\ \text dy
   \right]\\
   & = \frac{1}{2} \left[ f(x + ct) - f(ct - x) \right] +
   \frac{1}{2c}
   \left[
   -\int^{0}_{x - ct} g(-y)\ \text dy +
   \int^{x + ct}_{0} g(y)\ \text dy
    \right]
\end{align}

Here, let us perform substitution with $w = -y$,
%
\begin{align}
  & = \frac{1}{2} \left[ f(x + ct) - f(ct - x) \right] +
  \frac{1}{2c}
  \left[
  \int^{0}_{ct - x} g(w)\ \text dw +
  \int^{x + ct}_{0} g(y)\ \text dy
  \right]\\
  & = \frac{1}{2} \left[ f(x + ct) - f(ct - x) \right] +
  \frac{1}{2c}
  \left[
  \int^{x + ct}_{ct - x} g(y)\ \text dy
  \right]
\end{align}

If we look at the domain of dependence, the left line reflect back to our domain and the line is represented as $ct - x$.

\ex $u_{tt} = u_{xx}, x \in [0, \infty)$
%
\begin{align}
  u(x, 0) & =
  \begin{cases}
    1 & 4 < x < 5\\
    0 & \text{otherwise}
  \end{cases}\\
  u_t(x, 0) & = 0
\end{align}

\section{D'Alembert's Formula on a Bounded Domain}
\begin{center}
  $
  u_{tt} = c^2 u_{xx} \quad
  u(x, 0) = f(x) \quad
  u_t(x, 0) = g(x) \quad
  0 \leq x \leq L, t \in [0, \infty)
  $
\end{center}

How do we find the solution to the wave equation on
$(-\infty, \infty)$
to find a solution on $[0, L]$?

We extend the initial conditions to be odd and periodic with period $2L$.
%
\begin{align}
  \twiddle f(x) & =
  \begin{cases}
    f(x) & 0 < x < L\\
    0 & x = 0\\
    -f(-x) & -L < x < 0
  \end{cases}
\end{align}

Recall, we considered boundary conditions. Here, let us define boundary conditions as:
\begin{center}
  $u(0, t) = u(L, t) = 0$
\end{center}

Here, let us enforce $\twiddle f(x + 2L) = \twiddle f(x)$ to force periodicity. For $\twiddle g(x)$, let us write:
%
\begin{align}
  \twiddle g(x) & =
  \begin{cases}
    g(x) & 0 < x \leq L\\
    0 & x = 0\\
    -g(-x) & -L < x < 0
  \end{cases}
\end{align}

The solution will be:
%
\begin{align}
  u(x, t) & = \frac{1}{2}
  \left[ \twiddle f(x + ct) + \twiddle f(x - ct) \right] +
  \frac{1}{2} \int^{x + ct}_{x - ct} \twiddle g(y)\ \text dy\\
  u(0, t) & = \frac{1}{2}
  \left[ \twiddle f(ct) + \twiddle f(-ct) \right]\\
  u(L, t) & = \frac{1}{2}
  \left[ \twiddle f(L + ct) + \twiddle f(L - ct) \right] +
  \frac{1}{2c} \int^{L + ct}_{L - ct} \twiddle g(y)\ \text dy\\
  & = \frac{1}{2}
  \left[ \twiddle f(ct - L) + \twiddle f(L - ct) \right] +
  \frac{1}{2c} \left(
  \int^{0}_{L - ct} \twiddle g(y)\ \text dy +
  \int^{L + ct}_{0} \twiddle g(y)\ \text dy
  \right)\\
  & = \frac{1}{2}
  \left[ \twiddle -f(L - ct) + \twiddle f(L - ct) \right] +
  \frac{1}{2c} \left(
  \int^{0}_{ct - L} \twiddle g(y)\ \text dy +
  \int^{L + ct}_{0} \twiddle g(y)\ \text dy
  \right)\\
  & = \frac{1}{2c}
  \left(
  \int^0_{ct + L} \twiddle g(y)\ \text dy+
  \int_0^{L + ct} \twiddle g(y)\ \text dy
  \right)
\end{align}

We added an integral of length $2L$, which is $0$ since $\twiddle g$ is odd.
