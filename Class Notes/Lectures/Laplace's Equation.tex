\newpage
\section{Laplace's Equation}

\subsection{General Dirichlet Boundary Conditions}
\begin{itemize}
  \item $u(x, 0) = f_1(x)$
  \item $u(x, M) = f_2(x)$
  \item $u(0, y) = f_3(y)$
  \item $u(L, y) = f_4(y)$
\end{itemize}

Write our solution as the following:
\begin{align}
  u(x, y) = u_1(x, y) + u_2(x, y) + u_3(x, y) + u_4(x, y)
\end{align}

\begin{multicols}{4}
  \begin{itemize}
    \item $\Delta u_1 = 0$
    \begin{itemize}
      \item $u_1(x, 0) = f_1(x)$
      \item $u_1(x, M) = 0$
      \item $u_1(0, y) = 0$
      \item $u_1(L, y) = 0$
    \end{itemize}
    \item $\Delta u_1 = 0$
    \begin{itemize}
      \item $u_2(x, 0) = 0$
      \item $u_2(x, M) = f_2(x)$
      \item $u_2(0, y) = 0$
      \item $u_2(L, y) = 0$
    \end{itemize}
    \item $\Delta u_1 = 0$
    \begin{itemize}
      \item $u_3(x, 0) = 0$
      \item $u_3(x, M) = 0$
      \item $u_3(0, y) = f_3(y)$
      \item $u_3(L, y) = 0$
    \end{itemize}
    \item $\Delta u_1 = 0$
    \begin{itemize}
      \item $u_4(x, 0) = 0$
      \item $u_4(x, M) = 0$
      \item $u_4(0, y) = 0$
      \item $u_4(L, y) = f_4(y)$
    \end{itemize}
  \end{itemize}
\end{multicols}

This method works because Laplace's equation is linear.

We have already seen that for $u_2$ :
%
\begin{align}
  u_2(x, y)
  & = \sum^\infty_{n = 1} B_n \sin\left( \frac{n \pi x}{L} \right)
  \sinh\left( \frac{n \pi y}{L} \right)\\
  u_2(x, M) & = f_2(x)\\
  & \Rightarrow
  \sum^\infty_{n = 1} B_n \sin\left ( \frac{n \pi x}{L} \right)
  \sinh\left( \frac{n \pi M}{L} \right)\\
  & = f(x)
\end{align}

Here, $B_n \sinh\left( \frac{n \pi M}{L}\right)$ is the coefficient for Laplace.
%
\begin{align}
  B_n \sinh \left( \frac{n \pi M}{L} \right)
  & = \frac{2}{L} \int^L_0 f_2(x) \sin\left(\frac{n \pi x}{L}\right) \dx\\
  B_n
  & = \frac{2}{L \sinh\left(\frac{n \pi M}{L}\right)}
  \int^L_0 f_2(x) \sin\left(\frac{n \pi x}{L}\right) \dx
\end{align}

Similarly, for $u_4$,
%
\begin{align}
  u_4(x, y)
  & = \sum^\infty_{n = 1}
  D_n \sin\left(\frac{n \pi y}{M}\right)\sinh\left(\frac{n \pi x}{L}\right)\\
  u_4(L, y) & = f_4(y)\\
  & \Rightarrow \sum^\infty_{n = 1} D_n
  \sin\left(\frac{n \pi y}{M}\right) \sinh\left(\frac{n \pi L}{M}\right)\\
  & = f_4(y)
\end{align}

\bigbreak

\topic{February 11, 2022}

Recall we are consider $u = u_1 + u_2 + u_3 + u_4$. Let us write:
%
\begin{align}
  u_4(x, y)
  & = \sum^\infty_{n = 1} D_n
  \sin\left( \frac{n \pi y}{M}\right)\sinh\left(\frac{n \pi x}{M}\right)\\
  u_4(L, y)
  & = f_4(y)\\
  & = \sum^\infty_{n = 1} D_n \sin\left( \frac{n \pi y}{M} \right)
  \sinh\left( \frac{n \pi L}{M} \right) = f_4(y)
\end{align}

Recall, our coefficient is $D_n$ and the $\sinh$ function.
%
\begin{align}
  D_n \sinh\left( \frac{n \pi L}{M} \right) & =
  \frac{2}{M} \int^M_0 f_4(y) \sin\left( \frac{n \pi y}{M} \right) dy\\
  %
  D_n & = \frac{2}{M \sinh\left( \frac{n \pi L}{M} \right)} \int^M_0 f_4(y) \sin\left( \frac{n \pi y}{M} \right) dy
\end{align}

Let's look at $u_1$:
\begin{itemize}
  \item $\Delta u_1 = 0$
  \item $u_1(x, 0) = f_1(x)$
  \item $u_1(x, M) = 0$
  \item $u_1(0, y) = 0$
  \item $u_1(L, y) = 0$
\end{itemize}

\begin{enumerate}
  \item Here, let us consider $\Delta u_1 = 0$:

  \begin{align}
    \frac{X^{\prime\prime}}{X} & = -\frac{Y^{\prime\prime}}{y} = -\lambda\\
    u_1(x, y) & = X(x)Y(y)
  \end{align}

  Boundary Conditions
  %
  \begin{align}
    u_(x, M) & = 0 \Rightarrow X(x)Y(M) = 0 \Rightarrow Y(M) = 0\\
    u_(0, M) & = 0 \Rightarrow X(0)Y(M) = 0 \Rightarrow X(0) = 0\\
    u_(L, M) & = 0 \Rightarrow X(L)Y(M) = 0 \Rightarrow X(L) = 0\\
  \end{align}

\item $\lambda_n = \left( \frac{n \pi}{L} \right)^2$, $X_n(x) = \sin\left( \frac{n \pi x}{L} \right)$

\item Solve for $y$:
%
\begin{align}
  \frac{Y^{\prime\prime}}{Y_n} & = \lambda_n\\
  Y^{\prime\prime}_n & = \left( \frac{n \pi}{L} \right)^2 Y_n\\
  Y_n(y) & = C\sinh \left( \frac{n \pi y}{L} \right) + D \cosh \left( \frac{n \pi y}{L} \right)
\end{align}

Let us see what we have with $Y(m) = 0$:
%
\begin{align}
  C \sinh \left( \frac{n \pi M}{L} \right) + D \cosh \left( \frac{n \pi M}{L} \right) & = 0
\end{align}

Here, this does not work for us. Let us go back and change our $Y_n(y)$:
%
\begin{align}
  Y_n(y) & = C \sinh \left(\frac{n \pi (M - y)}{L} \right) + D \cosh \left( \frac{n \pi (M - y)}{L} \right)
\end{align}

Now, let us use our $Y$:
%
\begin{align}
  Y_n(M) & = C \sinh \left(\frac{n \pi (M - M)}{L} \right) + D \cosh \left( \frac{n \pi (M - M)}{L} \right)\\
  & = D = 0\\
  Y_n(y) & = \sinh\left( \frac{n \pi (M -y)}{L} \right)
\end{align}

\item Let us combine:
%
\begin{align}
  u_m(x, y) & = \FS \sinh\left( \frac{ n \pi (M - y)}{L} \right)
\end{align}

By linearity
%
\begin{align}
  u_1(x, y) & = \sum^\infty_{n = 1} A_n \FS \sinh\left( \frac{n \pi(M -y)}{L} \right)
\end{align}

\item Find coefficients:
%
\begin{align}
  u_1(x, 0) & = f_1(x)\\
  & = \sum^\infty_{n = 1} A_n \FS \sinh\left( \frac{n \pi M}{L} \right) = f_1(x)
\end{align}

Once more, we have our coefficient with $A_n$ and $\sinh$.
%
\begin{align}
  A_n \sinh \left( \frac{n \pi M}{L} \right) & = \frac{2}{L} \int^L_0 f_1(x) \FS \dx\\
  A_n & = \frac{2}{L \sinh \left( \frac{n \pi M}{L} \right)} \int^L_0 f_1(x) \FS \dx
\end{align}
\end{enumerate}

\subsection{Wave Equation}

\begin{center}
  \begin{tabular}{r|c|l}
    t & \# &\\
    & \# & \\
    $u = H_1$ & \# & $u = H_2$\\
    & \# &\\
    \hline
  \end{tabular}\\
  0 \quad L\\
  $u(x, 0) = f(x)$\\
  $u_t(x, 0) = g(x)$
\end{center}

Steady-state:
%
\begin{align}
  u_t & = 0 \Rightarrow u_{tt} = 0 \Rightarrow u_{xx} = 0 \Rightarrow u = \frac{H_2 - H_1}{L} x + H_1
\end{align}

Try a solution of the form : $u(x, t) = w(x, t) + u(x, \infty)$. Therefore, $w(x, t) = u(x, t) - u(x, \infty)$.
%
\begin{align}
  u_{tt} & = c^2 u_{xx} \Rightarrow w_{tt} = c^2 w_{xx}
\end{align}

Boundary conditions:
%
\begin{align}
  w(0, t) & = u(0, t) - u(0, \infty) = H_1 - H_1 = 0\\
  w(L, t) & = u(L, t) - u(L, \infty) = H_2 - H_2 = 0
\end{align}

Initial Conditions
%
\begin{align}
  w(x, 0) & = u(x, 0) - u(x, \infty) = f(X) - \frac{H_2 - H_1}{L} x + H_1\\
  w_t(x, t) & = u_t(x, t) \Rightarrow w_t(x, 0) = u_t(x, 0) = g(x)
\end{align}
%\end{enumerate}
