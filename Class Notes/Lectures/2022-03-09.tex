\topic{Wave Equation Solutions}
%
\begin{itemize}
  \item $u_t = u_xx$
  \item $x \in (-\infty, \infty)$, $t \in [0, \infty)$
  \item $u(x, 0) = f(x)$
  \item $\displaystyle u(x, t) = \frac{1}{\sqrt{4 \pi t}} \int^\infty_{-\infty} f(y) e^{- \frac{(x - y)^2}{4t}} \text{ dy}$
\end{itemize}

Let the initial condition be a ``delta function,'' $\delta(x)$.

\bigbreak

\emph{What is a delta function, $\delta(x)$?}

It has two main properties:
%
\begin{enumerate}
  \item $\delta(x) = 0$, $x \neq 0$.
  \item $\displaystyle \int^\infty_{-\infty} \delta(x) \text{ dx } = 1$
\end{enumerate}

The ``mass'' is centered at $x = 0$. The delta function is not a function because $\delta(0) = ?$. Actually, the delta function is a measure.

\underline{Calculations with Delta Functions}
%
\begin{align}
  \int^\infty_{-\infty} \delta(y) g(x - y) \text{ dy}
  & = \int^\infty_{-\infty} \delta(x- y) g(y) \text{ dy} = g(x)
\end{align}

Here, $\delta(y)$ is zero except when $y = 0$ and $\delta(x - y) = 0$ except when $x = y$.

Here, we have a convolution $\delta * g$, where our variables can switch.

What do we expect when $f(x) = \delta(x)$?

When $t = 0$, our area is the $t$ axis: $|$, however, as $t \to \infty$, then the area slowly flattens, akin to a candle.

Mathematically, what do we expect?
%
\begin{align}
  u(x, t) & = \frac{1}{\sqrt{4 \pi t}} \int^\infty_{-\infty} \delta(y) e^{- \frac{(x - y)^2}{4t}} \text{ dy}\\
  & = \frac{1}{\sqrt{4 \pi t}} e^{- \frac{x^2}{4 t}}
\end{align}

The $t$'s impact in the fraction reduces the amplitude and the $t$ in the exponent flattens out the curve.

This is the Gaussian Normal Distributions

What if $f(x) = 7 \delta(x) + 5 \delta(x - 3)$?
%
\begin{align}
  u(x, t) & = \frac{1}{\sqrt{4 \pi t}} \int^\infty_{-\infty} [7 \delta(y) + 5 \delta(y - 3)] e^{-\frac{(x - y)^2}{4 t}} \text{ dy}\\
  & = \frac{1}{\sqrt{4 \pi t}}
  \left[
    \int^\infty_{-\infty} 7 \delta(y) e^{- \frac{(x - y)^2}{4 t}} \text{ dy} +
    \int^\infty_{-\infty} 5 \delta(y - 3) e^{-\frac{(x - y)^2}{4t}} \text{ dy}
  \right]\\
  & = \frac{1}{\sqrt{4 \pi t}}
  \left[
    7e^{- \frac{x^2}{4 t}} + 5 e^{-\frac{(x - 3)^2}{4 t}}
  \right]
\end{align}

So for a general $f(x)$, think of $f(x)$ as a bunch of delta functions.

%______________________________________________________________________________%

\topic{Conservation of Energy}

The amount of heat stamp? constant
%
\begin{align}
  \frac{d}{dt} \int^\infty_{-\infty} u(x, t) \text{ dx}
  & = \int^\infty_{-\infty} u(x, t) \text{ dx}\\
  & = \alpha^2 \int^\infty_{-\infty} u_{xx} (x, t) \text{ dx}\\
  & = \alpha^2 u_x(x, t) \Big|^\infty_{-\infty}
\end{align}

\underline{Recall} We know $\lim_{x \pm \infty} u(x, t) = 0$, therefore the rate of change at both infinities is zero.
%
\begin{align}
  \alpha^2 u_x(x, t) \Big|^\infty_{-\infty} & = 0
\end{align}

\bigbreak

\topic{Dependence on Initial Condition}

The entire initial condition affects the solution at any point.

Range of Influence is the entire $(x, t)$ plane.

The solution to the heat equation for any fixed $t > 0$ is $C^\infty$ even if the initial condition is discontinuous.

\bigbreak

\topic{Reversal of Time} : Disaster

Solving the heat equation in backwards time does not work because even slight changes in the initial condition lead to drastically different solutions. Equations that exhibit this behavior are called unstable or ill-posed

\bigbreak

\topic{Expansion to Multi-Dimensions}

The solution to the heat equation can easily extend to $x \in \R^n (u_t = \Delta u)$.
%
\begin{align}
  u(x, t) & = \frac{1}{(4 \pi t)^{n/2}} \int^\infty_{-\infty} \ldots \int^\infty_{-\infty} f(y) e^{-\frac{(|x - y|^2)}{4t}} \text{ d}y_1 \text dy_2 \ldots \text dy_n
\end{align}

Here, $|x - y|$ can be onsidered the norm.
%
\begin{align}
  |x - y|^2 = (x_1 - y_1)^2 + (x_2 - y_2)^2 + \ldots + (x_n - y_n)^2
\end{align}

This is how heat looks like in multiple dimensions.

\bigbreak

We know the solution to $u_t = u_{xx}, u(x, 0) = f(x)$ is
%
\begin{align}
  u(x, t) & = \frac{1}{\sqrt{4 \pi t}} \int^\infty_{-\infty} f(y) e^{-\frac{(x - y)^2}{4 t}} \text{ dy}\\
  & = \int^\infty_{-\infty} K_t (x - y) f(y) \text{ dy}
\end{align}

Where we have Green's Function:
%
\begin{align}
  K_t(x - y) & = \frac{1}{\sqrt{4 \pi t}} e^{-\frac{(x - y)^2}{4t}}
\end{align}

$K_t$ is called the heat kernel.
