\newpage
\section{More general Boundary Conditions}

\topic{February 9, 2022}

Before, we used to deal with boundary conditions where $u$ starts and end at $0$. Now, let us consider boundary conditions where $u$ can start at any number.

% To view different conditions, credits to @199071614166499328.

The steady state solution is the following:
%
\begin{align}
  \frac{T_2 - T_1}{L} x + T_1
\end{align}

Ideas, we want to try the following: $u(x, t) = w(x, t) + u(x, \infty)$. Note that $\infty$ is the steady state. Here, as time goes to infinity, $w(x, t)$ cancels out. We want to solve for $w$:

We must specify $w$.
%
\begin{align}
  u_t & = \alpha^2 u_{xx}\\
  w_t & = \alpha^2 w_{xx}
\end{align}

We want to find out more about the boundary conditions. We also need the initial conditions to solve this.

\underline{Boundary Conditions}
%
\begin{itemize}
  \item $u(0, t) = T_1$
  \item $u(L, t) = T_2$
\end{itemize}

Let us consider the first boundary condition.
%
\begin{align}
  w(0, t) + u(0, \infty) & = T_1
\end{align}

Here, we know that for the steady state, $x$ is $T_1$ at $x = 0$. Therefore,
%
\begin{align}
  w(0, t) & = 0
\end{align}

We repeat with our second bullet.
%
\begin{align}
  u(L, t &) = T_2 \Rightarrow\\
  w(L, t) + u(L, \infty) & = T_2\\
  w(L, t) & = 0
\end{align}

\underline{Initial Conditions}
%
\begin{align}
  u(x, 0) & = f(x) \Rightarrow\\
  w(x, 0) + u(x, \infty) & = f(x)\\
  w(x, 0) & = f(x) - u(x, \infty)\\
  w(x, 0) & = f(x) - \left[ \frac{T_2 - T_1}{L} x + T_1 \right]
\end{align}

\ex

Solve
$u_t = u_{xx}$,
$u(0, t) = 2$,
$u(4, t) = 3$,
$u(x, 0) = -6 \sin(\pi x) + \frac{x}{4} + 2$.

First, find the steady-state solution:
%
\begin{align}
  u(x, \infty) & = \frac{3 - 2}{4} x + 2\\
  & = \frac{x}{4} + 2
\end{align}

Now, we assume $u(x, t) = w(x, t) + u(x, \infty)$.
We can make the following assumption:
%
\begin{align}
  u_t = u_{xx} & \Rightarrow w_t = w_{xx}
\end{align}

\underline{Boundary Conditions}
\begin{align}
  u(0, t) = 2 & \Rightarrow w(0, t) = u(0, t) - u(0, \infty) = 2 - 2 = 0\\
  u(4, t) = 3 & \Rightarrow w(4, t) = u(4, t) - u(4, \infty) = 3 - 3 = 0
\end{align}

Here, we plug in our $x$ into our steady-state solution and get 2, 3.

\underline{Initial Conditions}

\begin{align}
  w(x, 0) & = u(x, 0) - u(x, \infty)\\
  & = -\sin(\pi x) + \frac{x}{4} + 2 - \left(\frac{x}{4} + 2\right)\\
  & = -\sin(\pi x)
\end{align}

Now, solve for $w$:
\begin{enumerate}
  \item Assume $w(x, t) = X(x)T(t)$

  \underline{Boundary Conditions}
  \begin{align}
    w(0, t) = 0 & \Rightarrow X(0)T(t) = 0 \Rightarrow X(0) = 0\\
    w(4, t) = 0 & \Rightarrow X(4)T(t) = 0 \Rightarrow X(4) = 0\\
    w_t & = w_{xx} \Rightarrow XT^\prime\\
    & = X^{\prime\prime } T \Rightarrow \frac{T^\prime}{T}
    = \frac{X^{\prime\prime}}{X} = - \lambda
  \end{align}

  \item Solve for $X$ :
  %
  \begin{align}
    \frac{X^{\prime\prime}}{X} & = - \lambda \Rightarrow\\
    X^{\prime\prime} & = -\lambda X\\
    X(0) & = X(4) = 0
  \end{align}

  Here, let us write our general equation:
  %
  \begin{align}
    X(x) & = A \sin(\sqrt \lambda x) + B \cos(\sqrt \lambda x)\\
    X(0) = 0 & \Rightarrow B = 0\\
    X(4) = 0 & \Rightarrow A \sin(\sqrt \lambda 4) = 0\\
    & \Rightarrow \sqrt \lambda 4 = n \pi\\
    & \Rightarrow \lambda_n = \left(\frac{n \pi}{4} \right)^2\\
    & \Rightarrow X_n(x) = \sin\left( \frac{n \pi x}{4} \right)
  \end{align}

  \item Solve for $T$:
  %
  \begin{align}
    \frac{T^\prime_n}{T_n} & = -\lambda_n\\
    T^\prime_n & = -\left( \frac{n \pi}{4} \right)^2 T_n\\
    T_n(t) & = e^{- \left( \frac{n \pi}{4} \right)^2 t}
  \end{align}

  \item Combine to find $w_n$ and $w$:
  %
  \begin{align}
    w_n(x, t) & = \sin\left(\frac{n \pi x}{4} \right)e^{-\left(\frac{n \pi}{4} \right)^2 t}
  \end{align}

  By linearity,
  %
  \begin{align}
    w(x, t) & = \sum^\infty_{n = 1} A_n \sin\left(\frac{n \pi x}{4} \right)e^{-\left( \frac{n \pi}{4} \right)^2 t}
  \end{align}

  \item Find coefficients using Initial Condition
  %
  \begin{align}
    w(x, 0) & = -\sin(\pi x)\\
    w(x, 0) & = \sum^\infty_{n = 1} A_n \sin\left(\frac{n \pi x}{4}\right)\\
    & = -6 \sin(\pi x)\\
    w(x, t) & = -6 \sin(\pi x) e^{-\pi^2 t}
  \end{align}
  Here, $a_4 = -6$.
  %
  \begin{align}
    u(x, t) & = -6 \sin(\pi x)e^{-\pi^2 t} + \frac{x}{4} + 2
  \end{align}
\end{enumerate}
