\documentclass{article}
\usepackage{stocktonmacros} % Personal macros, import package
\setlist[itemize]{noitemsep, topsep=0pt} % Must keep within the same document as \begin{document}

\begin{document}

\maketitle

\newpage

\tableofcontents
%
\newpage
%
\import{Lectures/}{Partial Differential Equations - An Introduction}
\import{Lectures/}{Heat, Wave Equation}
\import{Lectures/}{Approximating Functions with Other Functions}
\import{Lectures/}{Heat Equation}
\import{Lectures/}{Wave Equation}
\import{Lectures/}{Boundary Conditions}
\import{Lectures/}{Laplace's Equation}
\import{Lectures/}{Polar Coordinates}
\import{Lectures/}{Complex Analysis, Fourier Transform}
\import{Lectures/}{Transport Equation}
\import{lectures/}{Image Processing}
\import{Lectures/}{Conservation Laws}
%______________________________________________________________________________%

\subsection*{Harmonic Function - Review Number 7}

Show that $u(x, y) = xy - x^2 + y^2$ is a harmonic function and find its min and max on $0 \leq x \leq 1$ and $0 \leq y \leq 1$

Here, let us consider the following properties for our $u(x, y)$:
%
\begin{itemize}
  \item $u_x = y - 2x$
  \item $u_{xx} = -2x$
  \item $u_y = x + 2y$
  \item $u_{yy} = 2$
\end{itemize}

Here, let us find the min/max on one boundary.
\begin{itemize}
  \item $y = 0 : u(x, 0) = -x^2$

  min: $-1$, max: $0$
  \item $y = 1 : u(x, 1) = x - x^2 + 1$

  The points are not obvious, $u^\prime(x, 1) = 1 - 2x$, we have our critical number at $x = \frac{1}{2}$.

  $u(\frac{1}{2}, 1)$, $u(0, 1)$, $u(1, 1)$ are evaluated and checked for min/max. Always check the endpoints.
\end{itemize}

\subsection*{Fourier Transform - Review Number 3}

Use Fourier Transform to solve $u_{xt} = 4u_x$, where $u(x, 0) = xe^{-x^2}$, with $x \in (-\infty, \infty)$ and $t \in [0, \infty)$.

\begin{enumerate}
  \item Fourier Transform on both sides.
  %
  \begin{align}
    F[u_{xt}] & = i \xi \hat u_t\\
    F[4u_x] & = 4 i \xi \hat u
  \end{align}

  With every $x$ partial, we obtain an $i \xi$. For each $t$ partial, we keep $t$.
  %
  \begin{align}
    i \xi \hat u_t & = 4 i \xi \hat u\\
    \hat u_t & = 4 \hat u
  \end{align}

  Recall our initial condition, $f(X) = u(x, 0)$,
  %
  \begin{align}
    F[f(x)] & = \hat f(\xi)
  \end{align}

  \item Evaluate
  \begin{align}
    \hat u(\xi, t) & = A(\xi) e^{4t}
  \end{align}

  When we plug in $t = 0$ for our initial condition, we simply get:
  %
  \begin{align}
    \hat u(\xi, 0) & = A(\xi) = \hat f(\xi)
  \end{align}

  Now, we can write:
  %
  \begin{align}
    \hat u(\xi, t) & = \hat f(\xi) e^{4t}
  \end{align}

  \item Retransform
  \begin{align}
    u(x, t) & = \frac{1}{\sqrt{2 \pi}}
    \int^\infty_{-\infty} \hat f(\xi) e^{4t} e^{i x \xi}\ \text d\xi\\
    & = \frac{e^{4t}}{\sqrt{2 \pi}}
    \int^\infty_{-\infty} \hat f(\xi) e^{i x \xi}\ \text d \xi\\
    & = e^{4t} f(x)
  \end{align}
\end{enumerate}
\end{document}
