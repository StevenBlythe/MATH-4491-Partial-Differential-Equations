\documentclass{article}
\usepackage{stocktonmacros} % Personal macros, import package
\setlist[itemize]{noitemsep, topsep=0pt} % Must keep within the same document as \begin{document}

\begin{document}

%\maketitle
%%
%\newpage
%%
%\tableofcontents
%%%
%\newpage
%%%
%\import{Lectures/}{Partial Differential Equations - An Introduction}
%\import{Lectures/}{Heat, Wave Equation}
%\import{Lectures/}{Approximating Functions with Other Functions}
%\import{Lectures/}{Heat Equation}
%\import{Lectures/}{Wave Equation}
%\import{Lectures/}{Boundary Conditions}
%\import{Lectures/}{Laplace's Equation}
%\import{Lectures/}{Polar Coordinates}
%\import{Lectures/}{Complex Analysis, Fourier Transform}
%\import{Lectures/}{Transport Equation}
%\import{Lectures/}{Image Processing}
%\import{Lectures/}{Exam 2 Review}
%\import{Lectures/}{Conservation Laws}
\import{Lectures/}{Wave Equation on Semi-Infinite Domain}
%______________________________________________________________________________%
\newpage
% Conservation Law?

\section{Wave Equation on Semi-Infinite Domain}
\begin{itemize}
  \item $x \in [0, \infty), t \in [0, \infty)$
  \item $u_{tt} = c^2 u_{xx}$
  \item $u(x, 0) = f(x)$
  \item $u_t(x, 0) = g(x)$
  \item $u(0, t) = 0$
\end{itemize}

Recall: If $x \in (-\infty, \infty)$, we use d'Alembert's Formula:
%
\begin{align}
  u(x, t) & = \frac{1}{2} \left[ f(x + ct) + f(x - ct) \right] +
  \frac{1}{2c} \int^{x + ct}_{x - ct} g(y)\ \text dy
\end{align}

We would like to use the solution to the wave equation for
$x \in (-\infty, \infty)$
to help solve the wave equation when
$x \in [0, \infty)$.

To do this, we use the odd extension of the initial conditions:
%
\begin{align}
  \twiddle f(x) & =
  \begin{cases}
    f(x) & x > 0\\
    0 & x = 0\\
    -f(-x) & x < 0
  \end{cases}\\
  \twiddle g(x) & =
  \begin{cases}
    f(x) & x > 0\\
    0 & x = 0\\
    -f(-x) & x < 0
  \end{cases}
\end{align}

This system can be solved using d'Alembert's Formula:
%
\begin{align}
  u(x, t) & = \frac{1}{2} \left[ \twiddle f(x + ct) + \twiddle f(x - ct) \right]
  + \frac{1}{2c} \int^{x + ct}_{x - ct} \twiddle g(y)\ \text dy
\end{align}

\note This solves some PDE on $[0, \infty)$, since it solves it on
$(-\infty, \infty)$.

\note
$u(0, t) = \frac{1}{2} \left[ \twiddle f(ct) + \twiddle f(-ct)\right]
+ \frac{1}{2} \int^{ct}_{-ct} \twiddle g(y)\ \text dy$, but our integral will
zero out since it is odd. In addition, since our functions are odd, the
$\twiddle f$ will cancel out as well.
%
\topic{Case 1}: $x - ct > 0$
%
\begin{align}
  u(x, t)
  & = \frac{1}{2} \left[ \twiddle f(x + ct) + \twiddle f(x - ct) \right] +
  \frac{1}{2c} \int^{x + ct}_{x - ct} \twiddle g(y)\ \text dy\\
  & = \frac{1}{2} \left[ f(x + ct) + f(x - ct) \right] +
  \frac{1}{2c} \int^{x + ct}_{x - ct} g(y)\ \text dy
\end{align}

Staying on the right, we do not hit a wall and nothing changes.

\topic{Case 2}: $x - ct < 0$
%
\begin{align}
  u(x, t)
  & = \frac{1}{2} \left[ \twiddle f(x + ct) + \twiddle f(x - ct) \right] +
  \frac{1}{2c} \int^{x + ct}_{x - ct} \twiddle g(y)\ \text dy\\
  & = \frac{1}{2} \left[ f(x + ct) - f(ct - x) \right] +
  \frac{1}{2c}
  \left[
  \int^{0}_{x - ct} \twiddle g(y)\ \text dy +
  \int^{x + ct}_{0} \twiddle g(y)\ \text dy
   \right]\\
   & = \frac{1}{2} \left[ f(x + ct) - f(ct - x) \right] +
   \frac{1}{2c}
   \left[
   -\int^{0}_{x - ct} g(-y)\ \text dy +
   \int^{x + ct}_{0} g(y)\ \text dy
    \right]
\end{align}

Here, let us perform substitution with $w = -y$,
%
\begin{align}
  & = \frac{1}{2} \left[ f(x + ct) - f(ct - x) \right] +
  \frac{1}{2c}
  \left[
  \int^{0}_{ct - x} g(w)\ \text dw +
  \int^{x + ct}_{0} g(y)\ \text dy
  \right]\\
  & = \frac{1}{2} \left[ f(x + ct) - f(ct - x) \right] +
  \frac{1}{2c}
  \left[
  \int^{x + ct}_{ct - x} g(y)\ \text dy
  \right]
\end{align}

If we look at the domain of dependence, the left line reflect back to our domain and the line is represented as $ct - x$.

\ex $u_{tt} = u_{xx}, x \in [0, \infty)$
%
\begin{align}
  u(x, 0) & =
  \begin{cases}
    1 & 4 < x < 5\\
    0 & \text{otherwise}
  \end{cases}\\
  u_t(x, 0) & = 0
\end{align}
\end{document}
