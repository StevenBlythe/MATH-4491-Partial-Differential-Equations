\documentclass{article}
\usepackage{stocktonmacros} % Personal macros, import package
\setlist[itemize]{noitemsep, topsep=0pt} % Must keep within the same document as \begin{document}

\begin{document}

%\maketitle
%
%\newpage
%
%\tableofcontents
%%
%\newpage
%%
%\import{Lectures/}{Partial Differential Equations - An Introduction}
%\import{Lectures/}{Heat, Wave Equation}
%\import{Lectures/}{Approximating Functions with Other Functions}
%\import{Lectures/}{Heat Equation}
%\import{Lectures/}{Wave Equation}
%\import{Lectures/}{Boundary Conditions}
%\import{Lectures/}{Laplace's Equation}
%\import{Lectures/}{Polar Coordinates}
%\import{Lectures/}{Complex Analysis, Fourier Transform}
%\import{Lectures/}{Transport Equation}
%\import{lectures/}{Image Processing}
%\import{Lectures/}{Conservation Laws}
%______________________________________________________________________________%


\topic{April 1, 2022}
%
\begin{align}
  \xi^\prime(t) & = \frac{f(u_L) - f(u_R)}{u_L - u_R}
\end{align}

For some example $f(u) = \frac{u^2}{2}$, $u_L = 1, u_R = 0$
%
\begin{align}
  \xi^\prime(t) & = \frac{\frac{1}{2} - 0}{1 - 0}\\
  & = \frac{1}{2}
\end{align}

There are other initial conditions that still lead to two solutions.
%
\begin{align}
  u_t + uu_x & = 0\\
  u(x, 0) & =
  \begin{cases}
    0 & x < 0\\
    1 & e \geq 0
  \end{cases}
\end{align}

Riemann Problem. The slope of our characteristic line is $\frac{1}{u}$.

In our solution, we have the verticals on the left side and slope = 1 on the right side, so the solutions do not collide. To remediate this, we add a shock in between and extend both solutions to the shock line.

R-H Jump Condition
%
\begin{align}
  \xi^\prime(t) & = \frac{0 - \frac{1}{2}}{0 - 1}\\
  & = \frac{1}{2}
\end{align}

Another solution is to make a fan (paper fan)

There would be no shock and the solution is continuous. The R-H jump condition is not used.
%
\begin{align}
  u(x, t) & =
  \begin{cases}
    0 & x < 0\\
    \frac{x}{t} & 0 \frac{x}{t} < 1\\
    1 & \frac{x}{t} \geq 1
  \end{cases}
\end{align}

Conservation Law: $u_t + [f(u)]_x = 0$.

If $f$ is smooth: $u_t + f^\prime(u) u_x = 0$.

$f^\prime(u)$ is the speed of the characteristic.

Slope of characteristic $= \frac{1}{f^\prime(u)}$

\note If the solution is continuous, the $R-H$ condition gives the slope of a characteristic, not the slope of shocks.

What is the actual solution to the last problem?
%
\begin{align}
  u_t + u u_x & = \eps u_{xx}
\end{align}

Therefore, if $\eps > 0$ is a smoothing term (as in heat) $\to C^\infty$.

Which solution is the solution that you get if you solve the last equation and let $\eps \to 0$?

This ends up giving us the lax entropy condition:

The characteristic curves can enter a shock as time increases, but they cannot exit (or be created) from a shock.

% Solution one: tree leaf, branch out from center stem
% Solution two: hut
Solution one violates the lax entropy condition, therefore solution two is the correct solution.

\thm There exists a unique solution to any conservation law $u_t + [f(u)]_x = 0$, $u(x, 0) = g(x)$, $x \in (-\infty, \infty), t \in [0, \infty)$ whose shocks satisfy the $R-H$ jump condition and the lax entropy condition.

\note The fan is called a carefaction wave.

For any general first order equation $F(\vv x, u(\vv x), \grad(\vv x)) = 0$

$\vv x = (x, t)$ is our conservation law, you always have characteristic curves. The difficulty lies with how to resolve what happens when characteristic collide.






\end{document}
