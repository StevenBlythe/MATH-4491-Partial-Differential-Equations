\documentclass{article}
\usepackage{stocktonmacros} % Personal macros, import package
\setlist[itemize]{noitemsep, topsep=0pt} % Must keep within the same document as \begin{document}

\begin{document}

%\maketitle
%%
%\newpage
%%
%\tableofcontents
%%%
%\newpage
%%%
%\import{Lectures/}{Partial Differential Equations - An Introduction}
%\import{Lectures/}{Heat, Wave Equation}
%\import{Lectures/}{Approximating Functions with Other Functions}
%\import{Lectures/}{Heat Equation}
%\import{Lectures/}{Wave Equation}
%\import{Lectures/}{Boundary Conditions}
%\import{Lectures/}{Laplace's Equation}
%\import{Lectures/}{Polar Coordinates}
%\import{Lectures/}{Complex Analysis, Fourier Transform}
%\import{Lectures/}{Transport Equation}
%\import{Lectures/}{Image Processing}
%\import{Lectures/}{Exam 2 Review}
\import{Lectures/}{Conservation Laws}
\import{Lectures/}{Wave Equation on Semi-Infinite Domain}
%______________________________________________________________________________%
\newpage

\section{D'Alembert's Formula on a Bounded Domain}
\begin{center}
  $
  u_{tt} = c^2 u_{xx} \quad
  u(x, 0) = f(x) \quad
  u_t(x, 0) = g(x) \quad
  0 \leq x \leq L, t \in [0, \infty)
  $
\end{center}

How do we find the solution to the wave equation on
$(-\infty, \infty)$
to find a solution on $[0, L]$?

We extend the initial conditions to be odd and periodic with period $2L$.
%
\begin{align}
  \twiddle f(x) & =
  \begin{cases}
    f(x) & 0 < x < L\\
    0 & x = 0\\
    -f(-x) & -L < x < 0
  \end{cases}
\end{align}

Recall, we considered boundary conditions. Here, let us define boundary conditions as:
\begin{center}
  $u(0, t) = u(L, t) = 0$
\end{center}

Here, let us enforce $\twiddle f(x + 2L) = \twiddle f(x)$ to force periodicity. For $\twiddle g(x)$, let us write:
%
\begin{align}
  \twiddle g(x) & =
  \begin{cases}
    g(x) & 0 < x \leq L\\
    0 & x = 0\\
    -g(-x) & -L < x < 0
  \end{cases}
\end{align}

The solution will be:
%
\begin{align}
  u(x, t) & = \frac{1}{2}
  \left[ \twiddle f(x + ct) + \twiddle f(x - ct) \right] +
  \frac{1}{2} \int^{x + ct}_{x - ct} \twiddle g(y)\ \text dy\\
  u(0, t) & = \frac{1}{2}
  \left[ \twiddle f(ct) + \twiddle f(-ct) \right]\\
  u(L, t) & = \frac{1}{2}
  \left[ \twiddle f(L + ct) + \twiddle f(L - ct) \right] +
  \frac{1}{2c} \int^{L + ct}_{L - ct} \twiddle g(y)\ \text dy\\
  & = \frac{1}{2}
  \left[ \twiddle f(ct - L) + \twiddle f(L - ct) \right] +
  \frac{1}{2c} \left(
  \int^{0}_{L - ct} \twiddle g(y)\ \text dy +
  \int^{L + ct}_{0} \twiddle g(y)\ \text dy
  \right)\\
  & = \frac{1}{2}
  \left[ \twiddle -f(L - ct) + \twiddle f(L - ct) \right] +
  \frac{1}{2c} \left(
  \int^{0}_{ct - L} \twiddle g(y)\ \text dy +
  \int^{L + ct}_{0} \twiddle g(y)\ \text dy
  \right)\\
  & = \frac{1}{2c}
  \left(
  \int^0_{ct + L} \twiddle g(y)\ \text dy+
  \int_0^{L + ct} \twiddle g(y)\ \text dy
  \right)
\end{align}

We added an integral of length $2L$, which is $0$ since $\twiddle g$ is odd.

\end{document}
