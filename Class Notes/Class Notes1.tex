\documentclass{article}
\usepackage{stocktonmacros, import} % Personal macros, import package
\setlist[itemize]{noitemsep, topsep=0pt} % Must keep within the same document as \begin{document}
\title{Partial Differential Equations - Class Notes} % Customized for \maketitle

\usepackage{xcolor}

\begin{document}
\maketitle
\newpage

%\import{Lectures/}{Legacy.tex}
%\import{Lectures/}{2022-03-02.tex}
%\import{Lectures/}{2022-03-04.tex}
%\import{Lectures/}{2022-03-07.tex}
%\import{Lectures/}{2022-03-09.tex}
\import{lectures/}{Image Processing}
%______________________________________________________________________________%
\topic{Conservation Laws}

Recall how we mentioned heat is conserved, accumulated heat is heat in - heat our.

1-D Conservation:
\begin{itemize}
  \item $u(x, t)$: Quantity that is conserved: energy, mass, momentum, $\ldots$.
\end{itemize}
  %o======|##|======o <- |x0   |x1
  %
  \begin{align}
    g(x, t) & = \text{ flux} = f(u(x, t))
  \end{align}

  Here, flux is dependent on the gradient. Before, we have:
  %
  \begin{align}
    g(x, t) & = g(u_x(x, t))
  \end{align}
  This was our gradient.

  \underline{onservation Law}

  \begin{center}
    Accumulation = in - out
  \end{center}
  \begin{align}
    \int^{x_1}_{x_0} u(x, t_1) \text dx - \int^{x_1}_{x_0} u(x, t_0) \text dx
    & = u(x_0, t)\\
    \int^x_{x_0} [u(x, t_1) - u(x, t_0)] \text dx
    & = \int^{t_1}_{t_0} [ q(x_0, t) - q(x_1, t)] \text dt\\
    \int^{x_1}_{x_0} \int^{t_1}_{t_0} q_t(x, t) \text dt\ \text dx
    & = - \int^{t_1}_{t_0} \int^{x_1}_{x_0} q_x(x, t) \text dx\ \text dt\\
    \int^{t_1}_{t_0} \int^{x_1}_{x_0} [u_t(x, t) + u_x(x, t)] \text dx\ \text dt
    & = -\int^{t_1}_{t_0} \int^{x_1}_{x_0} q_x(x, t) \text dx\ \text dt\\
    \int^{t_1}_{t_0} \int^{x_1}_{x_0} [u_t(x, t) + q_x(x, t)] \text dx\ \text dt & = 0
  \end{align}

  Here, $u_t + q_x = 0$ if $u_t$ and $q_x$ are continuous.

  Since $q = f(u)$, we get
  %
  \begin{align}
    u_t + [f(u)]_x & = 0
  \end{align}

  This is considered the conservation law. This is non-linear, first order.

  \ex Burger's Equation: For gas flow down a pipe.
  %
  \begin{align}
    u_t + \left( \frac{u^2}{2} \right)_x & = \eps u_{xx}
  \end{align}

  Here, $\eps$ is the viscosity and $u$ is the momentum. The viscosity of gases tend towards zero, therefore let us consider $\eps = 0$.
  %
  \begin{align}
    u_t + \left( \frac{u^2}{2} \right)_x & = 0
  \end{align}
  Here, let $f(u) = \frac{u^2}{2}$:
  \begin{align}
    u_t + u u_x & = 0
  \end{align}

  Domain: $x \in (-\infty, \infty), t \in [0, \infty)$
  % If we have any t partials, we need one initial condition.

  Initial condition: $u(x, 0) = g(x)$

  \underline{Recall}: Transport equation:

  $u_t = cu_x \Rightarrow u_t - cu_x = 0$

  $u(x, 0) = g(x)$.

  The slope of the characteristic $ = -\frac{1}{c}$
  % Graph with $-\frac{1}{c} + k, Vk

  The values of the solution is constant along characteristic.

  \subsection*{March 30, 2022}
  Burger's Equation:
  %
  \begin{align}
    u_t + uu_x & = 0
  \end{align}

  The slope is $\frac{1}{u}$.

  Recall, the slope of the characteristic of the equation $u_t - cu_x = 0$ is $-\frac{1}{c}$.

  \ex
  %
  \begin{align}
    u(x, 0) & =
    \begin{cases}
      1     &         x < 0\\
      1 - x & 0 \leq  x < 1\\
      0     & x \geq  1
    \end{cases}
  \end{align}

  % Graph, ////. /^ /^^ /^^^ . ||||||||

  At $t = 1$, the solution is no longer continuous, therefore $u_x$ is not defined at $x = 1, t = 1$.

  For $t \geq 1$, there is no differentiable solution (everywhere).

  % Slope 1 continues until hitting \infty slope

  If we allow solutions that aren't defined only for a single curve $\xi(t)$ of discontinuity, then we have an infinite number of solutions. From physics, we know that gas velocity (momentum) looks like this:

  % From x = t = 1, Slope 2 appears

  The shock in velocity causes a sonic boom
  %
  \begin{align}
    \text{Slope } & = 2\\
    \xi^\prime(t) & = \frac{1}{2}
  \end{align}
  %
  $u_t + [f(u)]_x = 0$ has no classical solution (existence problem) and $\infty$ number of solutions if we don't care about the equation being satisfied on shocks (uniqueness problem)

  We would like:
  %
  \begin{enumerate}
    \item a unique solution to exist
    \item We want this solution to be what we observe
  \end{enumerate}

  Define a weak solution to be $u(x, t)$ where
  %
  \begin{align}
    \int^{t_1}_{t_0} \int^{x_1}_{x_0} [u_t + [f(u)]_x] \text dx \text dt & = 0
  \end{align}

  where $u_t$ and $[f(u)]_x$ are measures (they can contain $\delta$ functions)

  This idea will lead to a formula for shock speed, $\xi^\prime(t)$.

  The integral in $x-$direction is enough.
  %
  \begin{align}
    \int^{x_1}_{x_0} [u_t + [f(u)]_x]\ \text dx|_{t = t_0} & = 0
  \end{align}

  Here,
  %
  \begin{align}
    \int^{x_1}_{x_0} u_t\ \text dx
    & = - \int^{x_1}_{x_0} [f(u)]_x\ \text dx |_{t = t_0}\\
    \frac{d}{dt} \int^{x_1}_{x_0} u\ \text dx |_{t = t_0}
    & = f(u(x_0, t_0) - f(u(x_1, t_0)))\\
    \frac{d}{dt} \left[ \int^{\xi(t)}_{x_0} u(x, t)\ \text dx + \int^{x_1}_{\xi(t)} u(x, t)\ \text dx \right] \big|_{t = t_0}
    & = f(u_L) - f(u_R)
  \end{align}

  Using the second fundamental theorem of calculus,
  %
  \begin{align}
    \int^{\xi(t)}_{x_0} u_t(x, t) \text dx + u(\xi(t), t) \xi^\prime(t) + \int^{x_1}_{\xi(t)} u_t(x, t) \text dx - u(\xi(t), t)\xi^\prime(t)|_{t = t_0} & = f(u_L) - f(u_R)
  \end{align}

  You cannot cancel the second and fourth term, as the second term is on the left of the shock and the fourth term is on the right of the shock.

  Take $\lim$ as $x_0 \to \xi(t)^-$,
  %
  \begin{align}
    \int^{\xi(t)}_{x_0} u_t(x, t) \text dx & = 0
  \end{align}

  Take $\lim$ as $x_1 \to \xi(t)^+$,
  %
  \begin{align}
    \int^{x_1} u_t(x, t) \text dx & = 0
  \end{align}

  We now have:
  %
  \begin{align}
    u_L \xi^\prime(t_0) - u_R \xi^\prime(t_0) & = f(u_L) - f(u_R)
  \end{align}

  $t_0$ is arbitrary.
  %
  \begin{align}
    \xi^\prime(t) & = \frac{f(u_L) - f(u_R)}{u_L - u_R}
  \end{align}

  Here, we have Rankine-Hugniot jump condition.

\end{document}
