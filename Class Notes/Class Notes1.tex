\documentclass{article}
\usepackage{stocktonmacros, import} % Personal macros, import package
\setlist[itemize]{noitemsep, topsep=0pt} % Must keep within the same document as \begin{document}
\title{Partial Differential Equations - Class Notes} % Customized for \maketitle

\usepackage{xcolor}

\begin{document}
\maketitle
\newpage

%\import{Lectures/}{Legacy.tex}
%\import{Lectures/}{2022-03-02.tex}
%\import{Lectures/}{2022-03-04.tex}
%\import{Lectures/}{2022-03-07.tex}
%\import{Lectures/}{2022-03-09.tex}

%______________________________________________________________________________%

\subsection*{Project Information}

If you want to mess with an image, you have to think about how computers view images. The image is stored. When we see it, we see an image, but what does the computer see?

Computers stores images as a point of pixels akin to a grid. Pixels contain RGB value information and location. Pictures are generally large.

The first thing is: An image is a grid of numbers (Array, matrix). This is what the computer sees.

Color: Saturation, hue, not the focus.

Focus: Grayscale, same functions, can be extended into color. Add extra dimensions.

Idea: Have a grid of values with pixels and a value at each point (Tells us the gray scale, point of scale between black and white).

\bigbreak

\topic{Image Processing}

Let's say we have a black and white digital image.

% ^ y
% |
% |  :^)
% |
% |________> x
% |

Here, we have:

\begin{itemize}
  \item $u(x, y)$ : $u$ (int) is the darkness (grayscale) of the image at point $(x, y)$.
  \item $u = 0$ black, $u = 255$ = white
\end{itemize}

Steps to our process:
%
\begin{enumerate}
  \item Remove Noise. In the case of our project, we will remove Salt and Pepper noise.
  \item Remove Blur.
\end{enumerate}

We know Heat, Laplace, Wave, and Transport.

We want to use Heat.

% ^ y
% | .  . .. .
% | . :^) .
% | .   . .
% |________> x
% |

Here, we want to remove the 'spikes' in our image. The idea is if we apply heat to our point, the peak of the spike reduces in magnitude and spreads out.

When we smoothen our picture, we also blur our ideal picture.

\begin{enumerate}
  \item First step: Take an image and blur it.

  The heat equation blurs things. This will cause the salt and pepper noise fade.

  Here, let us consider an initial image with noise. Our initial image has $t = 0$. Then, we apply
  %
  \begin{align}
    u_t & = u_{xx} + u_{yy}\\
    u(x, y, 0) & = f(x, y)
  \end{align}

  Here, $f(x, y)$ is our image, the initial condition.

  However, when we apply our heat function, we have a pro and a con:
  %
  \begin{itemize}
    \item Pro: Salt and pepper noise are pretty much gone.
    \item Con: Whole image is blurred.
  \end{itemize}

  If we just have one point of noise, Let's say we have an $m \times n$ white grid and black dot in the center,
  % |
  % |
  % |   *
  % |
  % |_______>
  % |
  then let us consider the following:
  %
  \begin{align}
    f(x, y) & = \delta\left(x - \frac{L}{2}\right) \delta\left(y - \frac{M}{2}\right)
  \end{align}

  Here, we have the following conditions:
  %
  \begin{enumerate}
    \item $u(x, y, 0) = f(x, y)$
  \end{enumerate}

  and
  %
  \begin{enumerate}
    \item $u(0, y, t) = f(0, y)$
    \item $u(L, y, t) = f(L, y)$
    \item $u(x, 0, t) = f(x, 0)$
    \item $u(x, M, t) = f(x, M)$
  \end{enumerate}
\end{enumerate}
  %


%______________________________________________________________________________%
\end{document}
