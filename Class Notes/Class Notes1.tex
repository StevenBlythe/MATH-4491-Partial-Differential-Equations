\documentclass{article}
\usepackage{stocktonmacros, import} % Personal macros, import package
\setlist[itemize]{noitemsep, topsep=0pt} % Must keep within the same document as \begin{document}
\title{Partial Differential Equations - Class Notes} % Customized for \maketitle

\usepackage{xcolor}

\begin{document}
\maketitle
\newpage

\import{Lectures/}{Legacy.tex}
\import{Lectures/}{2022-03-02.tex}
\import{Lectures/}{2022-03-04.tex}
\import{Lectures/}{2022-03-07.tex}

%______________________________________________________________________________%

\topic{Wave Equation Solutions}
%
\begin{itemize}
  \item $u_t = u_xx$
  \item $x \in (-\infty, \infty)$, $t \in [0, \infty)$
  \item $u(x, 0) = f(x)$
  \item $\displaystyle u(x, t) = \frac{1}{\sqrt{4 \pi t}} \int^\infty_{-\infty} f(y) e^{- \frac{(x - y)^2}{4t}} \text{ dy}$
\end{itemize}

Let the initial condition be a ``delta function,'' $\delta(x)$.

\bigbreak

\emph{What is a delta function, $\delta(x)$?}

It has two main properties:
%
\begin{enumerate}
  \item $\delta(x) = 0$, $x \neq 0$.
  \item $\displaystyle \int^\infty_{-\infty} \delta(x) \text{ dx } = 1$
\end{enumerate}

The ``mass'' is centered at $x = 0$. The delta function is not a function because $\delta(0) = ?$. Actually, the delta function is a measure.

\underline{Calculations with Delta Functions}
%
\begin{align}
  \int^\infty_{-\infty} \delta(y) g(x - y) \text{ dy}
  & = \int^\infty_{-\infty} \delta(x- y) g(y) \text{ dy} = g(x)
\end{align}

Here, $\delta(y)$ is zero except when $y = 0$ and $\delta(x - y) = 0$ except when $x = y$.

Here, we have a convolution $\delta * g$, where our variables can switch.

What do we expect when $f(x) = \delta(x)$?

When $t = 0$, our area is the $t$ axis: $|$, however, as $t \to \infty$, then the area slowly flattens, akin to a candle.

Mathematically, what do we expect?
%
\begin{align}
  u(x, t) & = \frac{1}{\sqrt{4 \pi t}} \int^\infty_{-\infty} \delta(y) e^{- \frac{(x - y)^2}{4t}} \text{ dy}\\
  & = \frac{1}{\sqrt{4 \pi t}} e^{- \frac{x^2}{4 t}}
\end{align}

The $t$'s impact in the fraction reduces the amplitude and the $t$ in the exponent flattens out the curve.

This is the Gaussian Normal Distributions

What if $f(x) = 7 \delta(x) + 5 \delta(x - 3)$?
%
\begin{align}
  u(x, t) & = \frac{1}{\sqrt{4 \pi t}} \int^\infty_{-\infty} [7 \delta(y) + 5 \delta(y - 3)] e^{-\frac{(x - y)^2}{4 t}} \text{ dy}\\
  & = \frac{1}{\sqrt{4 \pi t}}
  \left[
    \int^\infty_{-\infty} 7 \delta(y) e^{- \frac{(x - y)^2}{4 t}} \text{ dy} +
    \int^\infty_{-\infty} 5 \delta(y - 3) e^{-\frac{(x - y)^2}{4t}} \text{ dy}
  \right]\\
  & = \frac{1}{\sqrt{4 \pi t}}
  \left[
    7e^{- \frac{x^2}{4 t}} + 5 e^{-\frac{(x - 3)^2}{4 t}}
  \right]
\end{align}

So for a general $f(x)$, think of $f(x)$ as a bunch of delta functions.

\topic{Conservation of Energy}

The amount of heat stamp? constant
%
\begin{align}
  \frac{d}{dt} \int^\infty_{-\infty} u(x, t) \text{ dx}
  & = \int^\infty_{-\infty} u(x, t) \text{ dx}\\
  & = \alpha^2 \int^\infty_{-\infty} u_{xx} (x, t) \text{ dx}\\
  & = \alpha^2 u_x(x, t) \Big|^\infty_{-\infty}
\end{align}

\underline{Recall} We know $\lim_{x \pm \infty} u(x, t) = 0$, therefore the rate of change at both infinities is zero.
%
\begin{align}
  \alpha^2 u_x(x, t) \Big|^\infty_{-\infty} & = 0
\end{align}
%______________________________________________________________________________%
\end{document}
