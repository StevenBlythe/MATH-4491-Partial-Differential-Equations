\documentclass{article}
\usepackage{stocktonmacros, import} % Personal macros, import package
\setlist[itemize]{noitemsep, topsep=0pt} % Must keep within the same document as \begin{document}
\title{Partial Differential Equations - Class Notes} % Customized for \maketitle

\usepackage{xcolor}

\begin{document}
\maketitle
\newpage

%\import{Lectures/}{Legacy.tex}
%\import{Lectures/}{2022-03-02.tex}
%\import{Lectures/}{2022-03-04.tex}
%\import{Lectures/}{2022-03-07.tex}
%\import{Lectures/}{2022-03-09.tex}
%\import{lectures/}{Image Processing}
%______________________________________________________________________________%
\topic{Conservation Laws}

Recall how we mentioned heat is conserved, accumulated heat is heat in - heat our.

1-D Conservation:
\begin{itemize}
  \item $u(x, t)$: Quantity that is conserved: energy, mass, momentum, $\ldots$.
\end{itemize}
  %o======|##|======o <- |x0   |x1
  %
  \begin{align}
    g(x, t) & = \text{ flux} = f(u(x, t))
  \end{align}

  Here, flux is dependent on the gradient. Before, we have:
  %
  \begin{align}
    g(x, t) & = g(u_x(x, t))
  \end{align}
  This was our gradient.

  \underline{onservation Law}

  \begin{center}
    Accumulation = in - out
  \end{center}
  \begin{align}
    \int^{x_1}_{x_0} u(x, t_1) \text dx - \int^{x_1}_{x_0} u(x, t_0) \text dx
    & = u(x_0, t)\\
    \int^x_{x_0} [u(x, t_1) - u(x, t_0)] \text dx
    & = \int^{t_1}_{t_0} [ q(x_0, t) - q(x_1, t)] \text dt\\
    \int^{x_1}_{x_0} \int^{t_1}_{t_0} q_t(x, t) \text dt\ \text dx
    & = - \int^{t_1}_{t_0} \int^{x_1}_{x_0} q_x(x, t) \text dx\ \text dt\\
    \int^{t_1}_{t_0} \int^{x_1}_{x_0} [u_t(x, t) + u_x(x, t)] \text dx\ \text dt
    & = -\int^{t_1}_{t_0} \int^{x_1}_{x_0} q_x(x, t) \text dx\ \text dt\\
    \int^{t_1}_{t_0} \int^{x_1}_{x_0} [u_t(x, t) + q_x(x, t)] \text dx\ \text dt & = 0
  \end{align}

  Here, $u_t + q_x = 0$ if $u_t$ and $q_x$ are continuous.

  Since $q = f(u)$, we get
  %
  \begin{align}
    u_t + [f(u)]_x & = 0
  \end{align}

  This is considered the conservation law. This is non-linear, first order.

  \ex Burger's Equation: For gas flow down a pipe.
  %
  \begin{align}
    u_t + \left( \frac{u^2}{2} \right)_x & = \eps u_{xx}
  \end{align}

  Here, $\eps$ is the viscosity and $u$ is the momentum. The viscosity of gases tend towards zero, therefore let us consider $\eps = 0$.
  %
  \begin{align}
    u_t + \left( \frac{u^2}{2} \right)_x & = 0
  \end{align}
  Here, let $f(u) = \frac{u^2}{2}$:
  \begin{align}
    u_t + u u_x & = 0
  \end{align}

  Domain: $x \in (-\infty, \infty), t \in [0, \infty)$
  % If we have any t partials, we need one initial condition.

  Initial condition: $u(x, 0) = g(x)$

  \underline{Recall}: Transport equation:

  $u_t = cu_x \Rightarrow u_t - cu_x = 0$

  $u(x, 0) = g(x)$.

  The slope of the characteristic $ = -\frac{1}{c}$
  % Graph with $-\frac{1}{c} + k, Vk

  The values of the solution is constant along characteristic.
\end{document}
