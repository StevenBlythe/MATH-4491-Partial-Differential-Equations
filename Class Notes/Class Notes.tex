\documentclass{article}
\usepackage[legalpaper, margin=2cm]{geometry}
\usepackage{amsmath,amssymb,amsthm}		% If AMS-LaTeX is used, this comes first
\usepackage{color}

\usepackage{enumitem}
\usepackage{graphicx}
\setlist[itemize]{noitemsep, topsep=0pt}
\newcommand{\topic}[1]{\textbf{\underline{#1}}}
\newcommand{\R}{\mathbb R}
\newcommand{\Z}{\mathbb Z}
\newcommand{\N}{\mathbb N}
\newcommand{\p}{\partial }
\newcommand{\grad}{\nabla}
\newcommand{\Ex}{\underline{Example:} }
\newcommand{\ex}{\underline{Ex:} }
\newcommand{\dfn}{\underline{Definition:} }
\newcommand{\thm}{\underline{Theorem:} }
\graphicspath{ {./images/} }
\title{Partial Differential Equations - Class Notes}
\date{\today}
\author{Steven Blythe}

\begin{document}
\maketitle
\newpage
\section{Chapter 1}
\subsection*{Sidenotes}
\subsection{January 19, 2022}
\topic{What is a PDE?}\\
A PDE is an equation which contains partial derivatives of an unknown function and we want to find that unknown function.\\
\Ex $F(t, x, y, z, u, \frac{\partial u}{\partial t}, \frac{\partial u}{\partial x}, \frac{\partial u}{\partial y}, \frac{\partial u}{\partial z}, \frac{\partial^2 u}{\partial t^2}, \frac{\partial^2 u}{\partial x \partial y}, \ldots) = 0$.\\
Note, the first partial derivatives are considered \underline{$1^{st}$ ordered partials}
whereas the second ordered partials are considered \underline{$2^{nd}$ ordered partials}.\\
The variables that are not $u$ are considered independent variables and $u$ is considered a dependent variable.\\
What PDEs do we study?\\
Generally, we restrict our attention to equations that model some phenomenom from physics, engineering, economics, geology, \ldots\ etc. We can use physical intuition to help guide the math.\\
\topic{Classification of PDEs}
\begin{enumerate}
  \item Order of PDE: Highest derivative.\\
  \Ex $\frac{\partial^3 u}{\partial x^3} - \sin(y) u^7 = 3$ is a third order PDE.\\
  \Ex $(\frac{\partial y}{\partial t})^5 - \frac{\partial^2y}{\partial x \partial t} = e^x$ is a second order PDE.
  \item Number of independent variables.\\
  \Ex $\frac{du}{dt} = \frac{\partial^2 u}{\partial x^2}$ has two independent variables: $t, x$.\\
  This is the $1-D$ heat equation.\\
  \Ex $\frac{\partial u}{\partial t} = \frac{\partial^2 u}{\partial x^2} + \frac{\partial^2 u}{\partial y^2} + \frac{\partial^2 u}{\partial z^2} = \Delta u$ has 4 independent variables.\\
  This is the $3-D$ heat equation. $\Delta u$ is Laplacian of $u$.\\
  \underline{Notation}\\
  $
  \Delta u =
  \grad^2 u =
  \grad \cdot \grad u =
  (\frac{\partial}{\partial x}, \frac{\partial}{\partial y}, \frac{\partial}{\partial z}) \cdot (\frac{\partial u}{\partial x}, \frac{\partial u}{\partial y}, \frac{\partial u}{\partial z}) =
  \frac{\partial^2 u}{\partial x^2} + \frac{\partial^2 u}{\partial y^2} + \frac{\partial^2 u}{\partial z^2}
  $\\
  $\Delta u = 0$ is considered Laplace's equation.
  \item Linear vs non-linear\\
  A linear PDE is any equation of the form $L[u(x)] = f(x)$ where $f(x)$ is a known function is a linear partial differential operator.
\end{enumerate}
\dfn A differential operator is any rule that takes a function as its input and returns an expression that involves the derivatives of that function.\\
\Ex
\begin{align}
  u(x, t) & \qquad v(x, t)\\
  O[u] & = \frac{\partial^2 u}{\partial x^2} + \sin x + \pi - 7e^{tu}\\
  O[u + 3v] & = \frac{\partial^2}{\partial x^2}(u + 3v) + \sin x + \pi - 7e^{tu + 3tv}\\
  & = \frac{\partial^2 u}{\partial x^2} + 3 \frac{\partial^2 v}{\partial x^2}+ \sin x + \pi - 7e^{tu + 3tv}
\end{align}
\dfn A linear operator, L, is an operator that has the property:
\begin{align}
  L[au + bv] & = aL[u] + bL[v]
\end{align}
Where $a$ and $b$ are constants.\\
\thm If $u$ and $v$ are vectors and $L$ is linear, then $L$ can be represented by a matrix.\\
\thm If $L$ is linear ordinary operator, it must take the form:
\begin{align}
  L[u] = f_0(x)u + f_1(x)u^\prime + f_2(x)u^{\prime\prime} + \ldots + f_n(x)u^{(n)}
\end{align}
Where the $f_i$'s are known functions.\\
\dfn A linear ODE is any ODE of the form where $f(x)$ is known is the following:
\begin{align}
  L[u] = f(x)
\end{align}
If $f(x) = 0$, then the equation is homogeneous. Otherwise, the equation is non-homogeneous.\\
\ex $(u^\prime)^2 = 0 \Rightarrow u^\prime = 0 \rightarrow$ linear, homogeneous.\\
\thm If $L$ is a linear partial differential operator, it must take the form ($x$ is a vector with $n$ unknowns)
\begin{align}
  L[u(x)] = f_0(x)u + \sum^n_{i = 1} f_i(x) \frac{\partial u}{\partial x_i} +
  \sum^n_{i = 1} \sum^n_{j = 1} f_{ij}(x) \frac{\p^2 u}{\p x_i \p x_j} + \ldots
\end{align}
\dfn A linear PDE is any PDE of the form
\begin{align}
  L[u(x)] = f(x)
\end{align}
If $f(x) = 0$, the equation is homogeneous, else it is non-homogeneous.\\
\ex $u_t = 4u_x$ - Linear, homogeneous.
\end{document}
