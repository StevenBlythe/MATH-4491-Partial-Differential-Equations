\documentclass{article}
\usepackage{stocktonmacros}
\usepackage{import}
\begin{document}

\begin{enumerate}
  \import{Responses/}{5-1.tex}
  %%__________________________________________________________________w________%
  \newpage
  \setcounter{equation}{0}
  %%__________________________________________________________________________%
  \import{Responses/}{5-2.tex}
  %%__________________________________________________________________________%
  \newpage
  \setcounter{equation}{0}
  %%__________________________________________________________________________%
  \import{Responses/}{5-3.tex}
  %%__________________________________________________________________________%
  \newpage
  \setcounter{equation}{0}
  %%__________________________________________________________________________%
  \import{Responses/}{5-3.tex}
  % y = x + t, z = x-t.
  % mixed partial v_yz = 0
  % if Means that vy had no z's, vz had no y's.
  %v is a function of y + z,
  % v = A(y) + B(z)
  %
  % if u satisfies heat heation, so does v.
  % v_tt = u_xx
  %
  % partial with respect to y and z.
  % Find
  %   v
  % x   t
  %y  z  y   z

  % vy = \pv/\x \x/\y + \v/\t \t/\y
  % add the following for an equation
  %
  % y = x + t
  % z = x - t
  % y + z = 2x

  % Then to find t, subtract instead
  % 2t = y - z

  % We need them to find \x/\y = 1/2 and \pt/\py = 1/2
  % vy = 1/2 vx + 1/2 v_T
  % v_yz

  % go from v_y to v_z
  % follow the tree
  %
  %      vy
  %  x      t
  % y z    y z
  %
  % v_yz = \p v_y \p_x
  % vyz = \pvy/\x \pyx/\pz + \pvy/\pt \pt/\pz
  % = 1/2 v_xx + 1/2 v_tx

  % Now, we still have x = 1/2y + z and t = 1/2 y - z
  % = 1/2v_xx + 1/2v_tx \cdot 1/2 + 1/2v_xt + 1/2v_tt \cdot -1/2
  % This cancels out a bit
  % 1/2v_xx - 1/2_vtt
  % Recall from initial, v_tt = u_xx
  % Now, we can say v = A(y) + B(z)

  % We have two initial conditions.
  % Recall u(x, 0) = f(X)
  % u_t(x, 0) = g(x)
  % V = A(y) + B(z)
  % u(x, t) = v(x + t, x - t)

  % Plug in for u(x, 0)
  % v = A(y) + B(z) = A(x + t) + B(x - t)
  % A(x) + B(x) = f(X)

  % u_t says we need a t-partial
  % u_t = v_t
  % Use A(x + t) + B(x - t)
  % A\prime (only has one variable in reality)
  % u_t = v_t = A^\prime(x + t)\cdot 1 - B^\prime(x - t) \cdot 1
  % u_t(x, 0) A^\prime(x) - B^\prime(X) = g(x)

  % Let us rewrite
  % u  (x, 0) = f(x) -> A(x) + B(x) = f(X)
  % u_t(x, 0) = f(x) -> a^\prime(x) - B^\prime(x) = g(X)
  %
  % Integrate the second line

  % How we dealt with the wave equation
  % h(x) = \int g(X) dx
  % h(x) = \int^x_{-\infty} g(y) dy
  % g will head towards 0 at -infty

  % Integrating A\prime and B\pyime is simply
  % A(x) - B(x) = \int^x_{-\infty} g(y) dy
  % Add A(x) + B(x) = f(x) and A(x) - B(x) = integral

  % 2A(x) = f(x) + \int^x_-\infty g(y) dy
  % 2B(x) f(x) - integral

  % WE want to plug this into our v
  % Out A(x + t) + B(x - t) has x+t and x-t, swap these into our new functions

  % v = 1/2(f(x + t) + int^{x + t}_{-\infty} g(y)) dy + 1/2(f(x - t) - int^x - t_-\infty g(Y) dy)
  % SWap the limits of integration for the right side and combine



\end{enumerate}
\end{document}
