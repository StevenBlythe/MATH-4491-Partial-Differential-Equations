\item Derive d'Alembert's formula for $u_{tt} = u_{xx}$ by assuming that $u(x, t) = v(x + t, x - t) = v(y, z)$. Next show that the wave equation yields $v_{yz} = 0$ and hence $v = A(y) + B(z)$ and solve for $A$ and $B$ using the initial conditions $u(x, 0) = f(x)$ and $u_t(x, 0) = g(x)$.
\bigbreak
%_____________________________________________________________________________%

Here, let us take a look at our assumption:
%
\begin{align}
  u(x, t) & = v(x + t, x - t) = v(y, z)
\end{align}

Here, we have the following relationship:
%
% y = x + t, z = x-t.
\begin{align}
  \begin{cases}
    y & = x + t\\
    z & = x - t
  \end{cases}
\end{align}

% mixed partial v_yz = 0
Next, we want to show that the wave equation yields $v_{yz} = 0$. In orderfor
% if Means that vy had no z's, vz had no y's.
the equation to be zero, that means $v_y$ does not contain any $z's$, so when
you differentiate $v_y$ once more, then the non-z terms zero out. The same can
be argued for $v_z$, where $v_z$ does not contain any $y's$ and will zero out.

%v is a function of y + z,
Therefore, $v$ is a function of $y + z$ and we can write $v$ as
% v = A(y) + B(z)
$v = A(y) + B(z)$.

%__________________________________________________________________________%
% if u satisfies heat heation, so does v.
% v_tt = u_xx
%__________________________________________________________________________%

% partial with respect to y and z.
Here, let us find the partial of $v$ with respect to $y$ and $z$.

% Find
%   v
% x   t
%y  z  y   z
Let us consider the following tree:

\begin{center}
  \import{Snippets/}{tree_v}
\end{center}

Here, let us consider our tree and find $v_y$:
%
\begin{align}
  % vy = \pv/\x \x/\y + \v/\t \t/\y
  v_y & = \frac{\p v}{\p x} \frac{\p x}{\p y} + \frac{\p v}{\p t} \frac{\p t}{\p y}
\end{align}

Earlier, we found the following system of equations:
% y = x + t
% z = x - t
\begin{align}
  \begin{cases}
    y & = x + t\\
    z & = x - t
  \end{cases}
\end{align}

% add the following for an equation
Here, notice we can isolate $x$ or $t$ by adding or subtracting
the two equations together. First, let us add the equations to
obtain the following:
%
\begin{align}
  % y + z = 2x
  y + z & = 2x\\
  x & = \frac{1}{2}(y + z)
\end{align}

Using this information, we can find $\frac{\p x}{\p y}$:
%
\begin{align}
  \frac{\p x}{\p y} & = \frac{1}{2}
\end{align}

% Then to find t, subtract instead
Now, to find $t$, we subtract the equations:
%
\begin{align}
  % 2t = y - z
  y - z & = 2t\\
  t & = \frac{1}{2}(y - z)
\end{align}

Similarly, let us find $\frac{\p t}{\p y}$
%
\begin{align}
  \frac{\p t}{\p y} & = \frac{1}{2}
\end{align}

% We need them to find \x/\y = 1/2 and \pt/\py = 1/2
Here, now that we know $\frac{\p x}{\p y}$ and $\frac{\p t}{\p y}$, let us substitute this into line 3:
%
\begin{align}
  % vy = 1/2 vx + 1/2 v_T
  v_y & = \frac{1}{2} v_x + \frac{1}{2} v_t
\end{align}

% v_yz
Now, let us find $v_{yz}$. Let us rewrite our tree:
% go from v_y to v_z
% follow the tree
\begin{center}
  \import{Snippets/}{tree_vy}
\end{center}

Here, let us find $v_{yz}$ using our tree.
%
\begin{align}
  % v_yz = \p v_y \p_x
  % vyz = \pvy/\x \pyx/\pz + \pvy/\pt \pt/\pz
  v_{yz} & = \frac{\p v_y}{\p x} \frac{\p x}{\p y} + \frac{\p v_y}{\p t} \frac{\p t}{\p y}
\end{align}
% = 1/2 v_xx + 1/2 v_tx

% Now, we still have x = 1/2y + z and t = 1/2 y - z
% = 1/2v_xx + 1/2v_tx \cdot 1/2 + 1/2v_xt + 1/2v_tt \cdot -1/2
% This cancels out a bit
% 1/2v_xx - 1/2_vtt
% Recall from initial, v_tt = u_xx
% Now, we can say v = A(y) + B(z)

% We have two initial conditions.
% Recall u(x, 0) = f(X)
% u_t(x, 0) = g(x)
% V = A(y) + B(z)
% u(x, t) = v(x + t, x - t)

% Plug in for u(x, 0)
% v = A(y) + B(z) = A(x + t) + B(x - t)
% A(x) + B(x) = f(X)

% u_t says we need a t-partial
% u_t = v_t
% Use A(x + t) + B(x - t)
% A\prime (only has one variable in reality)
% u_t = v_t = A^\prime(x + t)\cdot 1 - B^\prime(x - t) \cdot 1
% u_t(x, 0) A^\prime(x) - B^\prime(X) = g(x)

% Let us rewrite
% u  (x, 0) = f(x) -> A(x) + B(x) = f(X)
% u_t(x, 0) = f(x) -> a^\prime(x) - B^\prime(x) = g(X)
%
% Integrate the second line

% How we dealt with the wave equation
% h(x) = \int g(X) dx
% h(x) = \int^x_{-\infty} g(y) dy
% g will head towards 0 at -infty

% Integrating A\prime and B\pyime is simply
% A(x) - B(x) = \int^x_{-\infty} g(y) dy
% Add A(x) + B(x) = f(x) and A(x) - B(x) = integral

% 2A(x) = f(x) + \int^x_-\infty g(y) dy
% 2B(x) f(x) - integral

% WE want to plug this into our v
% Out A(x + t) + B(x - t) has x+t and x-t, swap these into our new functions

% v = 1/2(f(x + t) + int^{x + t}_{-\infty} g(y)) dy + 1/2(f(x - t) - int^x - t_-\infty g(Y) dy)
% SWap the limits of integration for the right side and combine
