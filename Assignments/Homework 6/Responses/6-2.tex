\item Use Fourier Transforms to determine the solution to $u_t = cu_x$ is $u(x, t) = f(x + ct)$ where $u(x, 0) = f(x)$.

% f(x + ct) u_t = cu_x u(x, 0) = f(x)
Here, let us consider the given equation:
%
\begin{align}
  u_t & = cu_x
\end{align}

% Transform equation,
Here, let us use Fourier Transform to transform the equation:
%
% Transform middle Equations
% F[u_t] = F[cu_x]
% We pick up ix\xi at each derivative
\begin{align}
  F[u_t] & = F[cu_x]
\end{align}

Here, we are deriving the right side once. Since we're doing one derivative, we pick up one $c i \xi$.
%
% \hat u_t = c i\xi \hat u
\begin{align}
  \hat u_t & = c i \xi \hat u
\end{align}

% Solve ODE,
% Transform back

% x partials on the fourier becomes multiplication, get an ODE



Here, let us consider the following equation:
%
\begin{align}
  u(x, 0) & = f(x)\\
  \hat u(\xi, 0) & = \hat f(\xi)
\end{align}

% \hat u(\xi, 0) = \hat f(\xi)
% \hat u(\xi, t) =

Now, let us consider our equation $u_t = cu_x$. Here, recall we are picking up $c i \xi$ on the right. Let us attempt to write a general form for our equation:
%
\begin{align}
  % \hat u_t = ci\xi \hat u
  \hat u_t & = c i \xi \hat u\\
  % \hat u(\xi, t) = A(\xi) ^ci\xi \hat u -- REFER TO HEAT EQUATION
  \hat u(\xi, t) & = A(\xi)e^{c i \xi}
  % \hat(\xi, 0) = \hat f(\xi) = A(\xi)
\end{align}

Here, we want to find a way to solve for our general solution. If we consider what we wrote for $hat u(\xi, 0)$, we want to find:
%
\begin{align}
  \hat(\xi, 0) & = \hat f(\xi) = A(\xi)
\end{align}

So, our exponential must disappear at $t = 0$, which means we have the following:
%
\begin{align}
  \hat u(\xi, t) & = A(\xi) e^{i c \xi t}
\end{align}
% This tells me

Here, let us rewrite $A(\xi)$ as $f(\xi)$
%
\begin{align}
  \hat u(\xi, t) & = f(\xi) e^{i c \xi t}
\end{align}
% \hat u(\xi, t) = A(\xi) e^i c \xi t = \hat f(\xi)e^ic\xit

% Transform back
Now, let us take our equation and retransform back.
%
\begin{align}
  % u(x, t) = \frac{1}{\sqrt 2 \pi} \int^\infty_{-\infty \hat u(\xi, t) e^ix\xi d\xi = 1/\sqrt{2 \pi}} \int^\infty_{-\infty} \hat f(\xi) e^{i c \xi t} e^{i x \xi} d \xi
  u(x, t) & = \frac{1}{\sqrt{2 \pi}} \int^\infty_{-\infty} \hat u(\xi, t) e^{i x \xi} \text{ d}\xi\\
  & = \frac{1}{\sqrt{2 \pi}} \int^\infty_{-\infty} \hat f(\xi) e^{i c \xi t} e^{i x \xi} \text{ d}\xi
\end{align}

% Combine exponents
Here, let us combine our exponents.
%
\begin{align}
  % 1/\sqrt{2 \pi} \int^\infty_-\infty f(\xi) e^{i \xi(x + ct)} d \xi
  % = f(x + ct)
  u(x, t) & = \frac{1}{\sqrt{2 \pi}} \int^\infty_{-\infty} f(\xi) e^{i \xi(x + ct)} \text{ d}\xi
\end{align}

Here, we found that $u(x + t)$ is $f(x + ct) = \frac{1}{\sqrt{2 \pi}} \int^\infty_{-\infty} f(\xi) e^{i \xi(x + ct)} \text{ d}\xi$
