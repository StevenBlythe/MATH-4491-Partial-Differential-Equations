\documentclass{article}
\usepackage{enumitem}
\usepackage{amsmath,amssymb,amsthm}
\usepackage{esvect}
\usepackage[a4paper, margin=1cm]{geometry}
\usepackage{stocktonmacros}
\usepackage{graphicx}
\graphicspath{ {./images/} }

\begin{document}
\begin{enumerate}
  \item Let $f(x)$ be a $2 \pi$-period function on the interval $[-\pi, \pi]$ where
  $\displaystyle
  f(x) =
  \begin{cases}
    -1 & - \pi < x \leq 0\\
    1 &\ \ 0 < x \leq \pi
  \end{cases}
  $
  % Personal addition - add the Fourier Series Formula
  \begin{align}
    a_n & = \frac{1}{L}  \int^L_{-L} f(x) \sin\left( \frac{n \pi x}{L} \right) \dx\\
    b_n & = \frac{1}{L}  \int^L_{-L} f(x) \cos\left( \frac{n \pi x}{L} \right) \dx\\
    b_0 & = \frac{1}{2L} \int^L_{-L} f(x) \dx
  \end{align}
  \begin{enumerate}
%______________________________________________________________________________%
%
% ans
    \item Plot the function on the interval $[-3\pi, 3\pi]$

    %\begin{center}
    %  \includegraphics{}
    %\end{center}
    \item Plot its (infinite) Fourier series on $[-3 \pi, 3 \pi]$

    %\begin{center}
    %  \includegraphics{}
    %\end{center}
    \item Find the Fourier series of $f(x)$

    Here, let us consider the symmetry of our function.

    When we look at the graph of $f(x)$, we can see there is a reflection about the origin, making the function odd. $\sin$ is also an odd function, therefore $a_n$ is an even function.

    Looking at $b_n$, $\cos$ is an even function, therefore $b_n$ becomes an odd function.

    Finally, $b_0$ is always an odd function. When we integrate these three coefficients, we lose $b_n$ and $b_0$, but keep $a_n$. Since $a_n$ is even, we can write:
%
    \begin{align}
      a_n & = \frac{2}{L} \int^L_0 f(x) \sin\left( \frac{n \pi x}{L} \right) \dx
    \end{align}

    Here, we are looking at the interval from $0$ to $L$. Our given function, $f(x)$ runs from $-\pi$ to $\pi$, therefore our integral is:
%
    \begin{align}
      a_n & = \frac{2}{\pi} \int^\pi_0 1 \cdot \sin\left( \frac{n \pi x}{\pi} \right) \dx\\
      & = \frac{2}{\pi} \int^\pi_0 \sin\left( n x \right) \dx
    \end{align}

    From here, we can compute our integral:
    %
    \begin{align}
      a_n & = \frac{2}{\pi } \int^\pi_0 \sin\left( n x \right) \dx\\
      & = -\frac{2}{\pi n} \cos\left( n x \right) \bigg|^\pi_0\\
      & = \frac{2}{\pi n}\big( 1 - \cos( n \pi ) \big)
    \end{align}

    Here, we found our coefficient, $a_n$. Now, since $f(x)$ is odd, we are only interested in the following:
    %
    \begin{align}
      f(x) & = \sum^\infty_{n = 1} a_n \sin\left( \frac{n \pi x}{L} \right)\\
      & = \sum^\infty_{n = 1}
      \frac{2}{\pi n}\big( 1 - \cos( n \pi ) \big)
      \sin\left( \frac{n \pi x}{L} \right)
    \end{align}

    Here, since our interval is $-\pi$ to $\pi$, so let us write:
    %
    \begin{align}
      f(x) & =
      \sum^\infty_{n = 1}
      \frac{2}{\pi n}\big( 1 - \cos( n \pi ) \big)
      \sin\left( \frac{n \pi x}{\pi} \right)\\
      & =
      \sum^\infty_{n = 1}
      \frac{2}{\pi n}\big( 1 - \cos( n \pi ) \big)
      \sin\left( n x \right)
    \end{align}

    Here, we have our Fourier series.
%
%______________________________________________________________________________%
%
\newpage
%
%______________________________________________________________________________%
%
\end{enumerate}
  \item Let $f(x) = x^2$ be a $2 \pi$-periodic function on the interval $[-\pi, \pi]$.
  \begin{enumerate}
    \item Derive its Fourier series
    %

    % ans
    Let us consider the symmetry of our function. Our function, $f(x)$, is an even function. Therefore, we have the following coefficients:
%
    \begin{align}
      b_n & = \frac{1}{L}  \int^L_{-L} f(x) \cos\left( \frac{n \pi x}{L} \right) \dx\\
      b_0 & = \frac{1}{2L} \int^L_{-L} f(x) \dx
    \end{align}

    Since $f(x)$ is even, we can write:
    %
    \begin{align}
      b_n & = \frac{2}{L}  \int^L_0 f(x) \cos\left( \frac{n \pi x}{L} \right) \dx\\
      b_0 & = \frac{1}{L} \int^L_0 f(x) \dx
    \end{align}

    In addition, since we also know our interval and our function, we can write:
    %
    \begin{align}
      b_n & = \frac{2}{\pi} \int^\pi_0 x^2 \cos\left( n x \right) \dx\\
      b_0 & = \frac{1}{\pi} \int^\pi_0 x^2 \dx
    \end{align}

    First, let us find the integral of $b_n$. Let us rewrite $b_n$ first:
    %
    \begin{align}
      b_n & = \frac{2}{\pi} \int^\pi_0 x^2 \cos\left( n x \right) \dx
    \end{align}

    Here, we want to do integration by parts. We want $x^2$ as our derived function since we can derive that function to $0$.

    \begin{center}
      \begin{tabular}{c|c}
        $x^2$ & $\displaystyle \cos(n x)$\\
        \hline
        $2x$ & $\displaystyle \frac{1}{n} \sin(n x)$\\
        \hline
        $2$ & $\displaystyle -\frac{1}{n^2} \cos(n x)$\\
        \hline
        $0$ & $\displaystyle -\frac{1}{n^3} \sin(n x)$
      \end{tabular}
    \end{center}

    Here, we can write our integral as the following:
    %
    \begin{align}
      b_n & = \frac{2}{\pi} \left[ \frac{x^2}{n} \sin(n x) + \frac{2x}{n^2} \cos(n x) - \frac{2}{n^3} \sin(n x) \right]^\pi_0\\
      & = \frac{2}{\pi n} \left[ x^2 \sin(n x) + \frac{2x}{n} \cos(n x) - \frac{2}{n^2} \sin(n x) \right]^\pi_0\\
      & = \frac{2}{\pi n} \left[ \pi^2 \sin(\pi x) + \frac{2\pi}{n} \cos(n \pi) - \frac{2}{n^2} \sin(n \pi) \right] - \frac{2}{n \pi} \left[ 0^2 \sin(0) + \frac{0}{n} \cos(0) - \frac{2}{n^2} \sin(0) \right]
    \end{align}

    Here, the entire right term zeroes out. On the left, $\sin(n \pi)$ zeroes out, leaving us with:
    %
    \begin{align}
      b_n & = \frac{4}{n^2} \cos(n \pi)
    \end{align}

    Now, let us find $b_0$:
    %
    \begin{align}
      b_0 & = \frac{1}{\pi} \int^\pi_0 x^2 \dx\\
      & = \frac{1}{\pi} \left[ \frac{x^3}{3} \right]^\pi_0\\
      & = \frac{1}{3\pi} \left[ x^3 \right]^\pi_0\\
      & = \frac{1}{3\pi} \left[ \pi^3 - 0 \right]\\
      & = \frac{\pi^2}{3}
    \end{align}

    Now that we have our coefficients, we can write:
    %
    \begin{align}
      f(x) & = \frac{\pi^2}{3} + \sum^\infty_{n = 1} \frac{4}{n^2} \cos(n x) \sin(n x)
    \end{align}
    %
    \item Use Maple of Matlab to plot its finite Fourier series on $[-\pi, \pi]$ for $N = 10, 20, 50$ together with $f(x)$
    %

    % ans

    %
    \item Use your Fourier series from part (a) to show that $\frac{\pi^2}{6} = 1 + \frac{1}{2^2} + \frac{1}{3^2} + \frac{1}{4^2} + \ldots$.
    %

    % ans

    %
  \end{enumerate}
%
%______________________________________________________________________________%
%
\newpage
%
%______________________________________________________________________________%
%
  \item In the solution of the heat equation, we end up solving $X^{\prime\prime} = -\lambda X$. Show that if $\lambda < 0$ or $\lambda = 0$ there is only the trivial solution ($X(x) = 0$).

Here, we have the equation:
\begin{align}
  X^{\prime\prime} & = - \lambda X
\end{align}

We want to use this equation and set our boundary conditions as $X(0) = X(L) = 0$. Now, we must find an equation where after two derivatives on the right, we obtain a similar function on the left. On the left, we have a sign, coefficient, and function of $x$. Let us write a general solution for our equation:
%
\begin{align}
  X(x) & = A \cos(\sqrt \lambda x) + B \sin( \sqrt \lambda x)
\end{align}

Here, we can make three assumptions \ask{via trichotomy}: $\lambda < 0, \lambda = 0,$ or $\lambda > 0$. Let us look at the first two examples:

\begin{enumerate}
  \item $\lambda < 0$

  Here, let us consider the case when $\lambda$ is negative. Let us consider rewriting $\lambda$:
  %
  \begin{align}
    \lambda < 0\\
    \lambda \cdot -1 > 0 \cdot -1\\
    -1 \cdot \lambda > 0
  \end{align}

  Now, let us plug in our found value into our general equation:
  %
  \begin{align}
    X(x) & =
    A \cos (\sqrt{- 1 \cdot \lambda} x) +
    B \sin (\sqrt{- 1 \cdot \lambda} x)
  \end{align}

  Let us separate the terms under the radical:
%
  \begin{align}
    X(x) & =
    A \cos (\sqrt{- 1 \cdot \lambda} x) +
    B \sin (\sqrt{- 1 \cdot \lambda} x)\\
    & =
    A \cos (\sqrt{- 1} \sqrt{\lambda} x) +
    B \sin (\sqrt{- 1} \sqrt{\lambda} x)\\
    & =
    A \cos (i \sqrt{\lambda} x) +
    B \sin (i \sqrt{\lambda} x)
  \end{align}

  Here, in our expression, we see we are taking the square root of a negative number, which would give us an imaginary number. Here, we are evaluating our general solution with real numbers, therefore, the following form:
  %
  \begin{align}
    X(x) & =
    A \cos (i \sqrt{\lambda} x) +
    B \sin (i \sqrt{\lambda} x)
  \end{align}

  Where $X(x)$ is a real number would only have the trivial solution $X(x) = 0$.

  \item $\lambda = 0$

  Here, let us consider the case when $\lambda$ is zero. Now, let us write our general equation:
  %
  \begin{align}
    X(x) & = A \cos(\sqrt \lambda x) + B \sin(\sqrt \lambda x)
  \end{align}

  Here, since $\lambda = 0$, we can evaluate our equation:
  %
  \begin{align}
    X(x) & = A \cos(0) + B \sin(0)\\
    & = A
  \end{align}

  Now, let us evaluate our boundary condition for $X(x) = A$. First, we let $X(0) = 0$:
  %
  \begin{align}
    X(0) & = 0 = A
  \end{align}

  Here, we know $A$ is $0$. For the second condition, let us write:
  %
  \begin{align}
    X(L) & = 0 = A
  \end{align}

  Here, we will always have the trivial solution, $X(x) = 0$.

\end{enumerate}
%
%______________________________________________________________________________%
%
\newpage
%
%______________________________________________________________________________%
%
  \item Show that $u(x, t) = e^{-\lambda^2 a^2 t}\left[ A \cos(\lambda x) + B \sin(\lambda x) \right]$
%______________________________________________________________________________%
%

% ans


%
%______________________________________________________________________________%
%
\newpage
%
%______________________________________________________________________________%
%
  \item Solve $u_t = u_{xx}$ given $u(0, t) = u(1, t) = 0$ for $t \geq 0$ and $u(x, 0) = 1$ for $0 \leq x \leq 1$
%
%______________________________________________________________________________%
%

% ans
Let us consider the following conditions:
%
\begin{itemize}
  \item $u_t = u_{xx}$
  \item $u(0, t) = 0, t \geq 0$
  \item $u(1, t) = 0, t \geq 0$
  \item $u(x, 0) = 1, 0 \leq x \leq 1$
\end{itemize}

Let us begin finding our solution.
\begin{enumerate}
  \item Let us assume our solution is seperable. Therefore, we can write $u(x, t) = X(x)T(x)$. Now, using our initial conditions, let us write:
  %
  \begin{align}
    u(0, t) & = X(0)T(t) = 0 \Rightarrow X(0) = 0\\
    u(1, t) & = X(1)T(t) = 0 \Rightarrow X(1) = 0\\
    u(x, 0) & = X(x)T(0) = 0 \Rightarrow T(0) = 0\\
  \end{align}
\end{enumerate}
%
%______________________________________________________________________________%
%
\item Find the solution to the previous problem if $u(x, 0) = x - x^2$ for $0 \leq x \leq 1$

% ans
\begin{itemize}
  \item $u_t = u_{xx}$
  \item $u(0, t) = 0, t \geq 0$
  \item $u(1, t) = 0, t \geq 0$
  \item $u(x, 0) = x - x^2, 0 \leq x \leq 1$
\end{itemize}

%
%______________________________________________________________________________%
%
\newpage
%
%______________________________________________________________________________%
%
  \item Solve $u_t = u_{xx}$ given $u(0, t) = u(1, t) = 0$ for $t \geq 0$ and $u(x, 0) = 10^{-5} \sin(10^6 \pi x)$ for $0 \leq x \leq 1$. Determine $u(x, 2)$ and $u(x, -2)$ and look at their magnitudes. Note that when $t = -2$, we are looking at the backward heat equation and given the magnitude of $u(x, -2)$, what can you say about the solution to the backward heat equation?

  % ans
  Let us consider the following conditions:
  \begin{itemize}
    \item $u_t = u_{xx}$
    \item $u(0, t) = 0, t \geq 0$
    \item $u(1, t) = 0, t \geq 0$
    \item $u(x, 0) = 10^{-5} \sin(10^6 \pi x), 0 \leq x \leq 1$
    \item Determine the following and look at their magnitudes
    \begin{itemize}
      \item $u(x, 2)$
      \item $u(x, -2)$
    \end{itemize}
  \end{itemize}
\end{enumerate}
\end{document}
