\documentclass{article}
\usepackage{enumitem}
\usepackage{amsmath,amssymb,amsthm}
\usepackage{esvect}
\usepackage[a4paper, margin=1cm]{geometry}
\usepackage{stocktonmacros}
\usepackage{graphicx}
\graphicspath{ {./images/} }

\begin{document}
%
%______________________________________________________________________________%
%
\begin{enumerate}
  \item Solve $\Delta u = 0$ on $x \in [0, 4]$, $y \in [0, 3]$, with $u(x, 0) = u(x, 3) = 0$ where

  \begin{enumerate}
    \item $u_x(0, y) = 0$ and $u_x(4, y) = \cos(\pi y)$

    \item $u(0, y) = 1$ and $u(4, y) = 0$ (Hint: Translate the $x$ coordinate so that $u(0, y) = 0$. This means that $x$ goes between $-4$ and $0$. Now go back to $0$ to $4$ with a $y$-axis flip.)
  \end{enumerate}
\bigbreak
%_____________________________________________________________________________%
  Here, let us first solve part a) of the problem.
  %
  \begin{enumerate}
    \item First, let us assume our equation is separable:
    %
    \begin{align}
      u(x, y) & = X(x)Y(y)
    \end{align}

    Let us take note of our $u$:
    %
    \begin{align}
      u_{xx} + u_{yy} & = 0\\
      X^{\prime\prime}Y + XY^{\prime\prime} & = 0\\
      XY^{\prime\prime} & = -X^{\prime\prime}Y\\
      \frac{Y^{\prime\prime}}{Y} & = - \frac{X^{\prime\prime}}{X} = -\lambda
    \end{align}

    Here, let us consider our boundary conditions:
    %
    \begin{align}
      u(x, 0) = 0 & = X(0)Y(y)
    \end{align}

    Here, we either have $X(0) = 0$ or $Y(y) = 0$. We do not want to assume $Y(y) = 0$ because that would make $u(x, y) = 0 \ \forall y \in [0, 3]$. Therefore, we consider $X(0) = 0$. We will follow this assumption for the upcoming boundary conditions.
    %
    \begin{align}
      u(x, y) & = X(x)Y(y)\\
      u(x, 0) & = X(x)Y(0) = 0 \Rightarrow Y(0) = 0\\
      u(x, 3) & = X(x)Y(3) = 0 \Rightarrow Y(3) = 0
    \end{align}

    Now, let us consider $u_x(x, y)$ as $X^\prime(x)Y(y)$:
    %
    \begin{align}
      u_x(x, y) & = X^\prime(0) Y(y)\\
      u_x(0, y) & = X^\prime(0) Y(y) = 0
      \Rightarrow X^\prime(0) = 0
    \end{align}

    Now, let us write:
    %
    \begin{align}
      - \frac{X^{\prime\prime}}{X} & = \frac{Y^{\prime\prime}}{Y} = - \lambda
    \end{align}
    %
    \item Next we solve for $Y(y)$ since we have more information on $Y(y)$.
    %
    \begin{align}
      \frac{Y^{\prime\prime}}{Y} & = -\lambda\\
      Y^{\prime\prime} & = - \lambda Y
    \end{align}

    Here, let us write out our general equation:
    %
    \begin{align}
      Y_n(y) & = A \sin ( \sqrt \lambda x) + B \cos ( \sqrt \lambda x )
    \end{align}

    Here, we know $Y(0) = Y(3) = 0$. Let us input our $y$:
    %
    \begin{align}
      Y_n(0) = 0 & = B\\
      Y_n(3) = 0 & = A \sin( \sqrt \lambda 3 )
    \end{align}

    Here, let us not consider $A$ as $0$, instead, the inside of our sine function to be $0$.
    %
    \begin{align}
      \sqrt \lambda 3 & = n \pi\\
      \sqrt \lambda & = \frac{n \pi}{3}\\
      \lambda_n & = \left( \frac{n \pi}{3} \right)^2
    \end{align}

    Now we have the following:
    %
    \begin{align}
      Y_n(y) & = \sin\left( \frac{n \pi y}{3} \right )
    \end{align}

    \item Now, let us solve for $X$:
    %
    \begin{align}
      \frac{X^{\prime\prime}}{X} & = \lambda\\
      X^{\prime\prime} & = \lambda X
    \end{align}

    Here, we solved for $\lambda$ in the previous step:
    %
    \begin{align}
      X^{\prime\prime}_n & =
      \left(
      \frac{n \pi}{3}
      \right)^2
      X_n
      \\
      X_n(x) & =
      C
      \sinh \left(
      \frac{n \pi x}{3}
      \right) +
      D \cosh \left(
      \frac{n \pi x}{3}
      \right)
    \end{align}

    Here, let us find the derivative for $X$ and find $X^\prime(0)$:
    %
    \begin{align}
      X^\prime(0) & = \frac{n \pi x}{3} C = 0\\
      & = C = 0
    \end{align}

    Therefore, since $C = 0$, we get:
    %
    \begin{align}
      X_n(x) & = D \cosh\left( \frac{n \pi x}{3} \right)
    \end{align}
    \item Now, let us combine $u_n$ and $u$:
    %
    \begin{align}
      u_n(x, y) & =
      D \cosh\left( \frac{n \pi x}{3} \right) \sin\left( \frac{n \pi y}{3} \right)
    \end{align}

    By linearity, let us write:
    %
    \begin{align}
      u(x, y) & =
      % Summation
      \sum^\infty_{n = 1}
      % Coefficient D
      D_n
      % Cosh
      \cosh\left( \frac{n \pi x}{3} \right)
      % Sin
      \sin\left( \frac{n \pi y}{3} \right)
    \end{align}

    \item Here, let us use our boundary conditions to find our coefficients.
    %
    \begin{align}
      u_x(4, y) & = \cos( \pi y)
    \end{align}

    Here, let us derive our $u(x, y)$ to get $u_x(x, y)$
    %
    \begin{align}
      u_x(x, y) & =
      \sum^\infty_{n = 1} D_n
      \left( \frac{n \pi}{3} \right)
      \sinh\left( \frac{n \pi x}{3} \right)
      \sin\left( \frac{n \pi y}{3} \right)\\
      %_______________________________________________________________________%
      u_x(4, y) & =
      \sum^\infty_{n = 1} D_n
      \left( \frac{n \pi}{3} \right)
      \sinh\left( \frac{n \pi 4}{3} \right)
      \sin\left( \frac{n \pi y}{3} \right)
    \end{align}

    Here, let us consider our coefficient in the equation:
    %
    \begin{align}
      D_n \left(\frac{n \pi}{3}\right) \sinh\left(\frac{4 \pi n}{3}\right)
      & = \frac{2}{3} \int^3_0 \cos(\pi y) \sin\left(\frac{n \pi y}{3}\right) dy\\
      D_n & =
      \frac{2}{3}
      \frac{3}{n \pi}
      \frac{1}{\sinh\left(\frac{4 \pi n}{3}\right)}
      \int^3_0 \cos(\pi y) \sin\left(\frac{n \pi y}{3}\right)\\
      & =
      \frac{2}{n \pi \sinh\left(\frac{4 \pi n}{3}\right)}
      \int^3_0 \cos(\pi y) \sin\left(\frac{n \pi y}{3}\right)
    \end{align}

    Here, let us isolate our integral and solve for that. Afterwards, let us combine our function.
    %
    \begin{align}
      \int^3_0 \cos(\pi y) \sin\left(\frac{n \pi y}{3}\right)
    \end{align}

    %Here, we have b_n in between B_n and sin.
    % B_n(n pi/3) \sinh(4 pi n/3) = 2/3 \int^3_0 \cos(\pi y) \sin(n \pi y/3) dy
    % ...
    % B_n =
    % n cos(\pi n) + n / \pi n^2 - 9\pi
    % /
    % (n \pi/3) \sinh(4 n \pi/3)
    %
    % Final:
    % u(x, y) = C + \sum^\infty_{n = 1}
    % n^2 / 3(n^2 - 9)
    % [(-1)^n + 1]
    % \sinh(4 n \pi3)
    % \cosh(n \pi x/3)
    % \sin(n \pi y/3)



    % B: u(0, y) = 1 and u(4, y) = 0.
    % y_n(y) = \sin(n pi y/3)

    %% X(x) =C \sinh(\sqrt \lambda x) + D \cosh(\sqrt \lambda x), X(0) = X(4) = 0
    % X(x) = C \sinh(\sqrt \lambda (4 - x))

    %% u(x, y) = \sinh(n pi (4 - x) / 3) \sin(4 \pi y / 3)
    % Linearity
    % u(x, y) = \sum... A_n \sinh( n \pi(4 - x) / 3) \sin(n \pi y/3)
    % u(0, y) = \sum... A_n \sinh(4 n \pi / 3) \sin(n \pi y / 3) = 1
    % Here, take the first A_n \sinh:
    % A_n = \sinh(4 n pi / 3) = 2/3 \int^3_0 \sin(n pi y / 3) dy
    % A_n =
    % 2(1 - cos(\pi n))
    % /
    %n \pi \sinh(4 n \pi/3)
    %
    % Final
    % u(x, y) = \sum..
    % (2 (1 - cos(pi n))/(n pi sinh(4 n pi/3)))
    % sinh(n pi(4 -x)/3)
    % sin(n pi y/3)
  \end{enumerate}

\newpage
  %__________________________________________________________________________%
%
%______________________________________________________________________________%
%

%
%______________________________________________________________________________%
%
  \item Solve $u_{tt} = 4u_{xx}$ on $x \in [0, 3]$, $t \in [0, \infty)$, with $u(0, t) = u(3, t) = 0$ where

  \begin{enumerate}
    \item $u(x, 0) = 4 \sin(2 \pi x) + 7 \sin(6 \pi x) - 2\sin(\pi x)$, $u_t(x, 0) = 0$
    \item $u(x, 0) = x(3 - x)$, $u_t(x, 0) = \sin(\pi x)$
  \end{enumerate}

  Let us begin with our boundary conditions.
  %
  \begin{enumerate}
    \item Let us assume our equation is separable.
    %
    \begin{align}
      u_{tt} & = 2^2 u_{xx}\\
      XT^{\prime\prime} & = 2^2 X^{\prime\prime}T\\
      \frac{T^{\prime\prime}}{2^2 T} & = \frac{X^{\prime\prime}}{X} = - \lambda
    \end{align}

    \item Here, since we have more information regarding $X$, let us solve for $X$:
    %
    \begin{align}
      \frac{X^{\prime\prime}}{X} & = - \lambda\\
      X^{\prime\prime} & = - \lambda X
    \end{align}

    In this equation, we want to consider $\cos$ and $\sin$ for our general equation.
    %
    \begin{align}
      X(x) & = A \sin(\sqrt \lambda x) + B \cos (\sqrt \lambda x)
    \end{align}

    Here, we know $X(0) = X(3) = 0$, so let us write:
    %
    \begin{align}
      X(0) & = B = 0\\
      X(3) & = A \sin(\sqrt \lambda 3) = 0
    \end{align}

     Here, let us find when our sine function is $0$.
     %
     \begin{align}
       \sqrt \lambda 3 & = n \pi\\
       \sqrt \lambda & = \frac{n \pi}{3}\\
       \lambda & = \left( \frac{n \pi}{3} \right)^2
     \end{align}

     Now, let us write our general equation for $X$:
     %
     \begin{align}
       X_n(x) & = \sin \left( \frac{n \pi x}{3} \right)
     \end{align}
     %
     \item Next, let us solve for $T$:
     %
     \begin{align}
       \frac{T^{\prime\prime}}{2^2T} & = -\lambda\\
       T^{\prime\prime} & = - 2^2 T \lambda
     \end{align}

     Here, let us write the general equation:
     %
     \begin{align}
       T_n(t) & = C_n \sin\left( \frac{2 n \pi t}{L} \right) + D_n \cos \left( \frac{2 n \pi t}{L} \right)
     \end{align}

     % DIVERGING THEORIES, PATH 1: KEEP L
     \item Here, let us combine to find $u_n$ and $u$:
     %
     \begin{align}
       % Normal
       u_n(x, t) & =
       \sin\left( \frac{n \pi x}{3} \right)
       \left[
       C_n \sin \left( \frac{2 n \pi t}{L} \right) +
       D_n \cos \left( \frac{2 n \pi t}{L} \right)
       \right]\\
       % Series
       u(x, t) & =
       % Summation
       \sum^\infty_{n = 1}
       % First coefficient
       C_n
       \sin \left( \frac{n \pi x}{3} \right)
       \sin \left( \frac{2 n \pi t}{L} \right) +
       % Second coefficient
       D_n
       \sin \left( \frac{n \pi x}{3} \right)
       \cos \left( \frac{2 n \pi t}{L} \right)
     \end{align}

     \item Here, let us find the coefficients using the Initial Conditions:
     %
     \begin{itemize}
       \item
       $u(x, 0) =
       4 \sin ( 2 \pi x ) +
       7 \sin ( 6 \pi x ) -
       2 \sin( \pi x )$
       \item $u_t(x, 0) = 0$
     \end{itemize}

     Using this information, find our $t$ partial from our summation series:
     %
     \begin{align}
       u_t(x, t) & =
       % Summation
       \sum^\infty_{n = 1}
       % First coefficient
       C_n
       \sin \left( \frac{n \pi x}{3} \right)
       \frac{2 n \pi}{L}
       \cos \left( \frac{2 n \pi t}{L} \right) -
       % Second coefficient
       D_n
       \sin \left( \frac{n \pi x}{3} \right)
       \frac{2 n \pi}{L}
       \sin \left( \frac{2 n \pi t}{L} \right)
     \end{align}

     % From here, we do matching terms:
     % D_6 = 4
     % D_18 = 7
     % D_3 = -2
     % D_n = 0

     % u_t (x, 0) = 0
     % \sum Cn Sin(n pi x / 3) Cn pi/3 = 0
     % u(x, t) = 4\sin(2 pi x) cos(4 pi t) + 7 sin(6 pi x) cos(12 pi t) - 2sin(pi x) cos(2 pi t)

     % u(x, 0) = x(3 - x)
     % D_n = 2/3 \int^3_0 x(3-x)/(3x-x^2) sin(n pi x/3) dx
     %...
     % 27 n pi sin(n pi) + 54 cos(n pi) - 54/n^3 pi^3

     From here, let us plug in our initial condition, $u_t(x, 0) = 0$:
     %
     \begin{align}
       u_t(x, 0) & =
       \sum^\infty_{n = 1}
       C_n
       \frac{2 n \pi}{L} \sin \left( \frac{n \pi x}{3} \right) = 0
     \end{align}

%     % DIVERGING THEORIES, PATH 2: L = \infty %
%     Since our length spans over $\infty$, we can write:
%     %
%     \begin{align}
%       T_n(t) & = C_n \sin(0) + D_n \cos(0)\\
%       T_n(t) & = D_n
%     \end{align}
%    \item Here, let us combine to find $u_n$ and $u$:
%    %
%    \begin{align}
%      % Normal
%      u_n(x, t) & =
%      D_n
%      \sin\left( \frac{n \pi x}{3} \right)
%      \\
%      % Series
%      u(x, t) & =
%      % Summation
%      \sum^\infty_{n = 1}
%      D_n
%      \sin\left( \frac{n \pi x}{3} \right)
%    \end{align}

  \end{enumerate}

%
%______________________________________________________________________________%
%

\end{enumerate}
\end{document}
