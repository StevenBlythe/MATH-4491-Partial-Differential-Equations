\documentclass{article}
\usepackage{enumitem}
\usepackage{amsmath,amssymb,amsthm}
\usepackage{esvect}
\usepackage[a4paper, margin=1cm]{geometry}
\usepackage{stocktonmacros}
\usepackage{graphicx}
\graphicspath{ {./images/} }

\begin{document}
\begin{enumerate}
  \item Let $f(x)$ be a $2 \pi$-period function on the interval $[-\pi, \pi]$ where
  $\displaystyle
  f(x) =
  \begin{cases}
    -1 & - \pi < x \leq 0\\
    1 &\ \ 0 < x \leq \pi
  \end{cases}
  $
  % Personal addition - add the Fourier Series Formula
  \begin{align}
    f(x) & = a_n \sin\left( \frac{n \pi x}{L} \right) + b_n \cos\left( \frac{n \pi x}{L} \right)
  \end{align}
  \begin{enumerate}
%______________________________________________________________________________%
%
% ans
    \item Plot the function on the interval $[-3\pi, 3\pi]$
    %\begin{center}
    %  \includegraphics{}
    %\end{center}
    \item Plot its (infinite) Fourier series on $[-3 \pi, 3 \pi]$
    %\begin{center}
    %  \includegraphics{}
    %\end{center}
    \item Find the Fourier series of $f(x)$
    Here, let us consider a few points:
    \begin{itemize}
      \item $L = -\pi$
    \end{itemize}

%
%______________________________________________________________________________%
\newpage
\end{enumerate}
  \item Let $f(x) = x^2$ be a $2 \pi$-periodic function on the interval $[-\pi, \pi]$.
  \begin{enumerate}
    \item Derive its Fourier series
    \item Use Maple of Matlab to plot its finite Fourier series on $[-\pi, \pi]$ for $N = 10, 20, 50$ together with $f(x)$
    \item Use your Fourier series from part (a) to show that $\frac{\pi^2}{6} = 1 + \frac{1}{2^2} + \frac{1}{3^2} + \frac{1}{4^2} + \ldots$.
  \end{enumerate}
%______________________________________________________________________________%
%
% ans
Let us begin
%
%______________________________________________________________________________%
  \item In the solution of the heat equation, we end up solving $X^{\prime\prime} = -\lambda X$. Show that if $\lambda < 0$ or $\lambda = 0$ there is only the trivial solution ($X(x) = 0$).

%______________________________________________________________________________%
%
% ans
Here, we have the equation:
\begin{align}
  X^{\prime\prime} & = - \lambda X
\end{align}

We want to use this equation and set our boundary conditions as $X(0) = X(L) = 0$. Now, we must find an equation where after two derivatives on the right, we obtain a similar function on the left. On the left, we have a sign, coefficient, and function of $x$. Let us write a general solution for our equation:
%
\begin{align}
  X(x) & = A \cos(\sqrt \lambda x) + B \sin( \sqrt \lambda x)
\end{align}

Here, we can make three assumptions \ask{via trichotomy}: $\lambda < 0, \lambda = 0,$ or $\lambda > 0$. Let us look at the first two examples:

\begin{enumerate}
  \item $\lambda < 0$

  Here, let us consider the case when $\lambda$ is negative. Let us consider rewriting $\lambda$:
  %
  \begin{align}
    \lambda < 0\\
    \lambda \cdot -1 > 0 \cdot -1\\
    -1 \cdot \lambda > 0
  \end{align}

  Now, let us plug in our found value into our general equation:
  %
  \begin{align}
    X(x) & =
    A \cos (\sqrt{- 1 \cdot \lambda} x) +
    B \sin (\sqrt{- 1 \cdot \lambda} x)
  \end{align}

  Let us separate the terms under the radical:
%
  \begin{align}
    X(x) & =
    A \cos (\sqrt{- 1 \cdot \lambda} x) +
    B \sin (\sqrt{- 1 \cdot \lambda} x)\\
    & =
    A \cos (\sqrt{- 1} \sqrt{\lambda} x) +
    B \sin (\sqrt{- 1} \sqrt{\lambda} x)\\
    & =
    A \cos (i \sqrt{\lambda} x) +
    B \sin (i \sqrt{\lambda} x)
  \end{align}

  Here, in our expression, we see we are taking the square root of a negative number, which would give us an imaginary number. Here, we are evaluating our general solution with real numbers, therefore, the following form:
  %
  \begin{align}
    X(x) & =
    A \cos (i \sqrt{\lambda} x) +
    B \sin (i \sqrt{\lambda} x)
  \end{align}

  Where $X(x)$ is a real number would only have the trivial solution $X(x) = 0$.

  \item $\lambda = 0$

  Here, let us consider the case when $\lambda$ is zero. Now, let us write our general equation:
  %
  \begin{align}
    X(x) & = A \cos(\sqrt \lambda x) + B \sin(\sqrt \lambda x)
  \end{align}

  Here, since $\lambda = 0$, we can evaluate our equation:
  %
  \begin{align}
    X(x) & = A \cos(0) + B \sin(0)\\
    & = A
  \end{align}

  Now, let us evaluate our boundary condition for $X(x) = A$. First, we let $X(0) = 0$:
  %
  \begin{align}
    X(0) & = 0 = A
  \end{align}

  Here, we know $A$ is $0$. For the second condition, let us write:
  %
  \begin{align}
    X(L) & = 0 = A
  \end{align}

  Here, we will always have the trivial solution, $X(x) = 0$.

\end{enumerate}
%
%______________________________________________________________________________%

  \item Show that $u(x, t) = e^{-\lambda^2 a^2 t}\left[ A \cos(\lambda x) + B \sin(\lambda x) \right]$
%______________________________________________________________________________%
%

% ans


%
%______________________________________________________________________________%
%
\newpage
%
%______________________________________________________________________________%
%
  \item Solve $u_t = u_{xx}$ given $u(0, t) = u(1, t) = 0$ for $t \geq 0$ and $u(x, 0) = 1$ for $0 \leq x \leq 1$
%
%______________________________________________________________________________%
%

% ans


%
%______________________________________________________________________________%
%
  \item Find the solution to the previous problem if $u(x, 0) = x - x^2$ for $0 \leq x \leq 1$
%
%______________________________________________________________________________%
%

%
%______________________________________________________________________________%
%
  \item Solve $u_t = u_{xx}$ given $u(0, t) = u(1, t) = 0$ for $t \geq 0$ and $u(x, 0) = 10^{-5} \sin(10^6 \pi x)$ for $0 \leq x \leq 1$. Determine $u(x, 2)$ and $u(x, -2)$ and look at their magnitudes. Note that when $t = -2$, we are looking at the backward heat equation and given the magnitude of $u(x, -2)$, what can you say about the solution to the backward heat equation?

  % ans
  Let us consider the following conditions:
  \begin{itemize}
    \item $u(0, t) = 0, t \geq 0$
    \item $u(1, t) = 0, t \geq 0$
    \item $u(x, 0) = 10^{-5} \sin(10^6 \pi x), 0 \leq x \leq 1$
    \item Determine the following and look at their magnitudes
    \begin{itemize}
      \item $u(x, 2)$
      \item $u(x, -2)$
    \end{itemize}
  \end{itemize}
\end{enumerate}
\end{document}
