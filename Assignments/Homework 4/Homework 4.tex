\documentclass{article}
\usepackage{enumitem}
\usepackage{amsmath,amssymb,amsthm}
\usepackage{esvect}
\usepackage[a4paper, margin=1cm]{geometry}
\usepackage{stocktonmacros}
\usepackage{graphicx}
\graphicspath{ {./images/} }
\usepackage{multicol}
\usepackage{setspace}
\usepackage[parfill]{parskip}
\setstretch{1.25}
\begin{document}
\begin{enumerate}
  \item Solve $\Delta u = 0$ on $x \in [0, 2]$, $y \in [0, 5]$, with $u_x(0, y) = \cos(3 \pi y)$, $u_x(2, y) = 0$, $u_y(x, 0) = \sin(\pi x)$, $u_y(x, 5) = 0$ and $u(0, 0) = 3$.

  % Begin Answer
  Here, let us write out our given conditions (Neumann):
  \begin{enumerate}
    \item $x \in [0, 2], y \in [0, 5]$
    \item $\Delta u = 0 \Rightarrow u_{xx} + u_{yy} = 0$
    \item $u_x(0, y) = \cos(3 \pi y)$
    \item $u_x(2, y) = 0$
    \item $u_y(x, 0) = \sin(\pi x)$
    \item $u_y(x, 5) = 0$
    \item $u\ (0, 0) = 3$
  \end{enumerate}

  Now, let us begin solving for our equation.
  %
  \begin{enumerate}
    % First Step
    \item First, let us assume our equation is separable.
    %
    \begin{align}
      u_{xx} + u_{yy} & = 0\\
      u_{xx} & = -u_{yy}\\
      X^{\prime\prime}Y & = -XY^{\prime\prime}\\
      \frac{X^{\prime\prime}}{X} & = -\frac{Y^{\prime\prime}}{Y} = -\lambda
    \end{align}

    Let us consider $u_1$ and $u_2$, where $u_{1x}(0, y) = 0$ and $u_{2y}(x, 0) = 0$.
    % Second Step - Pick X
    \item Here, let us solve for $X_1$:
    %
    \begin{align}
      \frac{X^{\prime\prime}}{X} & = -\lambda\\
      X^{\prime\prime} & = -\lambda X
    \end{align}

    Here, the general equation for this form is sine and cosine:
    %
    \begin{align}
      X(x) & = A \cos(\sqrt \lambda x) + B \sin(\sqrt \lambda x)
    \end{align}

    Now, since we have information on $u_{1x}$, let us find the first derivative:
    %
    \begin{align}
      X^\prime(x) & = -A \sqrt \lambda \sin(\sqrt \lambda x) + B \sqrt \lambda \cos(\sqrt \lambda x)
    \end{align}

    Now, let us solve for $u_{1x}(0, y)$
    %
    \begin{align}
      X^\prime(0) & = -A \sqrt \lambda \sin(0) + B \sqrt \lambda \cos(0)\\
      X^\prime(x) & = A \sqrt \lambda = 0\\
      & = B = 0
    \end{align}

    Now, we have:
    %
    \begin{align}
      X^\prime(x) & = A \sqrt \lambda \sin(\sqrt \lambda x)\\
      X^\prime(2) & = A \sqrt \lambda \sin(\sqrt \lambda 2) = 0\\
      & = \sqrt \lambda 2 = n \pi\\
      & = \sqrt \lambda = \frac{n \pi}{2}\\
      & = \lambda_{1n} = \left(\frac{n \pi}{2}\right)^2
    \end{align}

    Here, we have:
    %
    \begin{align}
      X_1(x) & = A \cos\left(\frac{n \pi x}{2}\right)
    \end{align}

  \item Now, let us solve for $Y_1$:
  %
  \begin{align}
    \frac{Y^{\prime\prime}}{Y} & = \lambda
  \end{align}

  Here, we must use sinh and cosh and shift our variable:
  %
  \begin{align}
    Y_1(y) & = C \cosh(\sqrt \lambda (5 - y) + D \sinh(\sqrt \lambda (5 - y))
  \end{align}

  Now, let us take the first derivative:
  %
  \begin{align}
    Y\prime_{1y}(y) & = C \sqrt \lambda \sinh (\sqrt \lambda (5 - y)) - D \sqrt \lambda \cosh(\sqrt \lambda (5 - y))
  \end{align}

  Here, let $y = 5$,
  %
  \begin{align}
    Y\prime_{1y}(5) & = -D \sqrt \lambda = 0\\
    & = D = 0
  \end{align}

  Now, let us write again:
  %
  \begin{align}
    Y_{1}(y) & = C \sinh(\sqrt \lambda (5 - y))
  \end{align}

  Then let us input our $\lambda$:
  %
  \begin{align}
    Y_{1n}(y) & = C \cosh\left(\frac{n \pi (5 - y)}{2}\right)
  \end{align}

  \item Now, if we combine our functions, we can write:
  %
  \begin{align}
    u_{1n}(x, y) & =
    \cos\left(\frac{n \pi x}{2}\right)
    \cosh\left(\frac{n \pi(5 - y)}{2}\right)
  \end{align}

  By linearity, let us write:
  %
  \begin{align}
    u_{1}(x, y) & = \sum^\infty_{n = 1}
    \cos\left(\frac{n \pi x}{2}\right)
    \cosh\left(\frac{n \pi(5 - y)}{2}\right)
  \end{align}

  \setcounter{enumii}{1}
  \item Let us go back and solve for $u_{2}(x, y)$, starting with Y. Consider our separable equation:
  %
  \begin{align}
    \frac{X^{\prime\prime}}{X} & = - \frac{Y^{\prime\prime}}{Y}\\
    \frac{Y^{\prime\prime}}{Y} & = - \frac{X^{\prime\prime}}{X} = -\lambda\\
  \end{align}

  Here, we can perform the same series of steps to solve for $Y_2$ as we solved for $X_1$, swapping our $L$ from $2$ to $5$ in our new case.
  %
  \begin{align}
    \lambda_{2n} & = \left(\frac{n \pi}{5}\right)^2\\
    Y_2(x) & = A \cos\left(\frac{n \pi x}{5}\right)
  \end{align}
  \item Similarly, let us write the solution for $X_{2}$ as we solved for $Y_{1}$:
  %
  \begin{align}
    X_{2n}(x) & = C \cosh\left(\frac{n \pi(2 - y)}{5}\right)
  \end{align}
  \item Again, let us combine $u_2$ and $u_{2n}$:
  %
  \begin{align}
    u_{2n}(x, y) & = \cos\left(\frac{n \pi x}{5}\right) \cosh\left(\frac{n \pi(2 - y)}{5}\right)
  \end{align}

  By linearity,
  %
  \begin{align}
    u_{2}(x, y) & = \sum^\infty_{n = 1} \cos\left(\frac{n \pi x}{5}\right) \cosh\left(\frac{n \pi(2 - y)}{5}\right)
  \end{align}

  \item Here, let us combine our $u_1$ and $u_2$:
  %
  \begin{align}
    u(x, y) & = \sum^\infty_{n = 1}
    \cos\left(\frac{n \pi x}{2}\right)\cosh\left(\frac{n \pi(5 - y)}{2}\right) +
    \cos\left(\frac{n \pi x}{5}\right)\cosh\left(\frac{n \pi(2 - y)}{5}\right)
  \end{align}
\end{enumerate}
%______________________________________________________________________________%
\newpage
\setcounter{equation}{0}
%______________________________________________________________________________%
  \item Solve $u_t = 9u_{xx}$ on $x \in [0, 2]$ if $u(0, t) = 4$, $u(2, t) = 8$ and $u(x, 0) = 3 \sin(5 \pi x) - 11 \sin(9 \pi x) + 2x + 4$

  % Begin Answer
  Here, let us write out our given conditions:
  \begin{enumerate}
    \item $x \in [0, 2]$
    \item $u_t = 9u_xx$
    \item $u(0, t) = 4$
    \item $u(2, t) = 8$
    \item $u(x, 0) = 3\sin(5 \pi x) - 11\sin(9 \pi x) + 2x + 4$
  \end{enumerate}

  Let us consider general boundaries, as $u$ does not start and end at $0$. We have $T_1 = 4$ and $T_2 = 8$.

  Now, our line can be described as the following:
  %
  \begin{align}
    \frac{8 - 4}{2}x + 4\\
    2x + 4
  \end{align}

  Here, let us solve for $u(x, t) = w(x, t) + u(x, \infty)$. To begin, let us consider our steady state condition as well:
  %
  \begin{align}
    u(0, t) = 4 & \Rightarrow w(0, t) = u(0, t) - u(0, \infty) = 4 - 4 = 0\\
    u(2, t) = 8 & \Rightarrow w(2, t) = u(2, t) - u(2, \infty) = 8 - 8 = 0
  \end{align}

  Now, let us plug in our $x$ into our steady-state solution and get the next two solutions:
  %
  \begin{enumerate}
  \item Assume $w(x, t) = X(x)T(t)$
  %
  \begin{align}
    XT^\prime & = 9X^{\prime\prime}T\\
    \frac{T^\prime}{9T} & = \frac{X^{\prime\prime}}{X} = -\lambda
  \end{align}

  \item Here, let us solve for $X$:
  %
  \begin{align}
    \frac{X^{\prime\prime}}{X} & = -\lambda\\
    X^{\prime\prime} & = -\lambda X
  \end{align}

  Here, we want to use the general cosine and sine form:
  %
  \begin{align}
    X(x) & = A \sin \left( \sqrt \lambda x \right) + B \cos \left( \sqrt \lambda x \right)
  \end{align}

  Here, let us write our conditions:
  %
  \begin{align}
    X(0) & = B = 0\\
    X(x) & = A \sin \left( \sqrt \lambda x \right)\\
    X(2) & = A \sin \left( \sqrt \lambda 2 \right) = 0\\
    & \Rightarrow \sqrt \lambda 2 = n \pi\\
    & \Rightarrow \sqrt \lambda = \frac{n \pi}{2}\\
    & \Rightarrow \lambda = \left(\frac{n \pi}{2}\right)\\
    X(x) & = A \sin\left(\frac{n \pi x}{2}\right)
  \end{align}

  \item Let us solve for $T$:
  %
  \begin{align}
    \frac{T^\prime_n}{3^2 T_n} & = -\lambda_n\\
    T^\prime_n & = -\lambda_n T_n 3^2
  \end{align}

  Here, we want to consider the general form from an expontential. Let us write:
  %
  \begin{align}
    T_n(t) & = e^{- \left( \frac{n \pi 3}{2} \right)^2 t}
  \end{align}

  \item Combine and find $w_n$ and $w$:
  %
  \begin{align}
    w_n(x, t) & =
    \sin\left(\frac{n \pi x}{2}\right)
    e^{- \left( \frac{n \pi 3}{2} \right)^2 t}
  \end{align}

  By linearity,
  %
  \begin{align}
    w(x, t) & =
    \sum^\infty_{n = 1}
    \sin\left(\frac{n \pi x}{2}\right)
    e^{- \left( \frac{n \pi 3}{2} \right)^2 t}
  \end{align}

  Recall the characteristic of our line, $2x + 4$. When we solve for $u(x, 0)$ in terms of $w$, we get $w(x, 0) = 3 \sin(5 \pi x) - 11 \sin(9 \pi x)$:
  %
  \begin{align}
    w(x, 0) & =
    3 \sin(5 \pi x) - 11 \sin(9 \pi x)\\
    w(x, 0) & =
    \sum^\infty_{n = 1}
    A_n \sin\left(\frac{n \pi x}{2}\right)\\
    & = 3\sin(5 \pi x) - 11\sin(9 \pi x)
  \end{align}

  Here, we found $A_{10} = 3$ and $A_{18} = 11$. We can write:
  %
  \begin{align}
    u(x, t) & = 3\sin(5 \pi x)e^{- \left( \frac{n \pi 3}{2} \right)^2 t} - 11\sin(9 \pi x)e^{- \left( \frac{n \pi 3}{2} \right)^2 t} + 2x + 4
  \end{align}
\end{enumerate}

%______________________________________________________________________________%
\newpage
\setcounter{equation}{0}
%______________________________________________________________________________%
  \item Solve $u_{tt} = u_{xx}$ on $x \in [0, 1]$ if $u(0, t) = 5$, $u(1, t) = 2$, $u(x, 0) = x(1 - x) - 3x + 5$ and $u_t(x, 0) = 4$.

  % Begin Answer
  Here, let us write out our given conditions:
  \begin{enumerate}
    \item $x \in [0, 1]$
    \item $u_{tt} = u_{xx}$
    \item $u\ (0, t) = 5$
    \item $u\ (1, t) = 2$
    \item $u\ (x, 0) = x(1 - x) - 3x + 5$
    \item $u_t(x, 0) = 4$
  \end{enumerate}

  Here, similar to the last problem, let us consider $w$ through $T_1$ and $T_1$:
  %
  \begin{align}
    \frac{T_1 - T_2}{L} & = (2 - 5)x + 2\\
    & = 3x - 5
  \end{align}

  Now, let us consider our steady state:
  %
  \begin{align}
    u(0, t) & = 5 \Rightarrow w(0, t) = u(0, t) - u(0, \infty) = 5 - 5 = 0\\
    u(0, t) & = 5 \Rightarrow w(0, t) = u(0, t) - u(0, \infty) = 2 - 2 = 0
  \end{align}

  Now, let us begin:
  %
  \begin{enumerate}
    \item Let us assume $w(x, t) = X(x)T(t)$
    %
    \begin{align}
      XT^{\prime\prime} & = X^{\prime\prime}T\\
      \frac{T^{\prime\prime}}{T} & = \frac{X^{\prime\prime}}{X} = - \lambda
    \end{align}

    \item Let us solve for $X$
    %
    \begin{align}
      X^{\prime\prime} & = - \lambda X
    \end{align}

    Here, let us use the general cosine, sine form:
    %
    \begin{align}
      X(x) & = A \sin \left(\sqrt \lambda x \right) + B \cos \left(\sqrt \lambda x \right)\\
      X(0) = 0 & = B\\
      X(x) & = A \sin \left( \sqrt \lambda x \right)\\
      X(1) = 0 & = A \sin \left( \sqrt \lambda 1 \right)\\
      n \pi & = \sqrt \lambda 1\\
      \sqrt \lambda & = n \pi\\
      \lambda_n & = (n \pi)^2\\
      X_n(x) & = \sin(n \pi x)
    \end{align}

    \item Let us solve for $T_n$
    %
    \begin{align}
      T^{\prime\prime}_n(t) & = - (n \pi)^2T\\
      T_n(t) & = C_n \cos(n \pi t) + D_n \sin(n \pi t)
    \end{align}

    \item Combine and find $w_n$ and $w$

    Here, let us combined our values:
    %
    \begin{align}
      w_n(x, t) & = \sin(n \pi x)\left[ C_n \cos(n \pi t) + D_n \sin(n \pi t) \right]
    \end{align}

    By linearity,
    %
    \begin{align}
      w(x, t) & = \sum^\infty_{n = 1}
      \sin(n \pi x)\left[ C_n \cos(n \pi t) + D_n \sin(n \pi t) \right]\\
      & = \sum^\infty_{n = 1}
      C_n \sin(n \pi x) \cos(n \pi t) + D_n \sin(n \pi x) \sin(n \pi t)
    \end{align}

    \item Let us find the coefficients using the initial condition:
    %
    \begin{align}
      w(x, t) & = \sum^\infty_{n = 1}
      C_n \sin(n \pi x) \cos(n \pi t) + D_n \sin(n \pi x) \sin(n \pi t)\\
      w_t(x, t) & = \sum^\infty_{n = 1}
      -C_n n \pi \sin(n \pi x) \sin(n \pi t) + D_n n \pi \sin(n \pi x) \cos(n \pi t)\\
      w_t(x, 0) & = \sum^\infty_{n = 1}
      D_n n \pi \sin(n \pi x) = 4
    \end{align}

    Here, let us integrate:
    %
    \begin{align}
      D_n n \pi & = 2 \int^1_0 4 \sin(n \pi x) \text{ dx}\\
      D_n & = \frac{8}{n \pi} \int^1_0 \sin(n \pi x) \text{ dx}\\
      & = - \frac{8}{n \pi} \frac{1}{n \pi} \cos(n \pi x) \Big|^1_0\\
      & = - \frac{8}{n^2 \pi^2} \cos(n \pi x) \Big|^1_0\\
      & = - \frac{8}{n^2 \pi^2} (\cos(n \pi) - 1)\\
      & = \frac{8}{n^2 \pi^2} (1 - \cos(n \pi))\\
      & \Rightarrow \frac{8}{n^2 \pi^2} (1 - (-1)^n)
    \end{align}

    Now, let us find $w(x, 0)$:
    %
    \begin{align}
      w(x, t) & = \sum^\infty_{n = 1} C_n \sin(n \pi x) \cos(n \pi t) + \frac{8}{n^2 \pi^2} (1 - (-1)^n) \sin(n \pi x) \sin(n \pi t)\\
      w(x, 0) & = \sum^\infty_{n = 1} C_n \sin(n \pi x) = x - x^2
    \end{align}

    Here, let us integrate:
    %
    \begin{align}
      C_n & = 2 \int^1_0 x \sin(n \pi x) - x^2 \sin(n \pi x)
    \end{align}
  \end{enumerate}

  Let us create our integration tables:
  %
  \begin{center}
    \begin{tabular}{c|c}
      $x$ & $\sin(n \pi x)$\\
      \hline
      $1$ & $-\frac{1}{n \pi} \cos(n \pi x)$\\
      \hline
      $0$ & $-\frac{1}{n^2 \pi^2} \sin(n \pi x)$\\
      &
    \end{tabular}
    \begin{tabular}{c|c}
      $x^2$ & $\sin(n \pi x)$\\
      \hline
      $2x$ & $-\frac{1}{n \pi} \cos(n \pi x)$\\
      \hline
      $2$ & $-\frac{1}{n^2 \pi^2} \sin(n \pi x)$\\
      \hline
      $0$ & $\frac{1}{n^3 \pi^3} \cos(n \pi x)$
    \end{tabular}
  \end{center}

  Here, we have:
  %
  \begin{align}
    C_n & = 2 \left(
    \frac{x}{n \pi} \cos(n \pi x) - \frac{1}{n^2 \pi^2} \sin(n \pi x) -
    \frac{x^2}{n \pi} \cos(n \pi x) + \frac{2x}{n^2 \pi^2} \sin(n \pi x) + \frac{2}{n^3 \pi^3} \cos(n \pi x)
    \right)^1_0\\
    & = 2 \left( \frac{1}{n \pi} \cos(n \pi) - \frac{1}{n \pi} \cos(n \pi) - \frac{2}{n^3 \pi^3} \cos(n \pi) + \frac{2}{n^3 \pi^3} \right)\\
    & = 2 \left( - \frac{2}{n^3 \pi^3} \cos(n \pi) + \frac{2}{n^3 \pi^3} \right)\\
    & = 4 \left(\frac{1 - \cos(n \pi)}{n^3 \pi^3} \right)\\
    & = \left(\frac{4 - 4\cos(n \pi)}{n^3 \pi^3} \right)
  \end{align}

  Now, our heat equation is:
  %
  \begin{align}
    u(x, t) & =
    \sum^\infty_{n = 1} \frac{4 - 4\cos(n \pi)}{n^3 \pi^3} \sin(n \pi x) \cos(n \pi t) + \frac{8}{n^2 \pi^2} (1 - (-1)^n) \sin(n \pi x) \sin(n \pi t) - 3x + 5
  \end{align}
%______________________________________________________________________________%
\newpage
\setcounter{equation}{0}
%______________________________________________________________________________%
  \item Solve $\Delta u = 0$ on $x^2 + y^2 \leq 25$, where $u(5, \theta) = 7 \sin(3 \theta) - 6 \sin(8 \theta)$ and $u$ is bounded when $r = 0$.

  Here, let us consider $x^2 + y^2 \leq 25$. From our assumptions, we know $r$ is bounded between $[0, 5]$.

  Here, we have $u_{xx} + u_{yy} = 0$. First, let us write our $u_x$:
  %
  \begin{align}
    u_x & = \frac{\p u}{\p r} \frac{\p r}{\p x} + \frac{\p u}{\p \theta} \frac{\p \theta}{\p x}\\
    u_x & = u_r \cos \theta - u_\theta \frac{\sin \theta}{r}
  \end{align}

  Here, let us write our $u_{xx}$ and $u_{yy}$ in terms of polar coordinates:
  %
  \begin{align}
    u_{xx} & = u_{rr} \cos^2 \theta
    - 2 u_{\theta r} \frac{\sin \theta \cos \theta}{r}
    + 2 u_\theta \frac{\sin \theta \cos \theta}{r^2}
    + u_r \frac{\sin^2 \theta}{r} + u_{\theta \theta} \frac{\sin^2 \theta}{r^2}\\
    u_{yy} &
    = u_{rr} \sin^2 \theta
    + 2u_{\theta r} \frac{\sin \theta \cos \theta}{r}
    - 2u_\theta \frac{\sin \theta \cos \theta}{r^2}
    + u_r \frac{\cos^2 \theta}{r}
    + u_{\theta \theta} \frac{\cos^2 \theta}{r^2}\\
    \Delta u & = u_{xx} + u_{yy}\\
    \Delta u & = u_{rr} + \frac{u_r}{r} + u_{\theta\theta}{r^2} = 0
  \end{align}

  Here, let us consider our inner and outer boundary:
  %
  \begin{enumerate}
    \item Assume $u(r, \theta) = R(r)\Theta(\theta)$
    %
    \begin{align}
      R^{\prime\prime} \Theta + \frac{R^\prime \Theta}{r} + \frac{R\Theta^{\prime\prime}}{r^2} & = 0\\
      r^2 \frac{R}{R^{\prime\prime}} + r \frac{R^\prime}{R} & = - \frac{\Theta^{\prime\prime}}{\Theta} = \lambda
    \end{align}
    \item Here, let us solve for $\Theta$:
    %
    \begin{align}
      \Theta^{\prime\prime} & = - \lambda \Theta
    \end{align}

    If $\lambda > 0$, then
    %
    \begin{align}
      \Theta(\theta) & =
      A \sin(\sqrt \lambda \theta) + B \cos(\sqrt \lambda \theta)\\
      %%
      \Theta^\prime(\theta) & =
      A \sqrt \lambda \cos ( \sqrt \lambda \theta) - B \sqrt \lambda \sin (\sqrt \lambda \theta)\\
      %%
      \sqrt \lambda 2 \pi = 2 n \pi \Rightarrow \lambda_n = n^2, n \in \Z^+ &
      \begin{cases}
        \Theta(0) = \Theta(2\pi) & \Rightarrow
        B = A \sin(\sqrt \lambda 2 \pi) + B \cos(\sqrt \lambda 2 \pi)\\
        %%
        \Theta^\prime = \Theta^\prime(2 \pi) & \Rightarrow
        A \sqrt \lambda = A \sqrt \lambda \cos (\sqrt \lambda 2 \pi) - B \sqrt \lambda \sin (\sqrt \lambda 2 \pi)
      \end{cases}\\
      %%
      = n^2 & \Rightarrow
      \Theta(n)(\theta) = A_n \sin(n \theta) + B_n \cos(n \theta)
    \end{align}
    If $\lambda = 0$, then the second derivative is $0$.
    %
    \begin{align}
      \Theta^{\prime\prime}_0 & \Rightarrow
      \Theta_0(\theta) = A_0\Theta + B_0\\
      & \Rightarrow \Theta^\prime_0 (\theta) = A_0\\
      & \Rightarrow \Theta_0(0) = \Theta_0(2 \pi) \Rightarrow B_0 = 2 \pi A_0 + B_0 \Rightarrow A_0 = 0\\
      %%
      & \Rightarrow \Theta^\prime_0(0) = \Theta^\prime_0 (2 \pi) = 0
    \end{align}
    \item Next, we solve for $R$:
    %
    \begin{align}
      r^2 \frac{R^{\prime\prime}_n}{R_n} + r\frac{R^\prime_n}{R_n} & = \lambda_n
    \end{align}

    Here, let us consider the following homogeneous equation of our equation:
    %
    \begin{align}
      r^2 R^{\prime\prime}_n + rR^\prime_n - n^2 R_n = 0\\
    \end{align}

    Try $R_n(r) = R^m$, then
    %
    \begin{align}
      r^2 m( m - 1) r^{m - 2} + r mr^{m - 1} - n^2 r^m & = 0\\
      r^m \left[ m( m - 1) + m - n^2 \right] & = 0\\
      m^n - n^2 & = 0\\
      m & = \pm n
    \end{align}
    Next, let us write:
    %
    \begin{align}
      & \Rightarrow
      \begin{cases}
        R_n(r) & = C_n r^n + D_n r^{-n}, n \in \Z^+\\
        R_0(r) & = C_0 + D_0 \ln r
      \end{cases}
    \end{align}

    Recall our interval for $r$ is $[0, 5]$.
    \item Combine to find $u_n$ and $u$:
    %
    \begin{align}
      u_n(r, \theta) & =
      \begin{cases}
        B_0( C_0 + D_0 \ln r) & n = 0\\
        C_nr^n + D_n r^{-n}\left(A_n \cos(n \theta) + B_n \cos(n \theta) \right) & n \in Z^+
      \end{cases}
    \end{align}

    By linearity,
    %
    \begin{align}
      u(r, \theta) & = c_0 + d_0 \ln r + \sum^\infty_{n = 1} (a_nr^n + b_nr^{-n}) \sin(n \theta) + (c_nr^n + d_nr^{-n}) \cos(n \theta)
    \end{align}

    \item Next, let us find the coefficients using our boundary condition:
    %
    \begin{align}
      u(5, \theta) & = 7 \sin(3 \theta) - 6\sin(8 \theta)
    \end{align}

    Now, let us write:
    %
    \begin{align}
      u(5, \theta) & = c_0 + d_0 \ln 5 + \sum^\infty_{n = 1} (a_n 5^n + b_n r^{-n}) \sin(n \theta) + (c_n 5^n + d_n 5^{-n}) \cos(n \theta)
    \end{align}

    Here, for our coefficients, let us write:
    %
    \begin{align}
      \begin{cases}
        c_0 + d_0 \ln 5 & = 0\\
        c_n 5^n + d_n 5^{-n} & = 0\ \forall n\\
        a_3 5^3 + b_3 5^{-3} & = 7, n = 3\\
        a_8 5^8 + b_8 5^{-8} & = -6, n = 8\\
        a_n 5^n + b_n 5^{-n} & = 0\ \forall n, n \neq 3, 8
      \end{cases}
    \end{align}

    If $n \neq 5$:
    %
    \begin{align}
      c_0 + d_0 \ln 5 = 0 & \Rightarrow c_0 = d_0 = 0\\
      c_n + d_n = 0 & \Rightarrow c_0 = d_0 = 0
    \end{align}

    If $n \neq 3, 8$:
    %
    \begin{align}
      a_n 5^n + b_n 5^{-n} = 0 & \Rightarrow a_n = b_n = 0
    \end{align}

    If $n = 3$
    %
    \begin{align}
      a_3 5^3 + b_3 5^{-3} = 7
    \end{align}

    If $n = 8$
    %
    \begin{align}
      a_8 5^8 + b_8 5^{-8} = -6
    \end{align}

    From here, let us write $u$:
    %
    \begin{align}
      u(r, \theta) & = \sum^\infty_{n = 1}
      (a_n 5^n + b_n 5^{-n}) \sin(n \theta) +
      7 \sin(3 \theta) - 6 \sin(8 \theta)
    \end{align}
  \end{enumerate}
\end{enumerate}
\end{document}
