\item Solve $u_t = 9u_{xx}$ on $x \in [0, 2]$ if $u(0, t) = 4$, $u(2, t) = 8$ and $u(x, 0) = 3 \sin(5 \pi x) - 11 \sin(9 \pi x) + 2x + 4$

% Begin Answer
Here, let us write out our given conditions:
\begin{enumerate}
  \item $x \in [0, 2]$
  \item $u_t = 9u_xx$
  \item $u(0, t) = 4$
  \item $u(2, t) = 8$
  \item $u(x, 0) = 3\sin(5 \pi x) - 11\sin(9 \pi x) + 2x + 4$
\end{enumerate}

Let us consider general boundaries, as $u$ does not start and end at $0$. We have $T_1 = 4$ and $T_2 = 8$.

Now, our line can be described as the following:
%
\begin{align}
  \frac{8 - 4}{2}x + 4\\
  2x + 4
\end{align}

Here, let us solve for $u(x, t) = w(x, t) + u(x, \infty)$. To begin, let us consider our steady state condition as well:
%
\begin{align}
  u(0, t) = 4 & \Rightarrow w(0, t) = u(0, t) - u(0, \infty) = 4 - 4 = 0\\
  u(2, t) = 8 & \Rightarrow w(2, t) = u(2, t) - u(2, \infty) = 8 - 8 = 0
\end{align}

Now, let us plug in our $x$ into our steady-state solution and get the next two solutions:
%
\begin{enumerate}
\item Assume $w(x, t) = X(x)T(t)$
%
\begin{align}
  XT^\prime & = 9X^{\prime\prime}T\\
  \frac{T^\prime}{9T} & = \frac{X^{\prime\prime}}{X} = -\lambda
\end{align}

\item Here, let us solve for $X$:
%
\begin{align}
  \frac{X^{\prime\prime}}{X} & = -\lambda\\
  X^{\prime\prime} & = -\lambda X
\end{align}

Here, we want to use the general cosine and sine form:
%
\begin{align}
  X(x) & = A \sin \left( \sqrt \lambda x \right) + B \cos \left( \sqrt \lambda x \right)
\end{align}

Here, let us write our conditions:
%
\begin{align}
  X(0) & = B = 0\\
  X(x) & = A \sin \left( \sqrt \lambda x \right)\\
  X(2) & = A \sin \left( \sqrt \lambda 2 \right) = 0\\
  & \Rightarrow \sqrt \lambda 2 = n \pi\\
  & \Rightarrow \sqrt \lambda = \frac{n \pi}{2}\\
  & \Rightarrow \lambda = \left(\frac{n \pi}{2}\right)\\
  X(x) & = A \sin\left(\frac{n \pi x}{2}\right)
\end{align}

\item Let us solve for $T$:
%
\begin{align}
  \frac{T^\prime_n}{3^2 T_n} & = -\lambda_n\\
  T^\prime_n & = -\lambda_n T_n 3^2
\end{align}

Here, we want to consider the general form from an expontential. Let us write:
%
\begin{align}
  T_n(t) & = e^{- \left( \frac{n \pi 3}{2} \right)^2 t}
\end{align}

\item Combine and find $w_n$ and $w$:
%
\begin{align}
  w_n(x, t) & =
  \sin\left(\frac{n \pi x}{2}\right)
  e^{- \left( \frac{n \pi 3}{2} \right)^2 t}
\end{align}

By linearity,
%
\begin{align}
  w(x, t) & =
  \sum^\infty_{n = 1}
  \sin\left(\frac{n \pi x}{2}\right)
  e^{- \left( \frac{n \pi 3}{2} \right)^2 t}
\end{align}

Recall the characteristic of our line, $2x + 4$. When we solve for $u(x, 0)$ in terms of $w$, we get $w(x, 0) = 3 \sin(5 \pi x) - 11 \sin(9 \pi x)$:
%
\begin{align}
  w(x, 0) & =
  3 \sin(5 \pi x) - 11 \sin(9 \pi x)\\
  w(x, 0) & =
  \sum^\infty_{n = 1}
  A_n \sin\left(\frac{n \pi x}{2}\right)\\
  & = 3\sin(5 \pi x) - 11\sin(9 \pi x)
\end{align}

Here, we found $A_{10} = 3$ and $A_{18} = 11$. We can write:
%
\begin{align}
  u(x, t) & = 3\sin(5 \pi x)e^{- \left( \frac{n \pi 3}{2} \right)^2 t} - 11\sin(9 \pi x)e^{- \left( \frac{n \pi 3}{2} \right)^2 t} + 2x + 4
\end{align}
\end{enumerate}
