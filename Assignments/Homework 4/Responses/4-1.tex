\item Solve $\Delta u = 0$ on $x \in [0, 2]$, $y \in [0, 5]$, with $u_x(0, y) = \cos(3 \pi y)$, $u_x(2, y) = 0$, $u_y(x, 0) = \sin(\pi x)$, $u_y(x, 5) = 0$ and $u(0, 0) = 3$.

% Begin Answer
Here, let us write out our given conditions (Neumann):
\begin{enumerate}
  \item $x \in [0, 2], y \in [0, 5]$
  \item $\Delta u = 0 \Rightarrow u_{xx} + u_{yy} = 0$
  \item $u_x(0, y) = \cos(3 \pi y)$
  \item $u_x(2, y) = 0$
  \item $u_y(x, 0) = \sin(\pi x)$
  \item $u_y(x, 5) = 0$
  \item $u\ (0, 0) = 3$
\end{enumerate}

Since we have $c)$ and $e)$, let us consider $u_1$ and $u_2$:
%

\hrulefill

\begin{multicols}{2}
  Let us consider the following conditions for $u_1$
  %
  \begin{enumerate}
    \item $\Delta u_1 = 0 \Rightarrow u_{1xx} + u_{1yy} = 0$
    \item $u_{1x}(0, y) = 0$
    \item $u_{1x}(2, y) = 0$
    \item $u_{1y}(x, 0) = \sin(\pi x)$
    \item $u_{1y}(x, 5) = 0$
  \end{enumerate}

  Now, let us consider the following conditions for $u_2$\\
  %
  \begin{enumerate}
    \item $\Delta u_1 = 0 \Rightarrow u_{2xx} + u_{2yy} = 0$
    \item $u_{2x}(0, y) = \cos(3 \pi y)$
    \item $u_{2x}(2, y) = 0$
    \item $u_{2y}(x, 0) = 0$
    \item $u_{2y}(x, 5) = 0$
  \end{enumerate}
\end{multicols}
%
\hrulefill

Now, let us begin solving for our equation.
%
\begin{enumerate}
  % First Step
  \item First, let us assume our equation is separable.
  %
  \begin{align}
    u_{xx} + u_{yy} & = 0\\
    u_{xx} & = -u_{yy}\\
    X^{\prime\prime}Y & = -XY^{\prime\prime}\\
    \frac{X^{\prime\prime}}{X} & = -\frac{Y^{\prime\prime}}{Y} = -\lambda
  \end{align}

  From here, let us solve for $u_1$ and $u_2$, starting with $u_1$:
  % Second Step - Pick X
  \item Here, let us solve for $X$. First, let us consider that $\lambda \geq 0$, so we want to break our step into two cases: $\lambda = 0$ and $\lambda > 0$.
  %
  \begin{enumerate}
    \item Let us consider $\lambda > 0$:
  \begin{align}
    \frac{X^{\prime\prime}}{X} & = -\lambda\\
    X^{\prime\prime} & = -\lambda X
  \end{align}

  Here, the general equation for this form is sine and cosine:
  %
  \begin{align}
    X(x) & = A \cos(\sqrt \lambda x) + B \sin(\sqrt \lambda x)
  \end{align}

  Now, since we have information on $u_{1x}$, let us find the first derivative:
  %
  \begin{align}
    X^\prime(x) & = -A \sqrt \lambda \sin(\sqrt \lambda x) + B \sqrt \lambda \cos(\sqrt \lambda x)
  \end{align}

  Now, let us solve for $u_{1x}(0, y)$
  %
  \begin{align}
    X^\prime(0) & = -A \sqrt \lambda \sin(0) + B \sqrt \lambda \cos(0)\\
    X^\prime(x) & = A \sqrt \lambda = 0\\
    & = B = 0
  \end{align}

  Now, we have:
  %
  \begin{align}
    X^\prime(x) & = A \sqrt \lambda \sin(\sqrt \lambda x)\\
    X^\prime(2) & = A \sqrt \lambda \sin(\sqrt \lambda 2) = 0\\
    & = \sqrt \lambda 2 = n \pi\\
    & = \sqrt \lambda = \frac{n \pi}{2}\\
    & = \lambda_{n} = \left(\frac{n \pi}{2}\right)^2
  \end{align}

  Here, we have:
  \begin{align}
    X_{n}(x) & = A \cos\left(\frac{n \pi x}{2}\right)
  \end{align}
  %____________________________________________________________________________%
  \item Let us consider $\lambda = 0$:
  %
  \begin{align}
    \frac{X^{\prime\prime}}{X} & = 0\\
    X^{\prime\prime} & = 0
  \end{align}

  Here, we are looking for a function where our second derivative is $0$. We can use the general form of a line in this case:
  %
  \begin{align}
    X_1(x) & = m x + \alpha
  \end{align}

  From here, let us use our initial condition:
  %
  \begin{align}
    X_{x}(x) & = m\\
    X_{x}(0) & = m = 0\\
  \end{align}

  Here, we have $m = 0$. Therefore, let us write:
  %
  \begin{align}
    X(x) & = \alpha
  \end{align}

  Now, we are left with a constant.
\end{enumerate}

\item Now, let us solve for $Y$. Once again, let us consider the two cases for $\lambda$:
%
\begin{enumerate}
  \item $\lambda > 0$:
  %
  \begin{align}
    \frac{Y^{\prime\prime}}{Y} & = \lambda
  \end{align}

  Here, we must use sinh and cosh and shift our variable:
  %
  \begin{align}
    Y(y) & = C \cosh(\sqrt \lambda (5 - y) + D \sinh(\sqrt \lambda (5 - y))
  \end{align}

  Now, let us take the first derivative:
  %
  \begin{align}
    Y^\prime(y) & = -C \sqrt \lambda \sinh (\sqrt \lambda (5 - y)) + D \sqrt \lambda \cosh(\sqrt \lambda (5 - y))
  \end{align}

  Here, let $y = 5$,
  %
  \begin{align}
    Y^\prime(5) & = +D \sqrt \lambda = 0\\
    & = D = 0
  \end{align}

  Now, let us write again:
  %
  \begin{align}
    Y(y) & = C \sinh(\sqrt \lambda (5 - y))
  \end{align}

  Then let us input our $\lambda$:
  %
  \begin{align}
    Y_{n}(y) & = C \cosh\left(\frac{n \pi (5 - y)}{2}\right)
  \end{align}
  %____________________________________________________________________________%
  \item Next let us consider $\lambda = 0$:
  %
  \begin{align}
    Y^{\prime\prime} & = 0
  \end{align}

  Using this form, we can write the form of a general line:
  %
  \begin{align}
    Y(y) & = nx + \beta
  \end{align}
  Similar to $X$, we will derive a constant for $Y$:
  %
  \begin{align}
    Y(y) & = \beta
  \end{align}

\end{enumerate}

\item Now, if we combine our functions, we can write:
%
\begin{align}
  u_{1n}(x, y) & = \alpha + \beta +
  A\cos\left(\frac{n \pi x}{2}\right)
  \cosh\left(\frac{n \pi(5 - y)}{2}\right)
\end{align}

By linearity, let us write:
%
\begin{align}
  u_{1}(x, y) & = \alpha + \beta + \sum^\infty_{n = 1}
  A\cos\left(\frac{n \pi x}{2}\right)
  \cosh\left(\frac{n \pi(5 - y)}{2}\right)
\end{align}

\item Now, let us find our coefficient. Here, let us find our $y$ partial of $u_1$:
%
\begin{align}
  u_{1y}(x, y) & = \sum^\infty_{n = 1} -\left(\frac{n \pi}{2}\right)\cos\left(\frac{n \pi x}{2}\right) \sinh\left( \frac{n \pi(5 - y)}{2} \right)\\
  u_{1y}(x, 0) & = \sum^\infty_{n = 1} -\left(\frac{n \pi}{2}\right)\cos\left(\frac{n \pi x}{2}\right) \sinh\left( \frac{n \pi 5}{2} \right) = \sin(\pi x)
\end{align}

Note that we do not have a Fourier Sine Series, rather a Fourier Cosine Series. Here, let us find the integral:
%
\begin{align}
  - A_n \frac{n \pi}{2} \sinh\left( \frac{n \pi 5}{2} \right)
  & = \frac{2}{2} \int^2_0 \cos\left(\frac{n \pi x}{2}\right)\sin(\pi x)\\
  - A_n n \pi \sinh\left( \frac{n \pi 5}{2} \right)
  & =
  2 \int^2_0 \cos\left(\frac{n \pi x}{2}\right)\sin(\pi x)
\end{align}

Here, let us use our trig identity to separate our product:
%
\begin{align}
  - A_n n \pi \sinh\left( \frac{n \pi 5}{2} \right)
  & = 2 \int^2_0
  \cos\left(\frac{n \pi x}{2}\right)\sin(\pi x)
  \text{ dx}\\
  %
  & = \frac{2}{2} \int^2_0
  \sin\left(\pi x - \frac{\pi n x}{2}\right) + \sin\left(\pi x +
  \frac{\pi n x}{2}\right)
  \text{ dx}\\
  & = \int^2_0
  \sin\left(\frac{2 \pi x - n \pi x}{2}\right) +
  \int^2_0 \sin\left(\frac{2
  \pi x + n \pi x}{2}\right)
  \text{ dx}\\
  %__________________________________________________________________%
  % \self! I really don't feel like looking at long numbers so just
  % try to make substitutions even if it's tedious --
  % Keep below for reference then delete vvv
  %& = \int^2_0
  %\sin\left(\frac{2 \pi x - n \pi x}{2}\right) + \sin\left(\frac{2
  %\pi x + n \pi x}{2}\right)
  %\text{ dx}\\
  %%
  %& = \int^2_0
  %\sin\left(\frac{2 \pi- n \pi}{2} x \right) + \sin\left(\frac{2 \pi
  %+ n \pi}{2} x \right)
  %\text{ dx}\\
  %%
  %& = - \frac{2}{2 \pi - n \pi} \cos\left(\frac{2\pi - n\pi}{2}
  %x\right) - \frac{2}{2 \pi + n \pi} \cos\left( \frac{2 \pi + n
  %\pi}{2} x\right) \Bigg|^2_0
\end{align}

Here, let us create two substitutions:
%
\begin{align}
  u & = \frac{2 \pi - n \pi}{2}x\\
  du & = \frac{2 \pi - n \pi}{2} \text{ dx}\\
  du \frac{2}{2 \pi - n} & = \text{ dx}
\end{align}
 and
\begin{align}
  s & = \frac{2 \pi + n \pi}{2} x\\
  ds & = \frac{2 \pi + n \pi}{2}\\
  ds \frac{2}{2 \pi + n \pi} & = \text{ dx}
\end{align}

In addition, let us change the integral limits accordingly:
%
\begin{align}
  & =
  \frac{2}{2 \pi - n \pi} \int^{2\pi - n\pi}_0 \sin(u) \text{ du} +
  \frac{2}{2 \pi + n \pi} \int^{2\pi + n\pi}_0 \sin(s) \text{ ds}
\end{align}

Continue with the integration,
%
\begin{align}
  & = - \frac{2}{(2 - n) \pi} \cos(u) \Big|^{(2 - n) \pi}_0
  - \frac{2}{(2 + n) \pi} \cos(s) \Big|^{(2 + n) \pi}_0\\
  & =
  - \frac{2}{(2 - n) \pi}
  \left[
    \cos((2 - n)\pi) - 1
  \right]
  - \frac{2}{(2 + n)\pi}
  \left[
    \cos((2 + n)\pi) - 1
  \right]\\
  & =
  \frac{2}{(n - 2) \pi}
  \left[
    \cos((2 - n)\pi) - 1
  \right]
  - \frac{2}{(2 + n)\pi}
  \left[
    \cos((2 + n)\pi) - 1
  \right]
\end{align}

Here, let us take advantage of the even and periodic properties of cosine:
%
\begin{align}
  & =
  \frac{2}{(n - 2) \pi}
  \left[
    \cos((2 - n)\pi) - 1
  \right]
  - \frac{2}{(2 + n)\pi}
  \left[
    \cos((2 + n)\pi) - 1
  \right]\\
  & =
  \frac{2}{(n - 2) \pi}
  \left[
    \cos(-n\pi) - 1
  \right]
  - \frac{2}{(2 + n)\pi}
  \left[
    \cos(n\pi) - 1
  \right]\\
  & =
  \frac{2}{(n - 2) \pi}
  \left[
    \cos(n\pi) - 1
  \right]
  - \frac{2}{(n + 2)\pi}
  \left[
    \cos(n\pi) - 1
  \right]\\
  & =
  \frac{8}{(n + 2)(n - 2) \pi}
  \left[
    \cos(n\pi) - 1
  \right]\\
  & =
  \frac{8 (\cos (n \pi) - 1)}{(n + 2)(n - 2) \pi}
\end{align}

Let us plug in our left side from $41$:
%
\begin{align}
  -A_n n \pi \sinh\left(\frac{n \pi 5}{2}\right) & =
  \frac{8 (\cos (n \pi) - 1)}{(n + 2)(n - 2) \pi}\\
  A_n & =
  - \frac{1}{n \pi \sinh \left(\frac{n \pi 5}{2}\right)}
  \frac{8 (\cos (n \pi) - 1)}{(n + 2)(n - 2) \pi}\\
  A_n & =
  - \frac{1}{n \pi \sinh \left(\frac{n \pi 5}{2}\right)}
  \frac{8 (1 - \cos (n \pi))}{(n + 2)(n - 2) \pi}\\
  A_n & =
  - \frac{1}{n \pi \sinh \left(\frac{n \pi 5}{2}\right)}
  \frac{8 (1 - (-1)^n)}{(n + 2)(n - 2) \pi}
\end{align}


\hrulefill

\setcounter{enumii}{1}
\item Let us go back and solve for $u_2(x, y)$, starting with $Y$. Consider our separable equation:
%
\begin{align}
  \frac{X^{\prime\prime}}{X} & = - \frac{Y^{\prime\prime}}{Y}\\
  \frac{Y^{\prime\prime}}{Y} & = - \frac{X^{\prime\prime}}{X} = -\lambda\\
\end{align}

Here, we can perform the same series of steps to solve for $Y$ in $u_2$ as we solved for $X$ in $u_1$, swapping our $L$ from $2$ to $5$ in our new case.
%
\begin{align}
  \lambda_{n} & = \left(\frac{n \pi}{5}\right)^2\\
  Y(y) & = A \cos\left(\frac{n \pi y}{5}\right)
\end{align}

Recall we also investigated the case where $\lambda = 0$, which gave us a constant. When $\lambda = 0$, we had the general form:
%
\begin{align}
  Y(y) & = hy + \mu
\end{align}

Which gave us a constant of $\mu$ at the end.

\item Similarly, let us write the solution for $X$ in $u_2$ as we solved for $Y$ in $u_1$:
%
\begin{align}
  X_n(x) & = C \cosh\left(\frac{n \pi(2 - x)}{5}\right)
\end{align}

Similar to the previous item in the list, recall we investigated $\lambda = 0$. In this case, it would give us:
%
\begin{align}
  X(x) & = kx + \nu
\end{align}

Leaving us with a constant $\nu$
\item Again, let us combine $u_2$ and $u_{2n}$:
%
\begin{align}
  u_{2n}(x, y) & = \mu + \nu + \cos\left(\frac{n \pi y}{5}\right) \cosh\left(\frac{n \pi(2 - x)}{5}\right)
\end{align}

By linearity,
%
\begin{align}
  u_{2}(x, y) & = \mu + \nu + \sum^\infty_{n = 1} A \cos\left(\frac{n \pi y}{5}\right) \cosh\left(\frac{n \pi(2 - x)}{5}\right)
\end{align}

Let us combine the constants to $\aleph$
\begin{align}
  u_{2}(x, y) & = \aleph + \sum^\infty_{n = 1} A_n \cos\left(\frac{n \pi y}{5}\right) \cosh\left(\frac{n \pi(2 - x)}{5}\right)
\end{align}


\item Here, Let us look at our condition:
%
\begin{align}
  u_{2}(x, y) & = \aleph + \sum^\infty_{n = 1} A_n\cos\left(\frac{n \pi y}{5}\right) \cosh\left(\frac{n \pi(2 - x)}{5}\right)\\
  u_{2x}(x, y) & =
  \sum^\infty_{n = 1}
  - \frac{5}{n \pi} A_n\cos\left(\frac{n \pi x}{2}\right)
  \sinh\left(\frac{n \pi(2 - x)}{5}\right)\\
  u_{2x}(0, y) & =
  \sum^\infty_{n = 1}
  - \frac{5}{n \pi} A_n\cos\left(\frac{n \pi x}{2}\right)
  \sinh\left(\frac{n \pi 2}{5}\right) = \cos(3 \pi y)
\end{align}

Here, we have the Fourier Cosine Series, therefore we can write $A_6 = 1$.


\item Here, let us combine our $u_1$ and $u_2$:
%
\begin{align}
  u(x, y) & = \alpha + \aleph + \cos \left(\frac{\pi x}{2}\right) +
  \sum^\infty_{n = 1}
  - \frac{1}{n \pi \sinh \left(\frac{n \pi 5}{2}\right)}
  \frac{8 (1 - (-1)^n)}{(n + 2)(n - 2) \pi} \cos\left(\frac{n \pi x}{2}\right) \cosh\left( \frac{n \pi(5 - y)}{2}\right)
\end{align}

Here, let us $\alpha + \aleph$ to $\delta$. Now, Let us use our final condition: $u(0, 0) = 3$
%
\begin{align}
  u(0, 0) & = \delta + 1
  \sum^\infty_{n = 1}
  - \frac{1}{n \pi \sinh \left(\frac{n \pi 5}{2}\right)}
  \frac{8 (1 - (-1)^n)}{(n + 2)(n - 2) \pi} \cos\left(\frac{n \pi}{2}\right) \cosh\left( \frac{n \pi 5}{2}\right)\\
  & = \delta + 1 = 4\\
  & = \delta = 3
\end{align}

Now, our final equation is:
%
\begin{align}
  u(x, y) & = 3 + \cos \left(\frac{\pi x}{2}\right) +
  \sum^\infty_{n = 1}
  - \frac{1}{n \pi \sinh \left(\frac{n \pi 5}{2}\right)}
  \frac{8 (1 - (-1)^n)}{(n + 2)(n - 2) \pi} \cos\left(\frac{n \pi x}{2}\right) \cosh\left( \frac{n \pi(5 - y)}{2}\right)
\end{align}
\end{enumerate}
